%%%****************************************************************************
%%% $Id: listing.tex,v 1.7 1995/05/04 13:07:55 gerd Exp $
%%%============================================================================
%%% 
%%% This file is part of ProCom.
%%% It is distributed under the GNU General Public License.
%%% See the file COPYING for details.
%%% 
%%% (c) Copyright 1995 Gerd Neugebauer
%%% 
%%% Net: gerd@imn.th-leipzig.de
%%% 
%%%****************************************************************************
\def\RCSrevision{\RCSstrip$Revision: 1.7 $}
\def\RCSdate{\RCSstrip$Date: 1995/05/04 13:07:55 $}

\documentstyle [fleqn,%......... smash mathematics leftward
                fancybox,%...... 
                psulhead,%...... use underlined heading
                pcode,%......... obviously Prolog code
                Doc/gntitle,%... my title layout
                makeidx,%....... yes, we are making an index
                Doc/rcs,%....... decode RCS strings
                idtt,%.......... use | | for \verb
                multicol,%...... 
                dina4%.......... A4 sized paper
                ]{book}%........ This is some kind of a book
\pagestyle{ulheadings}

\headheight=8mm
\parindent=0pt

%% ATTENTION: This file has been generated automatically.
%%            Changes to this file may be overwritten.
%%            Consult INSTALL and Makefile for details.
%%
%% Last created: Thu Jul 6 11:04:26 MET DST 1995
%%      by user: gerd
%%      on host: upsilon
%%
%----------------------------------------------------------
\gdef\Version{1.69}

\gdef\ProTopFilters{%{\catcode`\_=12%
  \Filter{tptp}%
  \Filter{mult_taut_filter}%
  \Filter{mpp}%
  \Filter{tee}%
  \Filter{equality_axioms}%
  \Filter{E_flatten}%
  \Filter{constraints}%
}%}

\gdef\ProTopProvers{%{\catcode`\_=12%
  \Prover{procom}%
  \Prover{pool}%
  \Prover{otter}%
  \Prover{setheo}%
}%}



\author{Gerd Neugebauer}
\title{\ProTop \\\ProCom/CaPrI }
\subtitle{Implementation Description}
\version{\Version}
\edition{\RCSrevision}
\date{\today}
\address{Fachbereich Informatik, Mathematik und Naturwissnschaft
\\	 Hochschule f{\"u}r Technik, Wirtschaft und Kultur, Leipzig (FH)
\\       Postfach 66,
\\       04251 Leipzig (Germany)
\\       Net: gerd@imn.th-leipzig.de
}
%------------------------------------------------------------------------------
\vbadness=10000 %%%%%% I don't want to hear about underfull vboxes
%------------------------------------------------------------------------------
\newcommand\ProTop{ProTop}
\newcommand\ProCom{ProCom}
\newcommand\CaPrI{CaPrI}
\newcommand\eclipse{ECL$^i$PS$^e$}
%------------------------------------------------------------------------------
\newfont\sfi{cmssi10}
\def\FileId#1#2{{\small\it This is revision #2 of {\sfi \PrologFILE{}} written
    by #1.}\bigskip}
%------------------------------------------------------------------------------
\renewcommand\PredicateBoxRule{1pt}     
\renewcommand\PrologRuleWidth{.2pt}
\PrologNumberLinestrue

\renewcommand\PrologModule[2]{%
        \newpage%
        \markboth{The Module {\tt #1}}{The Module {\tt #1}}
        \section{The Module {\tt #1}}}

\renewcommand\PrologFile[2]{}

\newfont\Pfont{cmitt10 scaled 900}
\def\PrologFont{\Pfont}

\def\PredicateIndex#1{\index{#1@{\tt #1}}}

\PrologDialect{eclipse}
%------------------------------------------------------------------------------
\newenvironment{SampleCode}[1]{\begin{center}%
  \begin{minipage}{#1}\xdef\WD{#1}%
    \footnotesize\rule{\WD}{.1pt}\vspace{-\baselineskip}
}{\vspace{-\baselineskip}\rule{\WD}{.1pt}\end{minipage}\end{center}}

\newfont\itt{cmitt10}
%------------------------------------------------------------------------------
\newenvironment{BoxedSample}%
{\begin{Sbox}\begin{minipage}{.8\textwidth}\obeylines\obeyspaces\itt%
}{%
 \end{minipage}\end{Sbox}%
 \begin{center}%
  \cornersize*{3ex}%
  \Ovalbox{\mbox{\TheSbox}}
 \end{center}%
}
%------------------------------------------------------------------------------
\newenvironment{XREF}{%
  \begingroup
  \chapter*{Predicate Cross Reference}
  \def\Spec##1##2{\item[##1]\ \dotfill\ {\sf ##2}\\\raggedright}%
  \def\Used##1{{\scriptsize\sf ##1}\hfill}%
  \columnseprule=.1pt
  \multicolsep=21ptplus4ptminus3pt
  \begin{multicols}{2}%
  \small
  \catcode`_=\active
  \begin{description}%
}{\end{description}%
  \end{multicols}
  \endgroup
}
\makeindex
\begin{document} %%%%%%%%%%%%%%%%%%%%%%%%%%%%%%%%%%%%%%%%%%%%%%%%%%%%%%%%%%%%%%

\maketitle

\small
This documentaion has been typeset with \LaTeX. The following style options
have been used: fancybox, pcode, makeidx, multicol, idtt, dina4. In addition
the program |makeindex| has been used to generate the index.
\normalsize\vfill
Copyright {\copyright} 1995 Gerd Neugebauer
\medskip

Permission is granted to make and distribute verbatim copies of this manual
provided the copyright notice and this permission notice are preserved on all
copies.

% Permission is granted to process this file through TeX and print the
% results, provided the printed document carries a copying permission
% notice identical to this one except for the removal of this paragraph
% (this paragraph not being relevant to the printed manual).

Permission is granted to copy and distribute modified versions of this manual
under the conditions for verbatim copying, provided that the entire resulting
derived work is distributed under the terms of a permission notice identical
to this one.

Permission is granted to copy and distribute translations of this manual into
another language, under the above conditions for modified versions, except
that this permission notice may be stated in a translation approved by the
author.

\newpage %---------------------------------------------------------------------
\WithUnderscore{\tableofcontents}
\newpage %---------------------------------------------------------------------

\chapter*{Introduction}

This document contains the documented source code of the automated theorem
proving system \ProTop/\ProCom/\CaPrI. This system is made up of several
relatively indepenent modules.
\begin{description}
\item[\ProTop] 
  is a shell to run automated theorem provers. Certain tasks are performed by
  \ProTop{} to release the theorem provers from this burden. E.g. measuring
  times and generating a report of the activities are things that can be left
  to \ProTop. A scripting language allows the user to describe several
  experiments and let \ProTop{} carry them out later.

  Currently several theorem provers are availlable under the control of
  \ProTop. These are the external provers Setheo and Otter which are run in a
  separate process. The interpreter pool uses the internal representation of
  \ProTop{} and searches directly for a proof. The compiler \ProCom{}
  generates an intermediate Prolog Program which is run to perform the proof
  search.

\item[\ProCom]
  is a theorem prover based on the PTTP paradigm. This means that an
  intermediate Prolog program is generated which performs the search for a
  proof. \ProCom{} is designed rather general utilizing classical software
  technology like compiler and linker techniques and partial evaluation.
  
\item[\CaPrI]
  is the calculi programming interface to \ProCom. This interface allows the
  user to specify the calculus used for the final prover. Thus it is possible
  to exchange the inference engine easily.
  
\end{description}


\chapter{Toplevel and Report Generator}

\PrologInput{main.pl}
\PrologInput{ProTop/protop.pl}
\PrologInput{ProTop/proof_report.pl}
\PrologInput{ProTop/proof_tree.pl}
\PrologInput{ProTop/report_summary.pl}
\section{Sample configuration file}%
\markboth{ProTop Configuration File}{ProTop Configuration File}
\Listing{ProTop/protop.cfg}

\chapter{System Support}
\PrologInput{System/options.pl}
\PrologInput{System/hook.pl}
%\PrologInput{System/pipe.pl}
\PrologInput{System/message.pl}
\PrologInput{System/time.pl}
\PrologInput{System/eq_member.pl}
\PrologInput{System/maplist.pl}
\PrologInput{System/qsort.pl}
\PrologInput{System/unify.pl}
\PrologInput{System/find_file.pl}
\PrologInput{System/dynamic_module.pl}
\PrologInput{System/os.pl}
\PrologInput{System/proof.pl}
\PrologInput{System/add_arg.pl}
\PrologInput{System/varlist.pl}

\chapter{Parsing and Internal Representation}
\PrologInput{System/op_def.pl}
\PrologInput{System/literal.pl}
\PrologInput{System/matrix.pl}
\PrologInput{System/parse.pl}
\PrologInput{Prepare/complete_goals.pl}
\PrologInput{Prepare/prove.pl}
\section{Sample configuration file}
\Listing{Prepare/prove.cfg}

\chapter{Input Filters}
\PrologInput{System/run_filter.pl}
\PrologInput{Filter/none.pl}
\PrologInput{Filter/tee.pl}
\PrologInput{Filter/mpp.pl}
\PrologInput{Filter/mult_taut_filter.pl}
\PrologInput{Filter/equality_axioms.pl}
\PrologInput{Filter/E_flatten.pl}

\chapter{Using external Provers}
\PrologInput{Otter/otter.pl}
\section{Otter configuration file}
\Listing{Otter/otter.cfg}

\PrologInput{Setheo/setheo.pl}
\section{Setheo configuration file}
\Listing{Setheo/setheo.cfg}

\chapter{Executing Prover Code}
\PrologInput{Pool/pool.pl}
\PrologInput{System/run_prover.pl}

\chapter{ProCom/CaPrI}
\PrologInput{ProCom/capri.pl}
\PrologInput{ProCom/procom.pl}
\PrologInput{ProCom/capricore.pl}
\PrologInput{ProCom/p_body.pl}
\PrologInput{ProCom/p_options.pl}
\PrologInput{ProCom/p_driver.pl}
\PrologInput{ProCom/p_goal.pl}
\PrologInput{ProCom/p_put.pl}
\PrologInput{ProCom/linker.pl}
\PrologInput{ProCom/optimize.pl}
\PrologInput{ProCom/unfold.pl}
\section{ProCom configuration file}
\Listing{ProCom/procom.cfg}


\chapter{ProCom/CaPrI Search Strategies}

%%%****************************************************************************
%%% $Id: search.tex,v 1.4 1995/01/16 21:03:12 gerd Exp $
%%%============================================================================
%%% 
%%% This file is part of ProCom.
%%% It is distributed under the GNU General Public License.
%%% See the file COPYING for details.
%%% 
%%% (c) Copyright 1995 Gerd Neugebauer
%%% 
%%% Net: gerd@imn.th-leipzig.de
%%% 
%%%****************************************************************************

%------------------------------------------------------------------------------
\section{Implementing Search Strategies}


The selection of the search strategy for \ProCom{} can be done with the option
|search|. When  a  search strategy is   selected this way this  influences the
execution  of  an initialization predicate.   This initialization predicate is
called

\Predicate init\_search/0().

This predicate is used from the  module |p__|{\em search}\/ where {\em search}
is the functor of the search strategy.

Let us consider an example. Suppose the option |search| is
|iterative_deepening(1,2,3)|.  This setting is evaluated at the time when
\ProCom{} is compiling a matrix.  The functor of this option is
|iterative_deepening|.  Thus the predicate |init_search/0| is called in the
module with the prefix |p__| added, i.e.\ |p__iterative_deepening|.

This predicate has reading and writing access to all options and some internal
global variables which can be used to adjust the compilation process
accordingly.

\begin{description}
\item [Hint:] Since the arity and the arguments of the option |search| are not
  taken  into account in any way,  the search strategy can  use them for their
  own purposes.  Especially a scheme can  be implemented to use default values
  for  missing arguments,  or issue  an  warning message  when the appropriate
  values are not given.
\end{description}

Each search strategy is declared with  the predicate |define_search/1| in {\sf
  procom.cfg}. This  configuration  file is automatically  generated  from the
{\sf Makefile}.

A checklist for adding a new search strategy is given below.
\begin{itemize}
\item  Write   a   module  |p__|{\em search}     which exports  the  predicate
  |init_search/0|. Place  this  module file  on  the  search  path  of Prolog.

\item  Modify the variable |SUBDIRS| in {\sf Makefile}.

\item Declare the search strategy in {\sf Makefile}. For this purpose the
  value of the Makefile variable |PROCOM_SEARCH| must be adapted.

\item Recompile the system.
\end{itemize}


\PrologInput{ProCom/p__depth_first.pl}
\PrologInput{ProCom/p__iterative_deepening.pl}
\PrologInput{ProCom/p__iterative_inferences.pl}
\PrologInput{ProCom/p__iterative_widening.pl}


\chapter{CaPrI Description Files}

\PrologInput{Capri/extension_procedure.pl}
\PrologInput{Capri/me1.pl}
\PrologInput{Capri/me2.pl}
\PrologInput{Capri/me3.pl}
\PrologInput{Capri/me_para.pl}

\chapter{ProCom Libraries}

\renewcommand\PrologFile[2]{%
        \markboth{The Library {\tt #1}}{The Library {\tt #1}}
        \subsection{The Library {\tt #1}}}

\section{Prolog Initialization}

\PrologInput{ProCom/default/init.pl}
\PrologInput{ProCom/eclipse/init.pl}
\PrologInput{ProCom/quintus/init.pl}

\PrologInput{ProCom/eclipse/module.pl}
\PrologInput{ProCom/quintus/module.pl}


\newpage
\section{Path Management}

\PrologInput{ProCom/default/path.pl}
\PrologInput{ProCom/default/path-simple.pl}
\PrologInput{ProCom/default/path-linear.pl}
\PrologInput{ProCom/eclipse/path-regular.pl}

\newpage
\section{Tracer and Debugger}

\PrologInput{ProCom/default/debugger.pl}
\PrologInput{ProCom/eclipse/debugger.pl}
\PrologInput{ProCom/default/no-debugger.pl}
\PrologInput{ProCom/default/tracer.pl}
\PrologInput{ProCom/eclipse/tracer.pl}

\newpage
\section{As Time Goes By}

\PrologInput{ProCom/default/no-time.pl}
\PrologInput{ProCom/default/time.pl}
\PrologInput{ProCom/eclipse/time.pl}
\PrologInput{ProCom/quintus/time.pl}

\newpage
\section{Unification}

\PrologInput{ProCom/default/unify-no-oc.pl}
\PrologInput{ProCom/eclipse/unify.pl}
\PrologInput{ProCom/quintus/unify.pl}
\PrologInput{ProCom/default/member.pl}

\newpage
\section{Lemmas}

\PrologInput{ProCom/default/lemma.pl}
\PrologInput{ProCom/default/no-lemma.pl}

\newpage
\section{User Interaction}

\PrologInput{ProCom/default/show.pl}
\PrologInput{ProCom/eclipse/show.pl}

\PrologInput{ProCom/default/more.pl}
\PrologInput{ProCom/eclipse/more.pl}

\PrologInput{ProCom/default/proof_limit.pl}
\PrologInput{ProCom/eclipse/proof_limit.pl}

\PrologInput{ProCom/default/proof.pl}
\PrologInput{ProCom/eclipse/proof.pl}


\newpage
\section{Search Strategies}

\PrologInput{ProCom/eclipse/search_df.pl}
\PrologInput{ProCom/eclipse/search_id.pl}
\PrologInput{ProCom/eclipse/search_ii.pl}

\chapter{Misc}%

\Listing{../Makefile}

%%*****************************************************************************
%% $Id: xref.tex,v 1.1 1995/01/11 09:15:45 gerd Exp $
%%*****************************************************************************
%% Author: Gerd Neugebauer
%%-----------------------------------------------------------------------------
\documentstyle[multicol,dina4]{book}

\newenvironment{XREF}{%
  \begingroup
  \chapter*{Predicate Cross Reference}
  \def\Spec##1##2{\item[##1]\ \dotfill\ {\sf ##2}\\\raggedright}%
  \def\Used##1{{\scriptsize\sf ##1}\hfill}%
  \columnseprule=.1pt
  \multicolsep=21ptplus4ptminus3pt
  \begin{multicols}{2}%
  \small
  \catcode`_=\active
  \begin{description}%
}{\end{description}%
  \end{multicols}
  \endgroup
}

\begin{document}%%%%%%%%%%%%%%%%%%%%%%%%%%%%%%%%%%%%%%%%%%%%%%%%%%%%%%%%%%%%%%%


\input xref.ltx

\end{document}%%%%%%%%%%%%%%%%%%%%%%%%%%%%%%%%%%%%%%%%%%%%%%%%%%%%%%%%%%%%%%%%%
%
% Local Variables: 
% mode: latex
% TeX-master: nil
% End: 


\markboth{Index}{Index}
\WithUnderscore{\printindex}

\end{document} %%%%%%%%%%%%%%%%%%%%%%%%%%%%%%%%%%%%%%%%%%%%%%%%%%%%%%%%%%%%%%%%
