%%*****************************************************************************
%% $Id$
%%*****************************************************************************
%%% 
%%% This file is part of ProCom.
%%% It is distributed under the GNU General Public License.
%%% See the file COPYING for details.
%%% 
%%% (c) Copyright 1995 Gerd Neugebauer
%%% 
%%% Net: gerd@imn.th-leipzig.de
%%% 
%%%****************************************************************************

\chapter{Using Setheo within \ProTop}


\section{Overview}

\ProTop{} is able to handle internal and external provers. The capability to
handle external provers is demonstrated by the integration of otter. Otter is
a prover written in C which has to be run as a separate process. Setheo
consists of three programs which have to be run one after the other. The input
matrix is given to the first program. Intermediate files are created by Setheo
to hand on the current representation of the problem. Each program can receive
flags in the command line.

\ProTop{} has to translate to internal representation of the matrix in a form
acceptable by Setheo. The external processes have to be started and finally
the output has to be analyzed and translated into a form handeled by \ProTop.

To start Setheo the option |prover| has to be set to the value |setheo|. Then
the proof can be started as usual. This is shown in the example in
figure~\ref{fig:setheo} which assumes us to be in the \ProTop{} interactive
mode.

\begin{figure}[ht]
\begin{BoxedSample}
 ProTop -> |prover = setheo.|
 ok.
 ProTop -> |prove 'eder1-2'.|
 \% No input filter requested
 \% Reading matrix file eder1-2
 2 clauses read in 0 ms 
 Connection graph: 17 ms 
 complete\_goals.pl compiled traceable 1396 bytes in 0.03 seconds
 inwasm V3.2 Copyright TU Munich (March 25, 1994) 
 command line: /home/setheo/bin/inwasm -cons -verbose eder1-2 
 eder1-2.s generated in  0.00 seconds
 wasm V3.2 Copyright TU Munich (March 25, 1994)
 Command line: /home/setheo/bin/wasm -opt -verbose eder1-2 
 Assembler optimization: 16 labels read, 8 labels output
 eder1-2.hex generated in  0.03 seconds
 \% All negative clauses are considered as goals.
 complete\_goals 0 ms 

 \%\char"7C\ Proof with goal clause 1
 \%\char"7C\      ext
 \%\char"7C\      2 - 1
 ...
 ok.
 ProTop -> 
\end{BoxedSample}
\caption{A session with setheo.}\label{fig:setheo}
\end{figure}



\section{Options for Setheo}

The module setheo provides the following options to adjust its behaviour:

\begin{description}
  \item [setheo:flags] | = [cons,opt,verbose,dr].|
    \\
    This option contains a list of flags for Setheo. a detailed description can
    be found in section \ref{setheo:flags}.

  \item [setheo:executable] | = "setheo".|
    \\
    This option contains the executable UNIX command which starts setheo. If
    this executable can not be found on the search path of the UNIX shell used
    it might be neccesary to specify the full file name --- including the
    complete path.

    This option is initialized from the valued specified in the Makefile
    during the installation process. Usually it should not need modification.

  \item [setheo:tmpdirs] | = [indir,".","/tmp","~"].|
    \\
    This option specifies a list of directories which might be used to create
    intermediate files. The list is tried from left to right to find a
    directory which is writable. The first such directory is used to create
    the intermediate files for setheo.

    The elements of this list are interpreted as directories. They can be
    absolute or relative to the current directory. The tilde |~| can be used
    to refer to the home directory of the user. The special symbol |indir| can
    be used to refer to the directory containing the input file.

  \item [setheo:remove\_tmp\_files] | = off.|
    \\
    At the end of the proof attempt the intermediate files created for the
    communication with setheo are removed if this boolean option is |on|.
    Otherwise they are left unchanged. This can be useful to understand how
    setheo is called and get more details what setheo has done.

\end{description}


\section{Flags of Setheo}\label{setheo:flags}

Setheo can be configured with a large number of flags. Most of those
flags\footnote{Only those flags which might disturb \ProTop{} have been
  disabled.} are usable from within \ProTop. For a detailed discussion of the
flags we refer to the Setheo documentation. The following description only
mentiones a few. The main emphasis is on the way how to use them from within
\ProTop.

Setheo has two kind of flags: boolean flags and numeric flags. The
specifications for those flags is combined in the option |setheo:flags| which
contains a list of those.

The names of the flags are symbols in the sense of Prolog. Thus those names
are used to name the flags in the context of \ProTop{} as well. The Setheo
interface known to which program a flag belongs. Thus all relevant flags for
the programs are used and the others omitted.


\subsection{Boolean Flags}

Boolean flags can be turned on by specifying them on the command line. In
absense they are turned off.

E.g. if the list in |setheo:flags| contains the element |dr| then the Setheo
(|sam|) flag |dr| is set.\footnote{The flag |dr| turns on the iterative
  deepening search on the depth of the proof tree.}


\subsection{Numeric Flags}

Numeric flags can take numeric values. Arbitrary numbers can be assigned to
numeric flags.

To assign a value {\em n}\/ to a numeric flag {\em flag}\/ simply add an
element {\em flag {\tt=} n} to the list of flags |setheo:flags|.

E.g. if the list in |setheo:flags| contains the element |trail=1000000| then
the numeric flag |trail| (of |sam|) is set to the value 1000000.


\begin{figure}[t]
{\scriptsize
\catcode`|=\other
\def\SetheoFlag#1#2#3{\tt #1 \sl #2&#3\\\hline}%
\null\hfill\hfill
\begin{tabular}{|p{10em}c|}\hline \bf Flag & \bf Program \\\hline\hline
  \SetheoFlag{purity}{}{inwasm}
  \SetheoFlag{nopurity}{}{inwasm}
  \SetheoFlag{nofan}{}{inwasm}
  \SetheoFlag{reduct}{}{inwasm}
  \SetheoFlag{noreduct}{}{inwasm}
  \SetheoFlag{nosgreord}{}{inwasm}
  \SetheoFlag{noclreord}{}{inwasm}
  \SetheoFlag{notree}{}{inwasm}
  \SetheoFlag{randreord}{}{inwasm}
  \SetheoFlag{eqpred}{}{inwasm}
  \SetheoFlag{all}{}{inwasm}
  \SetheoFlag{reg}{}{inwasm}
  \SetheoFlag{subs}{}{inwasm}
  \SetheoFlag{taut}{}{inwasm}
  \SetheoFlag{cons}{}{inwasm}
  \SetheoFlag{foldup}{}{inwasm}
  \SetheoFlag{foldupx}{}{inwasm}
  \SetheoFlag{folddown}{}{inwasm}
  \SetheoFlag{folddownx}{}{inwasm}
  \SetheoFlag{partialtree}{}{inwasm}
  \SetheoFlag{nopartialtree}{}{inwasm}
  \SetheoFlag{verbose}{= number}{inwasm}
  \SetheoFlag{verbose}{}{inwasm}
  \SetheoFlag{verbose}{}{wasm}
  \SetheoFlag{opt}{}{wasm}
\end{tabular}\hfill
\begin{tabular}{|p{10em}c|}\hline \bf Flag & \bf Program \\\hline\hline
  \SetheoFlag{d}{= number}{sam}
  \SetheoFlag{d}{}{sam}
  \SetheoFlag{i}{= number}{sam}
  \SetheoFlag{i}{}{sam}
  \SetheoFlag{dr}{= number}{sam}
  \SetheoFlag{dr}{}{sam}
  \SetheoFlag{ir}{= number}{sam}
  \SetheoFlag{ir}{}{sam}
  \SetheoFlag{loci}{}{sam}
  \SetheoFlag{locir}{}{sam}
  \SetheoFlag{anl}{}{sam}
  \SetheoFlag{reg}{}{sam}
  \SetheoFlag{st}{}{sam}
  \SetheoFlag{cons}{}{sam}
  \SetheoFlag{code}{= number}{sam}
  \SetheoFlag{stack}{= number}{sam}
  \SetheoFlag{cstack}{= number}{sam}
  \SetheoFlag{heap}{= number}{sam}
  \SetheoFlag{trail}{= number}{sam}
  \SetheoFlag{symbtab}{= number}{sam}
  \SetheoFlag{seed}{= number}{sam}
  \SetheoFlag{v}{= number}{sam}
  \SetheoFlag{v}{}{sam}
  \SetheoFlag{verbose}{= number}{sam}
  \SetheoFlag{verbose}{}{sam}
\end{tabular}\hfill\hfill\null}

  \caption{The Flags of Setheo}\label{fig:setheo.flags}
\end{figure}

\subsection{List of Flags}

The table \ref{fig:setheo.flags} contains a complet list of Setheo flags as
understood by the \ProTop/Setheo interface.  {\em Program}\/ denotes one of
the setheo programs |inwasm|, |wasm|, or |sam|.  If the flag is followed by
``{\em = number}'' then a numeric argument is required. See the documentation
of Setheo for further details.

%
% Local Variables: 
% mode: latex
% TeX-master: nil
% End: 
