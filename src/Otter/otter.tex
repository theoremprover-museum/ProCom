%%*****************************************************************************
%% $Id: otter.tex,v 0.00 1994/06/14 15:34:29 gerd Exp $
%%*****************************************************************************
%%% 
%%% This file is part of ProCom.
%%% It is distributed under the GNU General Public License.
%%% See the file COPYING for details.
%%% 
%%% (c) Copyright 1995 Gerd Neugebauer
%%% 
%%% Net: gerd@imn.th-leipzig.de
%%% 
%%%****************************************************************************

\chapter{Using Otter within \ProTop}


\section{Overview}

\ProTop{} is able to handle internal and external provers. The capability to
handle external provers is demonstrated by the integration of otter. Otter is
a prover written in C which has to be run as a separate process. Otter gets
its problem specification (the matrix) together with the flags (options) in
one single file. The result consists of informative messages, like runtime,
and the final proof or failure indicators. Those results are also collected in
a file.

\ProTop{} has to translate to internal representation of the matrix in a form
acceptable by otter. The external process has to be started and finally the
output has to be analyzed and translated into a form handeled by \ProTop.

To start otter the option |prover| has to be set to the value |otter|. Then
the proof can be started as usual. This is shown in the example in
figure~\ref{fig:otter} which assumes us to be in the \ProTop{} interactive
mode.

\begin{figure}[ht]
\begin{BoxedSample}
  ProTop -> |prover = otter.|
  ok.
  ProTop -> |prove 'eder1-2'.|
  \% No input filter requested
  \% Reading matrix file Samples/eder1-2
  2 clauses read in 17 ms 
  ...

  time("user CPU time       ",20).
  time("system CPU time     ",70).
  time("wall-clock time     ",0).
  time("hyper\_res time      ",0).
  time("for\_sub time        ",0).
  time("back\_sub time       ",0).
  time("conflict time       ",0).
  time("demod time          ",0).
  ok.
  ProTop -> 
\end{BoxedSample}
\caption{A session with otter.}\label{fig:otter}
\end{figure}



\section{Options for Otter}

The module otter provides the following options to adjust its behaviour:

\begin{description}
  \item [otter:flags] | = [auto,-print_given].|
    \\
    This option contains a list of flags for Otter. a detailed description can
    be found in section \ref{otter:flags}.

  \item [otter:executable] | = "otter".|
    \\
    This option contains the executable UNIX command which starts otter. If
    this executable can not be found on the search path of the UNIX shell used
    it might be neccesary to specify the full file name --- including the
    complete path.

    This option is initialized from the valued specified in the Makefile
    during the installation process. Usually it should not need modification.

  \item [otter:tmpdirs] | = [indir,".","/tmp","~"].|
    \\
    This option specifies a list of directories which might be used to create
    intermediate files. The list is tried from left to right to find a
    directory which is writable. The first such directory is used to create
    the intermediate files for otter.

    The elements of this list are interpreted as directories. They can be
    absolute or relative to the current directory. The tilde |~| can be used
    to refer to the home directory of the user. The special symbol |indir| can
    be used to refer to the directory containing the input file.

  \item [otter:remove\_tmp\_files] | = off.|
    \\
    At the end of the proof attempt the intermediate files created for the
    communication with otter are removed if this boolean option is
    |on|. Otherwise they are left unchanged. This can be useful to understand
    how otter is called and get more details what otter has responded.

\end{description}



\section{Flags of Otter}\label{otter:flags}

Otter can be configured with a large number of flags. Most of those
flags\footnote{Only those flags which might disturb \ProTop{} have been
  disabled.} are usable from within \ProTop. For a detailed discussion of the
flags we refer to \cite{mccune:otter}. The following description only
mentiones a few. The main emphasis is on the way how to use them from within
\ProTop.

Otter has two kind of flags: boolean flags and numeric flags. The
specifications for those flags is combined in the option |otter:flags| which
contains a list of those.

The names of the flags are symbols in the sense of Prolog. Thus those names
are used to name the flags in the context of \ProTop{} as well.

\subsection{Boolean Flags}

Boolean flags can be set (on) or reset (off). If a boolean flag {\em flag}
occurs in the list of flags |otter:flags| then this is interpreted as a
positive occurrence and the corresponding Otter flag is turned on.

E.g. if the list in |otter:flags| contains the element |auto| then the Otter
flag |auto| is set.\footnote{The flag |auto| turns on the analysis automatic
  setting of flags according to the properties of the problem given.}

A negated flag in the list |otter:flags| is marked with a preceeding minus
|-|. Flags marked in this way are turned off in Otter.

E.g. if the list in |otter:flags| contains the element |-print_given| then the
Otter flag |print_given| is reset.\footnote{The flag {\tt print\_given} turns
  on the reporting of input clauses in the output.}


\subsection{Numeric Flags}

Numeric flags can take numeric values. Arbitrary numbers can be assigned to
some of the numeric flags. Other can take only values in a specific
range. Those restrictions are checked in the \ProTop{} interface to Otter.

To assign a value {\em n}\/ to a numeric flag {\em flag}\/ simply add an
element {\em flag {\tt=} n} to the list of flags |otter:flags|.

E.g. if the list in |otter:flags| contains the element |max_kept=1000000| then
the numeric flag |max_kept| is set to the value 1000000.\footnote{The flag
  {\tt max\_kept} contains the maximum number of resolvents kept.}


\subsection{List of Flags}

\begin{figure}[t]
{\scriptsize
\catcode`|=\other
\def\OtterFlag#1#2.{\tt #1&#2\\\hline}%
\null\hfill
\begin{tabular}{|p{15em}c|}\hline \bf Flag & \bf Type \\\hline\hline
  \OtterFlag{atom\_wt\_max\_args	}{boolean}.
  \OtterFlag{auto			}{boolean}.
  \OtterFlag{back\_demod		}{boolean}.
  \OtterFlag{back\_sub			}{boolean}.
  \OtterFlag{binary\_res		}{boolean}.
  \OtterFlag{bird\_print		}{default}.
  \OtterFlag{check\_arity		}{boolean}.
  \OtterFlag{control\_memory		}{boolean}.
  \OtterFlag{delete\_identical\_nested\_skolem}{boolean}.
  \OtterFlag{demod\_history		}{boolean}.
  \OtterFlag{demod\_inf			}{boolean}.
  \OtterFlag{demod\_limit		}{integer}.
  \OtterFlag{demod\_linear		}{boolean}.
  \OtterFlag{demod\_out\_in		}{boolean}.
  \OtterFlag{detailed\_history		}{boolean}.
  \OtterFlag{display\_terms		}{default}.
  \OtterFlag{dynamic\_demod		}{boolean}.
  \OtterFlag{dynamic\_demod\_all	}{boolean}.
  \OtterFlag{dynamic\_demod\_lex\_dep	}{boolean}.
  \OtterFlag{echo\_included\_files	}{boolean}.
  \OtterFlag{eq\_units\_both\_ways	}{boolean}.
  \OtterFlag{factor			}{boolean}.
  \OtterFlag{for\_sub			}{boolean}.
  \OtterFlag{for\_sub\_fpa		}{boolean}.
  \OtterFlag{fpa\_literals		}{integer(0,100)}.
  \OtterFlag{fpa\_terms			}{integer(0,100)}.
  \OtterFlag{free\_all\_mem		}{boolean}.
  \OtterFlag{hyper\_res			}{boolean}.
  \OtterFlag{index\_for\_back\_demod 	}{boolean}.
  \OtterFlag{input\_sos\_first		}{boolean}.
  \OtterFlag{interactive\_given		}{default}.
  \OtterFlag{interrupt\_given		}{default}.
  \OtterFlag{knuth\_bendix		}{boolean}.
  \OtterFlag{lex\_order\_vars		}{boolean}.
  \OtterFlag{lrpo			}{boolean}.
  \OtterFlag{max\_distinct\_vars	}{integer}.
  \OtterFlag{max\_gen			}{integer}.
  \OtterFlag{max\_given			}{integer}.
  \OtterFlag{max\_kept			}{integer}.
  \OtterFlag{max\_literals		}{integer}.
  \OtterFlag{max\_mem			}{integer}.
  \OtterFlag{max\_proofs		}{integer}.
  \OtterFlag{max\_seconds		}{integer}.
  \OtterFlag{max\_weight		}{integer}.
  \OtterFlag{min\_bit\_width		}{integer}.
\end{tabular}\hfill
\begin{tabular}{|p{15em}c|}\hline \bf Flag & \bf Type \\\hline\hline
  \OtterFlag{neg\_hyper\_res		}{boolean}.
  \OtterFlag{neg\_weight		}{integer}.
  \OtterFlag{no\_fanl			}{boolean}.
  \OtterFlag{no\_fapl			}{boolean}.
  \OtterFlag{order\_eq			}{boolean}.
  \OtterFlag{order\_history		}{boolean}.
  \OtterFlag{order\_hyper		}{boolean}.
  \OtterFlag{para\_all			}{boolean}.
  \OtterFlag{para\_from			}{boolean}.
  \OtterFlag{para\_from\_left		}{boolean}.
  \OtterFlag{para\_from\_right		}{boolean}.
  \OtterFlag{para\_from\_units\_only 	}{boolean}.
  \OtterFlag{para\_from\_vars		}{boolean}.
  \OtterFlag{para\_into			}{boolean}.
  \OtterFlag{para\_into\_left		}{boolean}.
  \OtterFlag{para\_into\_right		}{boolean}.
  \OtterFlag{para\_into\_units\_only 	}{boolean}.
  \OtterFlag{para\_into\_vars		}{boolean}.
  \OtterFlag{para\_ones\_rule		}{boolean}.
  \OtterFlag{para\_skip\_skolem		}{boolean}.
  \OtterFlag{pick\_given\_ratio		}{integer}.
  \OtterFlag{pretty\_print		}{default}.
  \OtterFlag{pretty\_print\_indent	}{integer(4,16)}.
  \OtterFlag{print\_back\_demod		}{boolean}.
  \OtterFlag{print\_back\_sub		}{boolean}.
  \OtterFlag{print\_given		}{boolean}.
  \OtterFlag{print\_kept		}{clear}.
  \OtterFlag{print\_lists\_at\_end	}{boolean}.
  \OtterFlag{print\_new\_demod		}{boolean}.
  \OtterFlag{print\_proofs		}{default}.
  \OtterFlag{process\_input		}{boolean}.
  \OtterFlag{prolog\_style\_variables	}{set}.
  \OtterFlag{propositional		}{default}.
  \OtterFlag{really\_delete\_clauses	}{default}.
  \OtterFlag{report			}{integer}.
  \OtterFlag{simplify\_fol		}{boolean}.
  \OtterFlag{sort\_literals		}{boolean}.
  \OtterFlag{sos\_queue			}{boolean}.
  \OtterFlag{sos\_stack			}{boolean}.
  \OtterFlag{stats\_level		}{default}.
  \OtterFlag{symbol\_elim		}{boolean}.
  \OtterFlag{term\_wt\_max\_args	}{boolean}.
  \OtterFlag{unit\_deletion		}{boolean}.
  \OtterFlag{ur\_res			}{boolean}.
  \OtterFlag{very\_verbose		}{default}.
\end{tabular}\hfill\null}

  \caption{The Flags of Otter}\label{fig:otter.flags}
\end{figure}

The table \ref{fig:otter.flags} contains a complet list of Otter flags as
understood by the \ProTop/Otter interface.
See the documentation of Otter \cite{mccune:otter} for further details.

{\em Type} can have one of the following values:

\begin{list}{}{\parsep=0pt\itemsep=0pt\labelwidth=9em\leftmargin=10em}
\item [\bf boolean] denotes a boolean flag. it can be turned on or off.
\item [\bf set] denotes a boolean flag which must be turned on.
\item [\bf clear] denotes a boolean flag which must be turned off.
\item [\bf default] denotes a flag which can not be changed.
\item [\bf integer] denotes a numeric flag which can take arbitrary values.
\item [\bf integer({\it min},{\it max})] denotes a numeric flag which can take
  values in the range from {\it min}\/ to {\it max}. 
\end{list}
