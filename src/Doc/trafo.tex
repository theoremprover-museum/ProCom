%%*****************************************************************************
%% $Id: trafo.tex,v 1.1 1994/10/18 18:42:27 gerd Exp $
%%*****************************************************************************
%% Author: Gerd Neugebauer
%%-----------------------------------------------------------------------------
\documentstyle[11pt,dina4,fleqn,logic]{article}


\def\Proof{\mbox{\it Proof}}
\def\goal#1{\mbox{\it goal#1}}

\begin{document}%%%%%%%%%%%%%%%%%%%%%%%%%%%%%%%%%%%%%%%%%%%%%%%%%%%%%%%%%%%%%%%

\section{Literal Transformations}

To translate a literal into a prolog term we have to consider the following
points:
\begin{itemize}
  \item We have to encode the negation into the predicate name. For safety we
  also encode the original arity.
  \item We have to add the path as additional argument.
  \item We have to add a depth argument.
  \item We have to add an argument to retrun the proof tree.
  \item To cope with a missing occurs check it may turn out necessary to
  separate the variables.
\end{itemize}



\section{Goal Transformations}

When transforming a set of goal clauses into a Prolog program several non
logical features have to be considered. These features are
\begin{itemize}
  \item To ensure a complete search strategy one might consider to iterate
  some kind of depth bound. This depth bound has to be initialized and managed
  appropriately. 
  \item Sometimes it is desirable to measure the proof time. For this purpose
  the timer has to be started and finally the time computed and printed.
  \item If you are not only interested in a simple {\em valid} or {\em
  invalid} we have to collect to answer substitution and maybe the proof and
  present them at the end. Additionally we may want to write this information
  to a protocol file.
  \item Finally it can be desirable to compute more than one solution. This
  can either be done by specifying how much, or which solution we are
  interested in. Alternatively we may implement a user interaction which
  allows us to ask for another solution.
\end{itemize}

Let $C=\{L_1,\ldots,L_m\}$\/ be a set of literals
\begin{Formula}
  \pi(C,D,P,\Proof) := \pi(L_1,D,P,\Proof_1),\ldots,\pi(L_m,D,P,\Proof_m),
  \Proof = (\Proof_1,\ldots,\Proof_m).
\end{Formula}


Let $G=\{G_1,\ldots G_n\}$\/ be a complete set of goals, where 


\begin{Formula}
  \goal{}(F) \IF
  \=	set\_time(T_0),
  \=	goals_0,
  \=	set\_depth(D),
  \=	goals_1(D),
  \=	init\_path(P),
  \=	\goal{\_dis}(D,P,\Proof,F),
  \=	show\_time(T_0,F),
  \=	show\_proof(\Proof,F).
\end{Formula}

\begin{Formula}
  \goal{\_dis}(D,P,\Proof,F) \IF
  \=	\pi(G_i,D,P,\Proof),
  \=	goals_{post}(D,P),
  \=	show\_vars(G_i).
\end{Formula}




\end{document}%%%%%%%%%%%%%%%%%%%%%%%%%%%%%%%%%%%%%%%%%%%%%%%%%%%%%%%%%%%%%%%%%
