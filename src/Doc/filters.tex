%%%****************************************************************************
%%% $Id: filters.tex,v 1.4 1995/04/24 21:29:11 gerd Exp $
%%%============================================================================
%%% 
%%% This file is part of ProCom.
%%% It is distributed under the GNU General Public License.
%%% See the file COPYING for details.
%%% 
%%% (c) Copyright 1995 Gerd Neugebauer
%%% 
%%% Net: gerd@imn.th-leipzig.de
%%% 
%%%****************************************************************************

\section{\ProTop{} Filters}

The first phase of the processing of an input file which is performed by
\ProTop{} consists of the application of a series of input filters. Those
filters take the ASCII representation and transform it. The result is given to
the next filter in the chain.

The filters which are used are set with the option |input_filter| (see
page~\pageref{opt:input_filter}). 






The details of writing your own filters can be found in
appendix~\ref{chap:writing.filters}. 

The following sections describe (some of) the filters provided with \ProTop.


\subsection{The Filter {\tt none}}

The filter {\tt none} is a very simple filter which does simply nothing. It is
mainly provided as a basis for your own development of a filter.


\subsection{The Filter {\tt tee}}

The filter {\tt tee} provides a means to trace or log the data flow in a
filter chain.  If this filter is part of a chain then it acts like the empty
filter.  I.e. the input is passed through. As a side effect it stores
everything passed through also in the file specified by the option
|tee:file|\index{tee:file}.

The value of the option |tee:file| is an atom or string denoting a file name.
Some atoms have a special meaning. |output| and |stdout| denote the output
stream.  |stderr| denotes the standart error stream. If such an reeserved
value is encountered then the associated stream is used instead of opening a
file.  If a file is used then the output is appended to this file.

The option |tee:file| defaults to the value |output|.

Instead of specifying the output file in the option |tee:file| it is possible
to give the name as additional argument to the filter. Thus it is possible to
redirect the output of several incarnations of tee to different files:

The following example logs the input and the output of the filter
|mult_taut_filter| in the two files |file1| and |file2| respectively.

\begin{BoxedSample}
  input\_filter = [tee(file1),mult\_taut\_filter,tee(file2)]
\end{BoxedSample}


\subsection{The Filter {\tt tptp}}

The filter {\tt tptp} provides an interface to the TPTP library (Thousand of
problems for theorem provers). It is written such that files in the TPTP
syntax can be read in. It follows the conventions of the TPTP library. See the
documentation of the TPTP library for details.

This filter is especially useful when you want to experiment with the problems
from the TPTP library. For this purpose this filter should be the first one.

This filter uses the following option:
\begin{description}
\item[tptp:home] This option contains the complete path to the installed TPTP
  library. Usually this options should be set at installation time properly.
\end{description}


\subsection{The Filter {\tt mult\_taut\_filter}}

The filter {\tt mult\_taut\_filter} perform the reductions mult and taut on
the input problem. The filter understands the input syntax of the parser. Thus
it is recommended to use this filter as one of the last ones.

The mult reduction deletes multiple occurences of literals in a clause and
leaves only one instance of such a literal. This filter only removes identical
literals. Unification is not involved. With this restriction this
transormation is equivalence preserving.

The taut reduction deletes clauses which contain a literal as well as its
negated form. Only identical literals --- up to the complementary sign --- are
considered. No unification is involved. With this restriction the
transformation is equivalence preserving.


\subsection{The Filter {\tt mpp}}

The filter {\tt mpp} provides a {\em M}\/acro {\em P}\/re{\em P}\/rocessor
similar to the C preprocessor |cpp|. It is possible to define macros which are
expanded in the remainder of the file. Inclusion of other files and
conditional parts can also be specified. Last but not least it is possible to
define additional operators for syntactic sugar.

The filter {\tt mpp} provides additional meta instructions. Those meta
instructions are evaluated by this filter and noit passed through. The meta
instructions start with the symbol |#|.\footnote{This is defined as a prefix
  operator in \ProTop.} The following meta instructions are evaluated by {\tt
  mpp}: 

\begin{description}
\item [\#include {\em File}.]\ \\
  The contents of file {\em File} is included instead of this instruction as
  if it was there already. Nevertheless only complete clauses can be contained
  in this file. It is not possible to include parts of clauses.

  The included files are searched with the same algorithm as files specified
  as input files. I.e. if they are not absolute then they are searched in the
  directories given in the option
  |input_path|\index{input\_path}. Additionally the extensions stored in the
  option |input_extensions|\index{input\_extensions} are taken into account.

  The following example includes the file {\sf common}:

\begin{BoxedSample}
  \#include "common".
\end{BoxedSample}

\item [\#op({\em Precedence}, {\em Associativity}, {\em Name}\/).]\ \\
  For the remainder of this file the operator {\em Name} with precedence {\em
    Precedence} and associativity {\em Associativity} is defined. See the
  Prolog documentation on |op/3| for details.

  {\bf Note:} This operator is not automatically defined in the next filter!

  It might be a good practice to define a translation rule for any
  operator. Thus it can be translated into a ``normal'' operator not written
  in infix, postfix, or prefix notation. This helps avoiding confusion in the
  following parts of the system.

  The following example defines an operator |==>| which is used
  afterwards. Without the declaration this example would lead to an syntax
  error.

\begin{BoxedSample}
  \#op(900,xfx,(==>)).
  p(X) ==> q(X,f(X)).
  q(11,32).
\end{BoxedSample}

\item [\#define {\em Head} = {\em Tail}.]\ \\
  The macro definition specified in this instruction is stored. This does not
  overwrite previous definitions but is added. In the macro expansion the
  first unifying definition is applied.  The usual Prolog rules for variables
  are used.

  The definitions are applied at every level. This means that they are used at
  the formula level as well as on the term level.

\begin{BoxedSample}
  \#define a = new\_a.
  \#define p(X,new\_a) = pa(X).

  q(A).
  p(X,a).
  a.
\end{BoxedSample}
  The result is
\begin{BoxedSample}
  q(\_g123).
  p(\_g123,new\_a).
  new\_a.
\end{BoxedSample}

  To complete our example started in section on operator declaration we can
  define a translation rule (macro) to translate |==>| terms into the normal
  implication form supported by \ProTop.

\begin{BoxedSample}
  \#op(900,xfx,(==>)).
  \#define (A ==> B) = (B :- A).
  p(X) ==> q(X,f(X)).
\end{BoxedSample}

\item [\#define {\em Head}.]\ \\
  This is the same as defining {\em Head}\/ to 1. This is especially useful in
  combination with conditional inclusion of clauses (see below).

\begin{BoxedSample}
  \#define use\_equality\_axioms
\end{BoxedSample}

\item [\#undef {\em Pattern}.]\ \\
  Since macros are added but not overwritten we need a way to get rid of a
  macro. The undef instruction removes all macros for which the head is
  unifiable with {\em Pattern}. If no matching macro is found nothing is done.

  In the following example two macros are defined and the first one is deleted
  afterwards. 
\begin{BoxedSample}
  \#define mac(1,X) m1(X)
  \#define mac(2,X) m2(X)
  \#undef mac(1,\_)
\end{BoxedSample}
  The following instruction would have deleted both macros:
\begin{BoxedSample}
  \#undef mac(\_,\_)
\end{BoxedSample}

\item [\#if {\em Condition}.]\ \\
  {\em Condition}\/ is evaluated. If it evaluates to true then the block to
  the next matching |#else| or |#endif| is used. The |#else| to |#endif| block
  is ignored. Otherwise the block between the next matching |#else| and the
  |#endif| is used.

\begin{BoxedSample}
  \#if defined(use\_equality\_axioms).
  \#include equality\_axioms.
  \#endif.
\end{BoxedSample}

  They following conditionals are recognized by the |#if| instruction:

  \begin{description}
  \item [defined({\em Pattern}\/)]\ \\
    This condition evaluates to true iff a macro is defined for which the head
    is unifiable with {\em Pattern}.

  \item [not({\em Condition})]\ \\
    This condition evaluates to true iff {\em Condition} does not evaluate to
    true.
  \item [{\em Condition},{\em Condition}]\ \\
    A conjunction evaluates to true iff all conjuncts evaluate to true.

  \item [{\em Condition};{\em Condition}]\ \\
    A disjunction evaluates not to true iff at least one disjunct evaluates
    not to true.
  \end{description}

\item [\#else.]\ \\
  Switch to the alternate block started with a previously matching |#if|,
  |#ifdef|, or |#ifndef|.

\item [\#endif.]\ \\
  This instruction ends a block started with |#if|, |#ifdef|, or |#ifndef|.

\item [\#ifdef {\em Spec}.]\ \\
  This is the same as |#if defined(|{\em Spec}\/|)|.
  
\item [\#ifndef {\em Spec}.]\ \\
  This is the same as |#if not(defined(|{\em Spec}\/|))|.
\end{description}


The filter {\tt mpp} provides the hook |mpp_hook/1| which is called just
before the first clause is read. This hook can be used to add definitions or
include files at startup under the control of a Prolog program. For this
purpose several Prolog procedures are exported. See the implementation
description of {\tt mpp} for details.


\subsection{The Filter {\tt equality\_axioms}}

The filter {\tt equality\_axioms} adds the axioms and axiom schemata needed to
handle the equality. This means that an axiomatization of equality is added
and the proof is carried out without any special knowledge of equality.

The predicate |=/2| is assumed to be the equality predicate. The generated
clauses use this predicate.



\subsection{The Filter {\tt E-flatten}}

The filter {\tt E-flatten} provides an alternative treatment of equality
predicates.


\subsection{The Filter {\tt constraints}}

The filter {\tt constraints} provides a means to divide the matrix into a
matrix with constraints and a Horn-(Prolog) program. An enhanced syntax is
used to specify those parts and their interaction.


