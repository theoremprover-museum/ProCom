%%*****************************************************************************
%% $Id: matrix.tex,v 1.7 1995/04/24 21:29:11 gerd Exp $
%%*****************************************************************************
%% Author: Gerd Neugebauer
%%-----------------------------------------------------------------------------

\section{The Library Matrix}
\def\PrologFILE{System/matrix.pl}

The matrix, which is read and processed with filters, is parsed and stored in
the module {\sf matrix}. The internal organization of this module should not
be relevant for all applications using it. In fact the module {\sf matrix} has
gone through several incarnations where the data has been stored in the Prolog
data base (assert), the indexed data base (record), and the \eclipse{} global
variables. Anybody using the interface predicates only will not even recognize
if a new version is installed --- except for different execution times.


Inside a Prolog program running under the control of \ProTop, it is
sufficient to request the interface with
\begin{BoxedSample}
  :- lib(matrix).
\end{BoxedSample}
It might also turn out as useful to use the library {\sf literal} as well.

The library {\sf matrix} uses hooks wherever possible to allow the user to
perform actions whenever the matrix is changed. The concept and use of hooks
is described in the documentation of the library {\sf hooks}. The hooks are
meant as a means to store additional information in other modules. For other
purposes they may not turn out as appropriate.

The predicates provided by the library {\sf matrix} can be categorized as
follows. Primarily there are predicates to access the information about the
matrix. Additionally there are predicates to add further information to the
matrix or to remove it. Finally printing routines for the matrix are provided.

\subsection{The Data Structures}\label{matrix:data.structures}

The module {\sf matrix} is able to store a single matrix in clause normal
form. Since we are not interested in manipulating more than one matrix of
treat the matrix as a object of its own we have no explicit data structure for
a matrix. Instead we manipulate clauses and literals.


As an illustrative example in the remainder of this section we will use the
following matrix.
\begin{BoxedSample}
       [p(X), -q(f(X),g(X))].
base:: [q(a,X)].
loop:: [q(X,Y), -q(f(X),Y)].
?-     [-p(Z)].
\end{BoxedSample}
This matrix is presented in its external representation. Two clauses have labels
assigned to them and one clause is marked as a goal. For convenience it is
possible to write the negation sign as |-| and to omit the positive sign at all.
Internally we use a more strict regime. The central naming conventions for the
data structures used are discussed next.

\begin{description}
\item[Literal] \ \\
  A literal is a Prolog term of the form |++|{\em Predicate}\/ or |--|{\em
    Predicate}. From our example we can consider the literals in external
  representation |q(a,X)| and |-q(f(X),Y)|. The internal representation would
  be the following two literals
\begin{BoxedSample}
    ++q(a,X)
    --q(f(X),Y)
\end{BoxedSample}

\item[Index]\ \\
  Any literal is uniquely identified by an index. This index is assigned to
  the literals when the clauses or contrapositives are added to the matrix.
  Predicates are provided to manipulate the literal index. Use those
  predicates and do not assume anything about the index except that it
  uniquely points to a literal.

  At the time this is written the index is of the form {\em C|-|L}, where
  {\em C}\/ is a number pointing to the clause and {\em L}\/ is a literal
  pointing to the literal in this clause. When you are reading this, the
  internal representation of an index may have changed already. For reasons of
  efficiency, atoms instead of compound terms may be used as indices.
  
  {\large\bf Attention:} Do {\em not}\/ assume that the index is constructed in
  a special way, i.e. to be of the form {\em C|-|L}. Do {\em not}\/ assume
  that {\em C}\/ or {\em L}\/ are numbers!

\item[Clause Index]\ \\
  A clause index is a unique identifier for a clause. Several predicates are
  provided which take a clause index as argument. Especially one predicate is
  provided to get the clause index from an index.

  Currently the clause index is a number. This may change soon.

  {\large\bf Attention:} Do {\em not}\/ assume that the clause index is a
  number!

\item[Contrapositive]\ \\
  A clause can be accessed through the various literals contained in it. This
  leads to the concept of a contrapositive. A contrapositive is a set of
  literals which (normally) corresponds to the literals of a clause where one
  literal is distinguished. This literal is called ``entry point''. A literal
  can be described by the sign and predicate or by its unique index. Thus a
  contrapositive relates a literal and its index to the remaining set of
  literals.

  For some purposes it seemed to be desirable to manipulate various
  contrapositives of a clause in different ways. One application are Horn
  problems where at most one contrapositive of every clause is needed. For the
  goal no contrapositive is needed at all. Nevertheless the prover has to
  start the prove process and thus it has to get the goal clause.

\item[Clause]\ \\
  A clause is a set of literals. It is uniquely identified by a clause
  index. The clause may have contrapositives assigned to it but this is not
  neccesary. A clause can exist without having any contrapositives.

\item[Connection]\ \\
  Two literals or contrapositives can be related when they are potentially
  conplementary. This relation is also managed in the module {\sf matrix}. The
  relation itself has to be specified externally. The only property maintained
  in this module is that connections can only exist between literals present
  in it.

  Note that the connections managed by this module are directed connections.
\end{description}



\subsection{Accessing the Matrix}

\Predicate Contrapositive/3(?Literal, ?Body, ?Index).

This predicate can be used to get the complete information about a
contrapositive in the matrix.\footnote{To be precise we have to note that
  different entry points to a clause may keep different information.} This
predicate assumes that either the index |Index| is ground or the literal
|Literal| is sufficiently instantiated. In the first case at most one solution
is returned. In the second case all solutions are returned upon backtracking.
If neither the literal nor the index is given all contrapositives are
enumerated. This is the worst case and should be avoided.

|Literal| is unified with a literal (in internal representation; see
\ref{matrix:data.structures}). |Index| is a literal index. |Body| is a list
containing elements of the form {\em literal(L,I)}\/ where {\em L}\/ is a
literal and {\em I}\/ is an index. |Body| contains all literals which are in
the contrapositive of |Literal| except |Literal| itself. The complete clause
represented by this contrapositive would be
\begin{BoxedSample}
  [ literal(Literal,Index) \char"7C Body ]
\end{BoxedSample}


Let us assume that the matrix given earlier is stored in the module {\sf
  matrix}. Then we can access it as shown in the next example.
\begin{BoxedSample}
  ?- 'Contrapositive'(q(A,B),Body,Index).
  A = a
  B = X
  Body = []
  Index = 2-1
  ;
  A = X
  B = Y
  Body = [literal(--q(f(X),Y),3-2)]
  Index = 3-1
  ;
  No more solution
\end{BoxedSample}


\Predicate PositiveClause/1(?ClauseIndex).

This predicate unifies the clause index |ClauseIndex| with those clauses which
contain positive clauses only. All literals currently
in the clause are considered. Even so a single contrapositive may or may not be positive the
whole clause can have an independent property.
\begin{BoxedSample}
  ?- 'PositiveClause'(CI).
  CI = 2
\end{BoxedSample}

\Predicate NegativeClause/1(?ClauseIndex).

This predicate unifies the clause index |ClauseIndex| with those clauses which
contain negative clauses only. All literals are considered which are currently
in the clause. Even so a single contrapositive may or may not be negative the
whole clause can have an independent property.
\begin{BoxedSample}
  ?- 'NegativeClause'(CI).
  CI = 4
\end{BoxedSample}

\Predicate ClauseLength/2(?ClauseIndex,?Length).

This predicate enumerates all pairs of clause indices |ClauseIndex| and the
length |Length| of the clause belonging to |ClauseIndex|. This information is
extracted for the complete clause. Single contrapositives may have a different
length.

\begin{BoxedSample}
  ?- 'ClauseLength(1-1,Len).
  Len = 2
\end{BoxedSample}


\Predicate Clause/1(?ClauseIndex).

This predicate unifies |ClauseIndex| with each index of a clause currently
stored in the module matrix in turn.
\begin{BoxedSample}
  ?- findall( CI, 'Clause'(CI), ClauseList ).
  ClauseList = [1,2,3,4]
\end{BoxedSample}

\Predicate Clause/2(?ClauseIndex, ?Body).

This predicate takes a clause index |ClauseIndex| and unifies |Body| with a
list of the form {\em literal(L,I)}\/ where {\em L}\/ is a literal and {\em
  I}\/ is an index. Body contains all literals left in the specified clause.

The literals in the clause are not modified when a single contrapositive is
changed. Only the deletion of a literal in a clause, building the resolvent,
or deleting a complete clause may change the information returned by this
predicate.

Upon backtracking all unifiable pairs of clause index and literal list are
returned.
\begin{BoxedSample}
  ?- 'Clause'(3,Body).
  Body = [literal(++q(X,Y),3-1),literal(--q(f(X),Y),3-2)]
\end{BoxedSample}

\Predicate GoalClause/1(?ClauseIndex).

This predicate unifies |ClauseIndex| with the index of each clause which is
marked as goal. Several solutions can be returned upon backtracking.
\begin{BoxedSample}
  ?- findall( CI, 'GoalClause'(CI), ClauseList ).
  ClauseList = [4]
\end{BoxedSample}

\Predicate Label/2(?ClauseIndex, ?Label).

This predicate enumerates all pairs consisting of a clause index |ClauseIndex|
and an associated label |Label|. Note that the goal label is a special kind of
label which is not returned with this predicate.
\begin{BoxedSample}
  ?- 'Label'(CI,Label).
  CI = 2
  Label = base
  ;
  CI = 3
  Label = loop
  ;
  No more solution
\end{BoxedSample}

\Predicate ClauseLiteral/2(?ClauseIndex, ?Index).

This predicate establishes a relation between the clause and literal indices.
It can be used to find one or all literals in a clause as well as to find the
clause index of an index.

\begin{BoxedSample}
  ?- 'ClauseLiteral'(1,Index).
  Index = 1-1
  ;
  Index = 1-2
  ;
  No more solution

  ?- 'ClauseLiteral'(CI,3-2).
  CI = 3
\end{BoxedSample}


\Predicate Connection/2(?FromIndex, ?ToIndex).

This predicate relates the indices |FromIndex| and |ToIndex| according to the
connectedness. The connection graph contains an arc from a literal $L$\/ to
another literal $L'$\/ if an extension or reduction from $L$\/ to $L'$\/ can
be part of the final proof. This information is provided by the module {\sf
  connection\_graph}. 


\Predicate new_clause_index/1(-ClauseIndex).

This predicate unifies |ClauseIndex| with a unique new clause index. This
predicate is not resatisfiable.
\begin{BoxedSample}
  ?- new\_clause\_index(CI).
  CI = 5
\end{BoxedSample}


\Predicate new_literal_index/1(+-Index).

This predicate unifies |Index| with a unique new index in the clause
|ClauseIndex|. The index is not reserved in any way. If no associated literal
is added then the new index will be returned upon the next invocation of
|new_literal_index/1| again. This predicate is not resatisfiable.
\begin{BoxedSample}
  ?- new\_literal\_index(3,I).
  I = 3-3
\end{BoxedSample}


\subsection{Adding Information to the Matrix}

\Predicate add_clause/1(+Literals).

This predicate takes a list of literals |Literals| and stores all
contrapositives in the module {\sf matrix}. The hook |add_clause_hook/2| is
evaluated with the index of the new clause and the list of literals |Literals|
as arguments. This hook is evaluated after the literals have been added.

The list of literals |Literals| can be preceded by the goal label |?-|. In
this case the goal label is set also for this clause and the hook
|add_goal_clause_hook/1| is evaluated at the end. The argument is the index of
the goal clause.

The following instructions add all clauses to the module {\sf
  matrix}. Nevertheless the labels and the connections have not been considered yet.
\begin{BoxedSample}
  ?- add\_clause([++p(X),--q(f(X),g(X)]).
  ?- add\_clause([++q(a,X)]).
  ?- add\_clause([++q(X,Y),--q(f(X),Y)]).
  ?- add\_clause((?-[--p(Z)])).
\end{BoxedSample}

\Predicate add_contrapositive/3(+Literal, +Body, +Index).

This predicate adds the contrapositive with the given information to the
matrix. The entry point of the contrapositive is the literal |Literal| which
has the index |Index|. |Body| is a list of elements of the form {\em
  literal(L,I)}\/ where {\em L}\/ is a literal and {\em I}\/ is its index. The
literals of |Body| are the literals in the contrapositive except |Literal|.

\begin{BoxedSample}
  ?- add\_contrapositive(++p(X),[literal(--q(f(X),g(X)),1-2)],1-1).
\end{BoxedSample}

\Predicate add_connection/2(+FromIndex, +ToIndex).

This predicate adds a connection from the literal |FromIndex| to the literal
|ToIndex|. If such a connection already exists then this predicate simply
succeeds. The arguments have to be ground.
\begin{BoxedSample}
  ?- add\_connection(4-1,1-1).
\end{BoxedSample}


\Predicate add_goal_label/1(+ClauseIndex).

This predicate marks the clause with the index |ClauseIndex| as goal. {\em
  No}\/ implicit negation is performed! If the given clause has a goal
label already then the predicate simply succeeds.

The following example adds the goal label to each negative clause.
\begin{BoxedSample}
  ?- 'NegativeClause'(NC), add\_goal\_label(NC), fail; true.
\end{BoxedSample}

\Predicate add_label/2(+ClauseIndex, +Label).

This predicate adds the label |Label| to the clause with the index
|ClauseIndex|. If the label already exists the predicate simply succeeds.

The following goal may be called from within the parser to store the label of
the third clause.
\begin{BoxedSample}
  ?- add\_label(3,loop).
\end{BoxedSample}

\Predicate add_resolvent/3(+Index1, +Index2, +Dest).

This predicate performs a single resolution step. The literals $L_1$\/
identified by |Index1| and the literal $L_2$\/ identified by |Index2| must
have complementary signs. Those signs are stripped and the pure predicates are
unified. This unification uses an occurs check.\footnote{This unification may
  also take into account \eclipse{} metaterms.}

The further operations are determined by the value of |Dest|. The following
possibilities are provided:
\begin{description}
\item[clause({\em ClauseIndex}\/)]\ \\
  If {\em ClauseIndex}\/ is instantiated then the resolvent replaces the clause
  {\em ClauseIndex}. If {\em ClauseIndex}\/ is a variable then a new clause is
  created containing the resolvent. {\em ClauseIndex}\/ is unified with the
  resulting clause.
\item[contrapositive({\em Index}\/)]\ \\
  The resolvent is only performed in the contrapositive {\em Index}. This
  contrapositive is replaced by the result. {\em Index}\/ has to be
  sufficiently instantiated. Additionally {\em Index}\/ has to be in a common
  contrapositive with either |Index1| or |Index2|.
\end{description}




\subsection{Removing Information from the Matrix}


\Predicate delete_matrix/0().

This predicate removes all information stored in the module {\sf matrix}.
Afterwards the information is no longer accessible. Before the matrix is
cleared the hook |delete_matrix_hook/0| is evaluated. At this time the matrix
is still intact. Afterwards the clauses are deleted. This may lead to the
execution of the hook |delete_clause/1|. See the documentation of the
predicate |delete_clause_hook/1| for details.
\begin{BoxedSample}
  ?- delete\_matrix.
\end{BoxedSample}

\Predicate delete_clause/1(?ClauseIndex).

This predicate deletes all clauses --- upon backtracking --- for which the
clause index unifies with |ClauseIndex|.

All information having any relation to the clause being removed is erased.

The hook |delete_clause_hook/1| is evaluated for each clause removed. The
argument is the instantiated clause index of the clause being deleted. The
hook is called before any other information about this clause is removed from
matrix. 

The following example deletes the clause with the index 3.
\begin{BoxedSample}
  ?- delete\_clause(3).
\end{BoxedSample}

\Predicate delete_contrapositive/1(?Index).

This predicate deletes a single contrapositive identified by |Index|. |Index|
may be un-instantiated in which case all contrapositives are removed upon
backtracking. The connections starting from the literal |Index| are also
removed.

The clause as a whole is not affected. Nor are the |ClauseLength| or other
informations related to the whole clause. As a consequence the clause can
continue to exist even though there are no contrapositives left for it.

The hook |delete_contrapositive_hook/1| is evaluated for each contrapositive
being removed. The hook is called before the information is erased from the
matrix. The argument of the hook is the instantiated index of the
contrapositive being deleted.

\begin{BoxedSample}
  ?- delete\_contrapositive(3-2).
\end{BoxedSample}

\Predicate delete_connection/2(?FromIndex, ?ToIndex).

This predicate deletes upon backtracking all connections for which the
starting point unifies with |FromIndex| and the end point unifies with
|ToIndex|. 

The hook |delete_connection_hook/2| is evaluated with the instantiated
|FromIndex| and |ToIndex| as argument for every connection being removed.

The following example deletes all connections stored in the matrix.
\begin{BoxedSample}
  ?- delete\_connection(\_,\_),fail ; true.
\end{BoxedSample}

\Predicate delete_literal/1(+Index).

This predicate deletes the literal identified by |Index| from the matrix. Any
information is abolished. All connections to and from this literal are
deleted. The literal does not occur any longer in any contrapositive or the
whole clause.

The literal has to be sufficiently instantiated. The predicate fails if the
specified literal does not exist.

The hook |delete_literal_hook/1| is evaluated before the literal is removed.
The argument of this hook is the index of the literal to be deleted.

The hooks |delete_contrapositive_hook/1| and |delete_connection_hook/2| may be
called when the respective information is deleted.

The following example deletes the literal |3-2| from the matrix.
\begin{BoxedSample}
  ?- delete\_literal(3-2).
\end{BoxedSample}

\Predicate delete_goal_label/1(?ClauseIndex).

This predicate deletes the goal label of the given clause. |ClauseIndex| is in
turn unified with all clause indices having a goal label. The   predicate is
resatisfiable until no more unifying goal clause is present.
\begin{BoxedSample}
  ?- delete\_goal\_label(CI).
  CI = 4
  ;
  No more solution
\end{BoxedSample}

\Predicate delete_label/2(?ClauseIndex, ?Label).

This predicate deletes the label |Label| of the given clause |ClauseIndex|.
The predicate is resatisfiable until no more labels are present.
\begin{BoxedSample}
  ?- delete\_label(CI,Label).
  CI = 2
  Label = base
  ;
  CI = 3
  Label = loop
  ;
  No more solution
\end{BoxedSample}


\subsection{Printing the Matrix}

\Predicate show_matrix/2(+Prefix, +Stream).

The current matrix is printed onto the stream |Stream|.  Each line written is
preceded by the string |Prefix|.  The format used is suitable for reading it
back in if |Prefix| is the empty string.

|Stream| can also be a symbolic stream of \eclipse. E.g. |output| directs the
matrix to the standard output stream.
\begin{BoxedSample}
  ?- show\_matrix("\%", output).
  \%      [ \% Clause 1
  \%        ++ p(X),
  \%        -- q(f(X),g(X))
  \%      ].
  \% base:: [  \% Clause 2
  \%        ++ q(a,X)
  \%      ].
  \% loop:: [  \% Clause 3
  \%        ++ q(X,Y),
  \%        -- q(f(X),Y)
  \%      ].
  \% ?-   [  \% Clause 4
  \%         -- p(Z)
  \%      ].
\end{BoxedSample}

\Predicate show_contrapositives/2(+Prefix, +Stream).

The current list of contrapositives is printed onto the stream |Stream|.  Each
line written is preceded by the string |Prefix|.

|Stream| can also be a symbolic stream of \eclipse. E.g. |output| directs the
matrix to the standard output stream.
\begin{BoxedSample}
  ?- show\_contrapositives(" ", output).
    ++ p(X) <- \% (1 - 1)
           -- q(f(X),g(X)).
    -- q(f(X),g(X)) <- \% (1 - 2)
           ++ p(X).
    ++ q(a,X). \% (2 - 1)
    ++ q(X,Y) <- \% (3 - 1)
           -- q(f(X),Y).
    -- q(f(X),Y) <- \% (3 - 2)
           ++ q(X,Y).
    -- p(Z). \% (4 - 1)
\end{BoxedSample}

%
% Local Variables: 
% mode: latex
% TeX-master: t
% End: 
