%%%****************************************************************************
%%% $Id: protop.tex,v 1.10 1995/07/03 11:35:12 gerd Exp gerd $
%%%============================================================================
%%% 
%%% This file is part of ProCom.
%%% It is distributed under the GNU General Public License.
%%% See the file COPYING for details.
%%% 
%%% (c) Copyright 1995 Gerd Neugebauer
%%% 
%%% Net: gerd@imn.th-leipzig.de
%%% 
%%%****************************************************************************

\chapter{Using \ProTop}

%------------------------------------------------------------------------------
\section{Introduction}

\ProTop{} is a shell which allows to perform all tasks necessary to run a
theorem prover. To accomplish this a set of instructions is provided to
control the actions of the prover top-level shell \ProTop. This chapter
describes the various possibilities provided by \ProTop.


\section{Interactive Use}

Two major uses of \ProTop{} can be envisioned: The interactive use and the use
as script language. First we describe  the interactive use. To start \ProTop{}
you just have to execute  the command |protop| from the  shell. For this might
be necessary  to make the location of  the executable  explicit or enlarge the
search path appropriately.

After \ProTop{} is started a welcome message is printed and a prompt appears.
The prompt is usually the string |ProTop ->|. Now you can type in commands.
Those commands are read by an interpreter and executed immediately.

Note that the reading apparatus of Prolog is used to parse the input. Thus it
is necessary to terminate any input with a full stop.

Consider the following (rather short) sample session:

\begin{BoxedSample}
\$ |protop|
\

\                --- Welcome to ProTop (\Version) ---

ProTop -> |stop.|

bye
\$ 
\end{BoxedSample}

The output of the computer is presented in a {\itt slanted teletype font}. The
input of a user is presented in an {\tt upright teletype font}.

As we see \ProTop{}   is started from the  shell.  At the prompt the   command
|stop| is  typed in. This terminates  the  session immediately and  control is
returned to the shell. Further commands and options to control the behavior of
\ProTop{} are described on the next few pages.

The most important  instructions are those which  leave \ProTop.   Most of the
possibilities one may think of are included. Thus it should  be fairly easy to
leave \ProTop.

\begin{description}
  \item [end]\index{ProTop!end}\ \\
  This instruction ends the session if executed at the top level.
  \item [stop]\index{ProTop!stop}\ \\
  This instruction ends the session if executed at the top level.
  \item [exit]\index{ProTop!exit}\ \\
  This instruction ends the session if executed at the top level.
  \item [halt]\index{ProTop!halt}\ \\
  This instruction ends the session when executed in an arbitrary context.
\end{description}

\section{Options Controlling \ProTop}

\begin{description}
  \item [ProTop:welcome] |= on|
    \\
    This option controls whether the startup message should be printed.
  \item [ProTop:debug] | = off|
    \\
    This option allows us to get more detailed information on the actions
    performed by \ProTop. This is especially useful when debugging script
    files.
  \item [ProTop:verbose] | = on|
    \\
    This option can be used to control the verbosity of the response given by
    \ProCom.
  \item [ProTop:resource\_file] | = ".protop"|
    \\
    This option specifies a file to be loaded when \ProTop{} is started.
    Usually it is only useful to redefine this option if \ProTop{} is started
    from within Prolog (see section \ref{ProTop:prolog.interface})
  \item [ProTop:path] | = [".","Scripts","~"]|
    \\
    This option specifies the search path where \ProTop\ scripts are searched
    for. It is strongly recommended to include the current directory in this
    list.
  \item [ProTop:script\_extension] | = ["",".pt"]|
    \\
    This option is used to specify the extensions which should be tried when
    searching for \ProTop{} script files.
  \item [ProTop:prover\_file] | = "...prover.pl"|
    \\
    This option specifies the intermediate file name of the prover.
  \item [ProTop:backup] | = on|\label{ProTop:backup}
    \\
    This option controls the action performed when discovering an old report
    file. If this option is turned on then the old report file is moved aside
    by appending some (negative) version number to it. Otherwise the old
    report file is overwritten.
\end{description}

\section{Getting Information About \ProTop}

\begin{description}
  \item [help]\index{ProTop!help}\ 
    \\
    This instruction prints a short information which should help you to carry
    on. The help command can also be given an argument. In this case some
    information on the argument is given. This is the same as executing the
    |status| command with an argument. For details see the description of
    |status|.
  %
\begin{BoxedSample}
  ProTop -> |help.|
  The command `show' can be used to get information on the
  system as a whole or on parts of it.
    show.
  presents the status of some important parts of ProTop.

    show commands.
  gives a list of all commands available.

  To leave ProTop type
    halt.
  ok.
  ProTop ->
\end{BoxedSample}

  \item [status]\index{ProTop!status}\ 
    \\
    This instruction prints the status of various parts of \ProTop{} to the
    scrren. The status of parts can be displayed seperately by the next
    instruction.
  %
\begin{BoxedSample}\def\Prover#1{          #1\par}\let\Filter\Prover
  ProTop -> |status.|
    This is ProTop version \Version
    Filters:
\ProTopFilters
          none
    Provers:
\ProTopProvers
    Matrix:
          No matrix loaded.
    Macros:
          0 macros are defined.
  ok.
\end{BoxedSample}

  \item [status({\em Part})]\index{ProTop!status}\ 
    \\
    This instruction displays the status of the requested part {\em Part} to
    the screen. {\em Part}\/ can also be a list of parts in which case the
    status of all parts given in the list is displayed. The following parts
    can be specified:
    \begin{description}
    \item [version]\ \\ prints the version of \ProTop.
    \item [macros]\ \\  prints a summary of macros defined in \ProTop.
    \item [prover]\ \\  prints a list of provers loaded.
    \item [filter]\ \\  prints a list of filters loaded.
    \item [matrix]\ \\  prints a summary of the matrix currently loaded.
    \item [clauses]\ \\ prints a list of clauses currently stored in the
      matrix.
    \item [contrapositives]\ \\ prints a list of contrapositives currently
      stored in the matrix.
    \item [modules]\ \\ prints a list of the loaded modules and the associated
      documentation strings.
    \item [module({\em Module}\/)]\ \\ prints documentation string for module
      {\em Module}. It fails if {\em Module}\/ is not instanciated properly or
      it contains no \ProTop{} module.
    \item [options]\ \\ prints a list of all options.
    \item [option({\em Option}\/)]\ \\ prints the value of the option {\em
        Option}. 
    \item [file({\em File}\/)]\ \\ displays the contents of the file {\em
        File} on the screen.
    \item [macros]\ \\ prints the list of all macros to the screen.
    \item [macro({\em Macro}\/)]\ \\ displays the definition of all macros for
      which the head unify with {\em Macro}.
    \item [all]\ \\     does {\bf version}, {\bf filter}, {\bf prover}, {\bf
        matrix} and {\bf macros}.
    \end{description}

    This instruction fails if one of the requested parts is not known or it is
    called with a variable argument.

  \item [info]\index{ProTop!info}\ 
    \\
    This is a variant of |status|. For details see above.

  \item [list]\index{ProTop!list}\ 
    \\
    This is a variant of |status|. For details see above.

  \item [show]\index{ProTop!show}\ 
    \\
    This is a variant of |status|. For details see above.

\end{description}


\section{Script Files}

Script files provide a means to keep a sequence of instructions beyond the end
of a single session.

One place where a script file is involved is at the startup of \ProTop. At
this occasion the file {\sf .protop} is sought in the current directory and in
the home directory of the user and instructions from this file are executed if
it is found. The primary intension is to allow the user to specify some
options to adapt the behavior of \ProTop. Nevertheless arbitrary instructions
can be placed there to be executed at startup time.

The following commands are related to script files

\begin{description}
  \item [include({\em File})]\index{ProTop!include}\ 
    \\
    This instruction reads commands from the file {\em File} and executes
    them.  The option |ProTop:path| can be used to specify the search path.
    The option |ProTop:script\_ext| can be used to specify the extensions to
    be appended when searching for the actual file.
  %
\begin{BoxedSample}
  ProTop -> |include("some_script").|
  |---| ProTop script file some\_script not found.
  no.
  ProTop -> |include(".protop").|
  ok.
  ProTop -> 
\end{BoxedSample}

  \item ["{\em File}"]\index{ProTop!include}\ 
    \\
    A file name marked as a string --- i.e. enclosed in double quotes --- Is
    taken as a file name to be included. Thus it is equivalent to
    {\bf include({\em File}\/)}.

  \item [end\_of\_file]\index{ProTop!end\_of\_file}\ \\
  This token is returned by the Prolog reading apparatus upon end of file.
  Thus it can be used in this sense. Upon end of file the session is usually
  ended. If encountered in a script file the rest of the file is ignored.
\end{description}

\section{Output and Input}

For convenience some commands of the underlying Prolog are made available as
\ProTop{} instructions. They can be used to write interactive script files.

\begin{description}
  \item [nl]\index{ProTop!nl}\ \\
  This instruction writes a newline to the screen.
  \item [write({\em Term})]\index{ProTop!write}\ \\
  This instruction writes {\em Term}\/ to the screen. {\em Term} is {\sc not}
  followed by a newline.
  %
\begin{BoxedSample}
  ProTop -> |write("hello world.").|
  hello world.ok.
  ProTop -> 
\end{BoxedSample}

  \item [writeln({\em Term})]\index{ProTop!writeln}\ \\
  This instruction writes {\em Term}\/ to the screen. {\em Term} is followed
  by a newline.
  %
\begin{BoxedSample}
  ProTop -> |writeln("hello world.").|
  hello world.
  ok. 
\end{BoxedSample}

  \item [printf({\em Format}, {\em Arguments})]\index{ProTop!printf}\ \\
  This instructions prints the {\em Arguments} according to the format string
  {\em Format}. {\em Format} is a string where \% is used as escape to specify
  the form of an argument. {\em Arguments} is a list of terms to be printed
  according to the format string. For details see the documentation of the
  \eclipse{} predicate |printf/2|.
  %
\begin{BoxedSample}
  ProTop -> |printf(".....%w.....%w.....\n",[1,2]).|
  .....1.....2.....
  ok.
\end{BoxedSample}

  \item [read\_string({\em String})]\index{ProTop!read\_string}\ \\
  This reads all characters up to the next newline into the string {\em
  String}. The end of file also terminates the reading.
  %
\begin{BoxedSample}
  ProTop -> |read_string(String), writeln(String).|
   |some string terminated by newline|
  some string terminated by newline
  ok.
\end{BoxedSample}

  \item [read\_string({\em Prompt}, {\em String})]\index{ProTop!read\_string}\
  \\
  This instruction writes {\em Prompt} to the screen and reads all characters
  up to the next newline into the string {\em String}. The end of file also
  terminates the reading.
  %
\begin{BoxedSample}
  ProTop -> |read_string("Enter a string:",String).|
  Enter a string: |some string terminated by newline|
  ok.
\end{BoxedSample}

  \item [read({\em Term})]\index{ProTop!read}\ \\
  This instruction reads a prolog term from the standard input and unifies it
  with {\em Term}. If the unification fails then this command fails. The term
  has to be terminated by a colon.
  %
\begin{BoxedSample}
  ProTop -> |read(Term), writeln(Term).|
   |some_term(Var,const).|
  some\_term(Var, const)
  ok.
  ProTop -> |read(term), writeln(Term).|
   |some_term(Var,const).|
  no.
\end{BoxedSample}

  \item [read({\em Prompt}, {\em Term})]\index{ProTop!read}\ \\
  This instruction writes {\em Prompt} to the screen and reads a prolog term
  from the standard input and unifies it with {\em Term}. If the unification
  fails then this command fails. The term has to be terminated by a colon.
  %
\begin{BoxedSample}
  ProTop -> |read("Term: ",Term), writeln(Term).|
  Term:  |some_term(Var,const).|
  some\_term(Var, const)
  ok.
\end{BoxedSample}

\end{description}

\section{Flow of Control}

\begin{description}
  \item [{[]}]\index{ProTop![]}\ \\
  An empty list of instructions simply succeeds.
  %
\begin{BoxedSample}
  ProTop -> |[].|
  ok.
\end{BoxedSample}

  \item [{[{\em Instruction}, \ldots, {\em Instruction}]}]\index{ProTop![...]}\ \\
  A list of instructions is executed like a conjunction (see below).
  %
\begin{BoxedSample}
  ProTop -> |[write(1),write(2),nl].|
  12
  ok.
\end{BoxedSample}

  \item [({\em Instruction}, \ldots, {\em Instruction})]\index{ProTop!(,)}\ \\
  Conjunctions are executed from left to right until one instruction fails.
  In this case the whole conjunction fails. If no instruction fails then
  the conjunction succeeds.
  %
\begin{BoxedSample}
  ProTop -> |write(1),write(2),nl.|
  12
  ok.
\end{BoxedSample}

  \item [({\em Instruction}; \ldots; {\em Instruction})]\index{ProTop!(;)}\ 
    \\
    Disjunctions are executed from left to right until one instruction
    succeeds.  In this case the whole disjunction succeeds. If no instruction
    succeeds then the disjunction fails.
  %
\begin{BoxedSample}
  ProTop -> |(writeln(first),fail); writeln(second).|
  first
  second
  ok.
\end{BoxedSample}

  \item [repeat {\em N}\/ times {\em Command}]\index{ProTop!repeat}\ 
    \\
    This instruction repeats the execution of {\em Command}\/ several times.
    {\em N}\/ is an integer denoting the number of repetitions.
  %
\begin{BoxedSample}
  ProTop -> |repeat 5 times write("----+").|
  ----+----+----+----+----+ok.
  ProTop ->       
\end{BoxedSample}

  \item [for {\em Var}\/ in {\em Spec}\/ do {\em Command}]\index{ProTop!for}\ 
    \\
    This instruction repeats {\em Command}\/ for all elements of {\em Spec}\/
    assigned to {\em Var}. This is done in a failure driven manor, i.e.
    variable bindings are undone after each loop. {\em Spec}\/ is expanded to
    a list before the loop is performed. The following constructs for {\em
      Spec}\/ are supported:
  \begin{description}
    \item [{[{\em ...}]}]\ \\
    A list of any elements is the simplest form of a specification. No
    expansion takes place in this case.
    %
    \begin{BoxedSample}
ProTop -> |for Var in [1,2,3] do writeln(Var).|
1
2
3
ok.
    \end{BoxedSample}

    \item [files({\em Dir})]\ 
      \\
      This specification expands to a list of all files in the directory {\em
        Dir}. Note that the files are not sorted.
    %
\begin{BoxedSample}
  ProTop -> |for F in files(".") do writeln(F).|
  ./COPYING
  ./README
  ./INSTALL
  ./Makefile
  ...
  ok.
\end{BoxedSample}

    \item [files({\em Dir}, {\em Pattern})]\ 
      \\
      This specification expands to a list of all files in the directory {\em
        Dir} which match the pattern {\em Pattern}. The syntax of pattern is
      the syntax allowed for Bourne shell commands.
    %
\begin{BoxedSample}
  ProTop -> |for F in files(".","*.pl") do writeln(F).|
  ./main.pl
  ./...prover.pl
  ok.
\end{BoxedSample}

    \item [directories(Dir)]\ 
      \\
      This specification expands to a list of all directories in the directory
      {\em Dir}.
    %
\begin{BoxedSample}
  ProTop -> |for D in directories(".") do writeln(D).|
  ./Doc
  ./System
  ./ProCom
  ./Reductions
  ./Capri
  ./inputs
  ...
  ok.
\end{BoxedSample}

  \end{description}

  \item [if {\em Condition}\/ then {\em Then}\/ else {\em
      Else}]\index{ProTop!if}\ 
    \\
    This instruction evaluates the commands {\em Condition}. If this
    evaluation succeeds then the commands {\em Then}\/ are evaluated
    afterwards. Otherwise the commands {\em Else}\/ are executed.
    %
\begin{BoxedSample}
  ProTop -> |if exists("/etc/motd")|
            |then show(file("/etc/motd"))|
            |else writeln("Sorry").|
  SunOS Release 4.1.3 (GENERIC) \#3: Mon Jul 27 16:44:16 PDT 1992
  ok. 
\end{BoxedSample}

  \item [if {\em Condition}\/ then {\em Then}]\index{ProTop!if}\ 
    \\
    This instruction evaluates the commands {\em Condition}. If this
    evaluation succeeds then the commands {\em Then}\/ are evaluated
    afterwards. Otherwise it succeeds without performing any other actions.
    %
\begin{BoxedSample}
  ProTop -> |if not(exists("/etc/motd")) then|
              |writeln("Sorry").|
  ok.
\end{BoxedSample}

  \item [true]\index{ProTop!true}\ \\
  This instruction always succeeds and does nothing else.
  %
\begin{BoxedSample}
  ProTop -> |true.|
  ok.
\end{BoxedSample}

  \item [fail]\index{ProTop!fail}\ \\
  This instruction always fails and does nothing else.
  %
\begin{BoxedSample}
  ProTop -> |fail.|
  no.
\end{BoxedSample}

  \item [exists({\em File})]\index{ProTop!exists}\ \\
  This instruction checks the existence of the file {\em File}. {\em File}
  must be a symbol or string. Otherwise a type error is raised and the
  instruction fails.
  %
\begin{BoxedSample}
  ProTop -> |exists("~/.login").|
  ok.
  ProTop -> |exists("~/.protop").|
  no.
\end{BoxedSample}

  \item [not({\em Condition}\/)]\index{ProTop!not}\ 
    \\
    This instruction executes {\em Condition}\/ and reverses the return
    status.  I.e. it succeeds if {\em Condition}\/ fails and vice versa.
  %
\begin{BoxedSample}
  ProTop -> |not(true).|
  no.
  ProTop -> |not(fail).|
  ok.
\end{BoxedSample}

  \item [find\_file({\em File},{\em Path},{\em Extensions},{\em
      FullFile})]\index{ProTop!find\_file}\ 
    \\
    This instruction tries to find the existing file {\em File} in a list of
    directories {\em Path}. A list of extensions {\em Extensions}\/ is used to
    augment the file name with an additional extension. The result is unified
    with {\em FullFile}.
  %
\begin{BoxedSample}
  ProTop -> |find_file(".PROTOP",[".","~"],[""],File),|
            |writeln(File).|
  no.
  ProTop -> |find_file(".protop",[".","~"],[""],File),|
            |writeln(File).|
  ./.protop
  ok.
\end{BoxedSample}
  
  \end{description}

\section{Macros}

Macros are a means to ease live. Tasks which have to be performed repeatedly
can be defined as a macro and executed when necessary. The simplest case is
a macro as abbreviation for some other instructions as in the following
example:
\begin{BoxedSample}
  ProTop -> |my_macro := write("hello "),|
  |            write("world"), write("."), nl.|
  ok.
  ProTop -> |my_macro.|
  hello world.
  ok.
\end{BoxedSample}
Ok, it's rather simple. This macro simply writes |hello world.| followed by a
newline --- an effect one can achieve simpler. The execution is triggered by a
call with its name.

Macros may also have arguments. Prolog variables are used to denote variable
parts of the arguments. Let us again consider out previous example and make it
a little bit more complicated:
\begin{BoxedSample}
  ProTop -> |my_macro(Arg) :=|
  |           write("hello "), write(Arg), write("."), nl.|
  ok.
  ProTop ->
\end{BoxedSample}

Note that this macro can coexist with the previous one since the number of
actual parameters distinguishes them. If we call this macro with the command
|my_macro("world").| we get the same as before. But now we can also specify
another argument, as in
\begin{BoxedSample}
  ProTop -> |my_macro("world").|
  hello world.
  ok.
  ProTop -> |my_macro("friends").|
  hello friends.
  ok.
\end{BoxedSample}

The coexistence of macros goes further. The selection of an appropriate macro
is done via unification of its head. We exploit this fact by further
developing our example:
\begin{BoxedSample}
  ProTop -> |my_macro(all,999) := my_macro("world").|
  ok.
  ProTop -> |my_macro(some,12) := my_macro("friends").|
  ok.
\end{BoxedSample}

These two instructions establish {\em two} definitions of |my_macro/2|. Now we
can test them and see how they work:

\begin{BoxedSample}
  ProTop -> |my_macro(all,X).|
  hello world.
  ok.
  ProTop -> |my_macro(some,X).|
  hello friends.
  ok.
  ProTop -> |my_macro(A,B), writeln('A'=A), writeln('B'=B).|
  hello world.
  A = all
  B = 999
  ok.
\end{BoxedSample}

To conclude we summarize all macro related commands we have already seen and
some others, not mentioned yet.

\begin{description}
  \item [{\em Macro} := {\em Code}]\index{ProTop!:=}\ \\
  This instruction stores {\em Code} as replacement text for {\em Macro}.
  Several clauses for the same macro can be specified.

  \item [delete macro({\em Macro})]\index{ProTop!delete macro}\ \\
  This instruction deletes all macros that unify with {\em Macro}. As a
  consequence we can delete all macros with the command
  %
\begin{BoxedSample}
  ProTop -> |delete macro(_).|
  ok.
\end{BoxedSample}

  Note that the underscore |_| is the anonymous variable like in Prolog.
  A more natural use would be the following one:
  %
\begin{BoxedSample}
  ProTop -> |delete macro(my_private_macro(_,_)).|
  ok.
\end{BoxedSample}

  This example deletes all entries for the macro |my_private_macro| with two
  arguments.

  \item [show macro({\em Macro})]\index{ProTop!show macro}\ \\
  This instruction lists all macros which unify with {\em Macro}. If none is
  found then an informative message is displayed.
  %
\begin{BoxedSample}
  ProTop -> |show macro(my_macro).|
  my\_macro := write(hello ), write(world), write(.), nl
  ok.
  ProTop -> 
\end{BoxedSample}

  \item [defined({\em Macro})]\index{ProTop!defined}\ \\
  This instruction checks for the existence of macros which unify with {\em
  Macro}. The return status indicates success or failure.
  %
\begin{BoxedSample}
  ProTop -> |defined(mac).|
  no.
  ProTop -> |mac := true.|
  ok.
  ProTop -> |defined(mac).|
  ok.
\end{BoxedSample}

  \item [{\em Macro}]\ \\
  A macro is executed when its name is encountered as command. Since the
  built-in commands have precedence over macros it is not possible to redefine
  a built-in.

  \item [op({\em Precedence}, {\em Accosiativity}. {\em Operator})]\ 
    \index{ProTop!op}\\
    This instruction can be used to define operators like in Prolog. This
    feature can be used to make the input to \ProTop{} better readable for
    humans. 
  %
\begin{BoxedSample}
  ProTop -> |op(900,fx,say).|
  ok.
  ProTop -> |say(X) := writeln(X).|
  ok.
  ProTop -> |say hello.|
  hello
  ok.
\end{BoxedSample}
\end{description}

\section{Options Management}

\begin{description}
  \item [{\em Option} = {\em Value}]\index{ProTop!=}\ \\
  This instruction sets or gets the value of the option {\em Option}. If {\em
  Value} is not a variable then this value is stored in {\em Option}.
  Otherwise {\em Value}\/ is unified with the actual value of {\em Option}.
  {\em Option} is required to be a symbol.

\begin{BoxedSample}
  ProTop -> |verbose = off.|
  ok.
  ProTop -> |verbose = VERB, writeln(VERB).|
  off
  ok.
\end{BoxedSample}

  \item [reset\_options]\index{ProTop!reset\_options}\ \\
  This instruction resets all options to their default values. It always
  succeeds and produces no output.

  \item [list options]\index{ProTop!list options}\ \\
  This instruction prints a list of all options together with their values to
  the screen.

  \item [list option({\em Option})]\index{ProTop!list option}\ \\
  This instruction prints a single option {\em Option}\/ and its value to the
  screen.
  %
\begin{BoxedSample}
  ProTop -> |list option prover.|
         prover = procom(extension\_procedure)
  ok.
\end{BoxedSample}

\end{description}

\section{Proving}

\begin{description}

  \item [prove({\em File})]\index{ProTop!prove}\ \\
  This instruction performs compilation and running of the compiled prover.
\end{description}


\section{Report Generation}


\begin{description}
  \item [report:file] |= "report.tex"|\\
  This option names the report file to be generated.

  \item [report:ignore\_options]|= ['run:remove_prover',match('ProTop'),...]|\\
  This option contains a list of specifications for options which should not be
  carried over to the report file. Several types of elements of such a list
  are recognized:
  \begin{description}
    \item [{\em 'symbol'}] A symbol selects the option named.
    \item [match({\em 'symbol'})] A match selects all options contain {\em
    symbol} as substring.
  \end{description}

  \item [report:style\_options] |= ["p-r"]|\\
  This option contains a list of style options (their names as strings) which
  are used when generating a proof report. |"p-r"| is needed to print the
  constructs in the generates \LaTeX{} file. You may consider providing an
  alternative style options file to change the appearance. Otherwise you can
  add options like |"11pt"| to modify the size of the fonts used etc.

  \item [report:style\_path] |= "..."|\\
  This option names the place where additional style files for \LaTeX{} can be
  found. It is automatically set during the installation process to point into
  the installation directory (in fact the {\sf input} subdirectory).
  The value of this option is a string.

  Several paths can be given by separating them with a colon (:).

  \item [report:latex] |= latex|\\
  This option names a command executable in the Bourne shell (sh) to start
  \LaTeX{}. It may be necessary to specify the full path name if the command
  is not on the shell search path (environment variable |$PATH|).%$

  \item [report:flags] |= [preamble,titlepage,text,comment,time,options,matrix,tree]|\\
  This option controlls which parts of the proof report are to be generated.
  It contains a list of the following key words:

  \def\X[#1]{{\bf #1}&}
  \begin{tabular}{lp{.7\textwidth}}
    \X[preamble]                \\
    \X[titlepage]               \\
    \X[text]            \\
    \X[comment]         \\
    \X[time]            \\
    \X[options]         \\
    \X[matrix]          \\
    \X[tree]
  \end{tabular}

  \item [report:sections] |= [comment, text, matrix, options, times, proof]|\\
  This option controlls which subsections to include for each run of the
  prover. The following key words are recognized:

  \begin{tabular}{lp{.7\textwidth}}
    \X[comment] Include the comments as \TeX\ comments. \\
    \X[text]    Include the descriptive text. \\
    \X[matrix]  Include the internal representation of the matrix. \\
    \X[options] Include a list of options into the report. \\
    \X[times]   Include a table of times into the report. \\
    \X[proof]   Include the proof (tree/description) into the report.
  \end{tabular}

  \item [report:tree\_flags] |= [dx(100),dy(100),tree,info]|\\
  This option contains a list of keywords controlling the appearance of the
  proof. The following key words are recognized:

  \begin{tabular}{lp{.7\textwidth}}
    \X[tree]            \\
    \X[info]            \\
    \X[dx($\Delta x$)]          \\
    \X[dy($\Delta y$)]
  \end{tabular}

\end{description}


\begin{description}
  \item [report init({\em ReportFile})]\index{ProTop!report init}\ \\
  This instruction initializes the report file {\em ReportFile}. Subsequent
  actions may leave a message in this file. Depending on the value of the
  option |ProTop:backup| an existing file is saved or overwritten (see page
  \pageref{ProTop:backup}).

  \item [report init]\index{ProTop!report init}\ \\
  This instruction initializes the report file from the option |log_file|. See
  also above.

  \item [report author({\em Text})]\index{ProTop!report author}\ \\
  This instruction writes the text {\em Text} to the report file. This text is
  intended to be used as title of an automatically generated report.

  \item [report title({\em Text})]\index{ProTop!report title}\ \\
  This instruction writes the text {\em Text} to the report file. This text is
  intended to be used as author of an automatically generated report.

  \item [report comment({\em Text})]\index{ProTop!report comment}\ \\
  This instruction writes the text {\em Text} as comment to the report file.
  This report file has to be initialized with |report_init| before this
  instruction is used.

  \item [report text({\em Text})]\index{ProTop!report text}\ \\
  This instruction writes the text {\em Text} as descriptive text to the
  report file. The report file has to be initialized with |report_init|
  before this instruction is used. 

  \item [report section({\em Text})]\index{ProTop!report section}\ \\
  This instruction writes the text {\em Text} as short title of the following
  proof attempt to the report file. It may also be used when tables are
  typeset.  The report file has to be initialized with |report_init|
  before this instruction is used. 

  \item [report label({\em Label})]\index{ProTop!report label} This
  instruction writes the string {\em Label} as identifier for the following
  proof attempt to the report file. It is also used when tables are typeset.
  The report file has to be initialized with |report_init| before this
  instruction is used.

  \item [define\_table({\em Name}, {\em Spec})]\index{ProTop!define\_table}

  \item [remove\_table({\em Name})]\index{ProTop!remove\_table}

  \item [generate\_report]\index{ProTop!generate\_report}\ \\
  This instruction reads the log file and generated the \LaTeX{} source from
  it. This report file can be processed using |latex_report| of processed by
  \LaTeX{} manually.

  {\bf Note:} A \LaTeX{} style file is required to process the generated
  \LaTeX{} code. This style file can be found in the {\sf input} subdirectory
  of the \ProTop{} installation directory.

  \item [generate\_report({\em Files})]\index{ProTop!generate\_report}\ \\
  This instruction reads the file {\em Files} and generated the \LaTeX{}
  source from it.

  \item [latex\_report]\index{ProTop!latex\_report}\ \\
  This instruction tries to run \LaTeX{} on a formerly generated report.

  \item [make\_report]\index{ProTop!make\_report}\ \\
  This instruction generates a report and runs \LaTeX{} on the resulting
  source file. The file given in the option |log_file| is used for analysis.

  \item [make\_report({\em Files})]\index{ProTop!make\_report}\ \\
  This instruction generates a report from the files {\em Files}\ and runs
  \LaTeX{} on the resulting source file.

\end{description}

\section{Misc Instructions}

\begin{description}
  \item [pwd]\index{ProTop!pwd}\ 
    \\
    This instruction prints the current working directory to the standard
    output stream.
  %
\begin{BoxedSample}
  ProTop -> |pwd.|
  /system/ProCom/
  ok.
\end{BoxedSample}

  \item [pwd(Dir)]\index{ProTop!pwd}\ 
    \\
    This instruction unifies it's argument with the current directory. Thus it
    is possible to get your hands on the current directory in script files.
  %
\begin{BoxedSample}
  ProTop -> |pwd(X).|
  ok.
  ProTop -> |pwd(X),writeln(X).|
  /system/ProCom/
  ok.
\end{BoxedSample}

  \item [cd(Dir)]\index{ProTop!cd}\ \\
  This instruction tries to set the current directory to the one given as
  argument. |Dir| has to be a string or atom which corresponds to a valid
  directory name.
  %
\begin{BoxedSample}
  ProTop -> |pwd.|
  /system/ProCom/
  ok.
\end{BoxedSample}

  \item [ls]\index{ProTop!ls}\ \\
  This instruction lists all files in the current directory on the standard
  output stream.


  \item [ls(Dir)]\index{ProTop!ls}\ \\
  This instruction lists all files in the directory |Dir| on the standard
  output stream.

  \item [call({\em Goal})]\index{ProTop!eval}\ \\
  This instruction forwards {\em Goal}\ to the underlying Prolog for execution.
  The Prolog goal |Goal| is called in the module |eclipse|.

  This instruction is strongly discouraged. It may be disabled in a future
  version of \ProTop.
\end{description}

The \ProTop{} instructions shown in this and the previous sections have been
written with parentheses around arguments. For most commands with one argument
ist is also possible to omit the parentheses. This feature is shown in the
following example.

\begin{BoxedSample}
  ProTop -> |cd '/etc'.|
  ok.
  ProTop -> |pwd.|
  /etc
  ok.
  ProTop -> |status version.|
    This is ProTop version \Version
  ok.
\end{BoxedSample}



\section{Command Line Flags of \ProTop}

If \ProTop{} has been proberly installed then the executable\footnote{In fact
  this is an \eclipse{} saved state.} {\sf protop} should exist. This program
can be started like any other program by typing its name in a shell --- if it
is on your shells search path.  In addition the executable accepts the
following command line arguments:

\begin{description}
\item[--a]\ \\
  The backward compatibility file is loaded when this flag is given. The
  loading occurs at once. Thus it is possible to overwrite things later.
\item[--d]\ \\
  The debugging of script files is turned on. This is identical to turning on
  the option |ProTop:debug|.
\item[--f {\em file}]\ \\
  The script file {\em file}\/ is evaluated. This has the same effect as the
  \ProTop{} instruction {\bf include({\em file}\/)}. The success or failure is
  ignored. I.e. the initializing sequence continues even if the file could not
  be found.
\item[--h]\ \\ Write a short summary of the command line options and terminate
  the \ProTop{} process. The exit status 0 is returned to the operating
  system.
\item[--q]\ \\
  Ususally the user's file {\sf .protop} is evaluated after the command line
  arguments. This switch turns off this behaviour.
\item[--R]\ \\
  The user's {\sf .protop} file is loaded at once. The additional evaluation
  after the end of the command line arguments is disabled. No message is
  printed in case it is not found. Thus this switch can be used to suppress
  the message indicating a missing {\sf .protop} file.
\item[--v]\ \\ The version of \ProTop{} is printed and then the \ProTop{}
  process is terminated. The exit status 0 is returned to the operating
  system.
\item[--x]\ \\
  The \ProTop{} process is terminated. The exit status 0 is returned to the
  operating system.
\end{description}

Command line arguments which are not understood are silently ignored.

Consider the following example:

\begin{BoxedSample}
  \$ protop -R -f ancient
\end{BoxedSample}

In this example the user's {\sf .protop} file is searched at once. Then the
script file ancient is loaded. \ProTop{} comes with a script file {\sf
  ancient.pt} This script files defines some instructions which have become
obsolete. It might be neccesary to load those macros in order to run older
scripts. Thus you can see this an compatibility mode of \ProTop.




\section{The Prolog Interface}\label{ProTop:prolog.interface}

\ProTop{} instructions can also be executed from within Prolog. This is simpy
done by passing a list of instructions to the predicate |protop/1|. This
assumes that the module |protop| has been activated.
%
\begin{BoxedSample}\raggedright
\$ |eclipse|
[eclipse 1]: |ensure_loaded('ProTop/protop.pl'),|
             |use_module(protop).|
protop.cfg compiled traceable 4480 bytes in 0.02 seconds
/usr/local/lib/eclipse/3.4.5/lib/lists.pl compiled traceable ...
/usr/local/lib/eclipse/3.4.5/lib/util.pl compiled traceable ...
...
protop.pl  compiled traceable 26208 bytes in 2.65 seconds

yes.
[eclipse 2]: |protop([show(version)]).|
        This is ProTop version \Version

yes.
[eclipse 3]: 
\end{BoxedSample}

\def\PrologFILE{protop}

\Predicate protop/0().

This predicate starts an interactive session with the \ProTop{} interpreter.
Commands are read from the standart input and executed immediately. Messages
may be written to the standart output.

\Predicate protop/1(+Instructions).

This predicate takes a list of \ProTop{} instructions |Instructions| and
executes them in turn. |Instructions| can also be a symbol or string. In this
case it is interpreted as a file name and commands are executed from this
scripts file if it can be found.

This predicate succeeds if and only if the instruction does.




\chapter{Proving with ProTop}%
\begin{figure}
  \begin{center}
    \mbox{\psfig{file=skeleton.eps}}
    \caption{Proving with ProTop}\label{fig:skeleton}
  \end{center}
\end{figure}

The view of ProTop on the proving process is depicted in figure
\ref{fig:skeleton}. The starting point is a problem. This problem is usually
given in normal form as a set of clauses. Before the contents of this file is
parsed and stored in the internal matrix format it is piped through a series
of input filters. These filters provide a powerful means to preprocess the
problem. E.g. a normal form transformation can take place in a filter.

A filter communicates the result via a stream to the next filter. If no filter
is left to process the result is parsed and stored to be used by the further
phases.

One further step is a preprocessing in the classical sense. Here well known
reductions are applied and the matrix is modified accordingly.

The next step is the activation of the selected prover. Provers can be
classified into two major categories: compilers and interpreters. Compilers
generate a stand alone Prolog program which is loaded into the running ProTop
and executed.  The interpreters do not need to load a program into ProTop.
This means that interpreters work directly on the internal representation of
the problem. Alternatively external provers can be attached to ProTop. They
are also treated like interpreters.

An an example for an external prover {\em otter} has been integrated into
\ProTop. External provers are started via a system call after the problem has
been written to a file in a prover specific format. At the end the result has
to be read back in from a file and converted into something that ProTop can
work with.

Most provers can be controlled by a variety of means. Most common are flags
which can turn on or off certain features or set linits etc. The prover
\ProCom{} which was developed in parallel to \ProTop{} offers additionally the
possibility to load prolog modules controlling its behavior dynamically.
Furthermore Prolog code can be fed in by using libraries which are linked to
the compiled code.

%
% Local Variables: 
% mode: latex
% TeX-master: "manual"
% End: 
