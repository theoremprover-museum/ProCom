%%%****************************************************************************
%%% $Id: manual.tex,v 1.8 1995/04/24 21:29:11 gerd Exp $
%%%============================================================================
%%% 
%%% This file is part of ProCom.
%%% It is distributed under the GNU General Public License.
%%% See the file COPYING for details.
%%% 
%%% (c) Copyright 1995 Gerd Neugebauer
%%% 
%%% Net: gerd@imn.th-leipzig.de
%%% 
%%%****************************************************************************
\def\RCSrevision{\RCSstrip$Revision: 1.8 $}
\def\RCSdate{\RCSstrip$Date: 1995/04/24 21:29:11 $}

\documentstyle [11pt,%.......... 
                dina4,%......... use european A4 size paper
                fleqn,%......... smash mathematics leftward
                fancybox,%...... 
                floatfig,%...... floating text around figures
                psfig,%......... 
                verbatim,%...... use input of files
                pcode,%......... obviously Prolog code
                gntitle,%....... 
                old-fonts,%..... compatibility pack for xfig figures
                makeidx,%....... yes, we are making an index
                named,%......... 
                rcs,%........... decode RCS information
                psulhead,%...... use underlined heading
                idtt,%.......... use | | for verbatim
                epic%........... 
                ]{book}%..... 
\pagestyle{ulheadings}

\setlength{\headheight}{3ex}
\setlength{\parindent}{0Em}
\setlength{\parskip}{1ex}

\author{Gerd Neugebauer}
\title{   \ProCom/CaPrI
        \\and the Shell \ProTop}
\subtitle{User's Guide}
\version{\Version}
\edition{\RCSrevision}
\date{\today}
\address{Fachbereich Informatik, Mathematik und Naturwissnschaft
\\	 Hochschule f{\"u}r Technik, Wirtschaft und Kultur, Leipzig
\\       Postfach 66
\\       04251 Leipzig (Germany)
\\       Net: {\small\tt gerd@imn.th-leipzig.de}
}

\WithUnderscore{%% ATTENTION: This file has been generated automatically.
%%            Changes to this file may be overwritten.
%%            Consult INSTALL and Makefile for details.
%%
%% Last created: Thu Jul 6 11:04:26 MET DST 1995
%%      by user: gerd
%%      on host: upsilon
%%
%----------------------------------------------------------
\gdef\Version{1.69}

\gdef\ProTopFilters{%{\catcode`\_=12%
  \Filter{tptp}%
  \Filter{mult_taut_filter}%
  \Filter{mpp}%
  \Filter{tee}%
  \Filter{equality_axioms}%
  \Filter{E_flatten}%
  \Filter{constraints}%
}%}

\gdef\ProTopProvers{%{\catcode`\_=12%
  \Prover{procom}%
  \Prover{pool}%
  \Prover{otter}%
  \Prover{setheo}%
}%}

}

\newenvironment{Sample}{%
        \begin{center}
          \begin{minipage}{.5\textwidth}
            \rule{\textwidth}{.1pt}\vspace{-1ex}%
            }{%
            \vspace{-2ex}\rule{\textwidth}{.1pt}
          \end{minipage}
        \end{center}%
        }

\let\PrologFILE=\relax
\let\PredicateFileExtension=\relax

\newcommand\eclipse{ECL\(^i\)PS\(^e\)}
\newcommand\CaPrI{CaPrI}
\newcommand\ProCom{ProCom}
\newcommand\ProTop{{Pro\hspace{-.1em}Top}}

\newfont\itt{cmitt10}

\newenvironment{BoxedSample}%
{\begin{Sbox} \begin{minipage}{.8\textwidth}\obeylines\obeyspaces\itt%
}{%
 \end{minipage} \end{Sbox}%
 \begin{center}%
  \cornersize*{3ex}%
  \Ovalbox{\mbox{\TheSbox}}
 \end{center}%
}

\renewcommand\textfraction{.1}
\renewcommand\topfraction{1}
\renewcommand\bottomfraction{1}

\makeindex
\begin{document} %%%%%%%%%%%%%%%%%%%%%%%%%%%%%%%%%%%%%%%%%%%%%%%%%%%%%%%%%%%%%%
\initfloatingfigs

\maketitle
Copyright {\copyright} 1995 Gerd Neugebauer\\[.5\baselineskip]
{\small	Fachbereich Informatik, Mathematik und Naturwissnschaft
\\	Hochschule f{\"u}r Technik, Wirtschaft und Kultur, Leipzig
\\      Postfach 66
\\      04251 Leipzig (Germany)
\\      Net: {\tt gerd@imn.th-leipzig.de}
}
\vfill

Permission is granted  to make and  distribute verbatim copies of  this manual
provided the copyright notice and this permission  notice are preserved on all
copies.

%  Permission is  granted  to  process this file  through  TeX  and  print the
%  results,  provided the printed document carries a copying permission notice
%  identical  to  this  one  except for the  removal  of this paragraph  (this
%  paragraph not being relevant to the printed manual).

Permission is granted to copy and distribute  modified versions of this manual
under the conditions for verbatim  copying, provided that the entire resulting
derived work is distributed  under the terms of  a permission notice identical
to this one.

Permission is granted to copy and  distribute translations of this manual into
another language, under  the  above conditions  for modified versions,  except
that this permission  notice may be stated  in a  translation approved by  the
author.


\tableofcontents

\part{\ProTop\ for Users}

\chapter*{Introduction}
\begin{floatingfigure}{.4\textwidth}
  \begin{center}
    \mbox{\psfig{file=overview.eps}}
    \caption{Overview}\label{overview}
  \end{center}
\end{floatingfigure}

\ProCom{} is a family of theorem provers which can be used out of the box as
well as a test bed for developing own theorem provers. The techniques uses in
\ProCom{} are close to PTTP-like theorem provers \cite{stickel:prolog}. We
will not assume any knowledge on these techniques but it might help
understanding some of the details.

Implementing this paradigm means to translate a given problem into a Prolog
program which behaves like a theorem prover. A overview of this idea can be
seen in figure~\ref{overview}.

The first phase of a proof attempt takes a problem --- typically in the
language of first order logic --- and translates it into Prolog clauses. This
task is performed by the \ProCom{} system. The result is a Prolog program.
This Prolog program may contain specific constructs of the target Prolog
dialect.\footnote{Currently only \eclipse\ and Quintus Prolog are supported.}

Even so it is not necessary we will sketch the way \ProCom{} works.  After the
problem has been read and stored in matrix form, a preprocessor is applied
which performs several reductions. The reduced matrix is translated into
Prolog code in four phases. One phase takes the clauses of the matrix,
clusters them in procedures and generates code for those procedures. A second
phase translates the set of goals. Goals deserve a special treatment which is
different from usual procedures. A third phase is adds Prolog code from a
library. The final phase consists of an optimizer which tries to remove
redundancies from the generated Prolog code.

Having in mind this scheme simply using \ProCom{} goes as follows. First, a
problem is specified. This means it is written into a file in an extended
Prolog notation. Another file containing various options to control the
behavior of \ProCom{} may be used. Those input files are given to \ProCom.
The result is a stand-alone Prolog program. This Prolog program can be
executed.

As a minimal result the answer {\tt yes} or {\tt no} is given (if any).
Additional the presentation of a answer substitution or even a complete proof
tree can be achieved.

\ProTop\ is a shell which contains the functionality needed to run a theorem
prover. Even so \ProTop\ has been developed in parallel to \ProCom\ it
provides a more general environment. This is demonstrated by linking the
theorem prover otter \cite{mccune:otter} to \ProTop.

%%%****************************************************************************
%%% $Id: protop.tex,v 1.10 1995/07/03 11:35:12 gerd Exp gerd $
%%%============================================================================
%%% 
%%% This file is part of ProCom.
%%% It is distributed under the GNU General Public License.
%%% See the file COPYING for details.
%%% 
%%% (c) Copyright 1995 Gerd Neugebauer
%%% 
%%% Net: gerd@imn.th-leipzig.de
%%% 
%%%****************************************************************************

\chapter{Using \ProTop}

%------------------------------------------------------------------------------
\section{Introduction}

\ProTop{} is a shell which allows to perform all tasks necessary to run a
theorem prover. To accomplish this a set of instructions is provided to
control the actions of the prover top-level shell \ProTop. This chapter
describes the various possibilities provided by \ProTop.


\section{Interactive Use}

Two major uses of \ProTop{} can be envisioned: The interactive use and the use
as script language. First we describe  the interactive use. To start \ProTop{}
you just have to execute  the command |protop| from the  shell. For this might
be necessary  to make the location of  the executable  explicit or enlarge the
search path appropriately.

After \ProTop{} is started a welcome message is printed and a prompt appears.
The prompt is usually the string |ProTop ->|. Now you can type in commands.
Those commands are read by an interpreter and executed immediately.

Note that the reading apparatus of Prolog is used to parse the input. Thus it
is necessary to terminate any input with a full stop.

Consider the following (rather short) sample session:

\begin{BoxedSample}
\$ |protop|
\

\                --- Welcome to ProTop (\Version) ---

ProTop -> |stop.|

bye
\$ 
\end{BoxedSample}

The output of the computer is presented in a {\itt slanted teletype font}. The
input of a user is presented in an {\tt upright teletype font}.

As we see \ProTop{}   is started from the  shell.  At the prompt the   command
|stop| is  typed in. This terminates  the  session immediately and  control is
returned to the shell. Further commands and options to control the behavior of
\ProTop{} are described on the next few pages.

The most important  instructions are those which  leave \ProTop.   Most of the
possibilities one may think of are included. Thus it should  be fairly easy to
leave \ProTop.

\begin{description}
  \item [end]\index{ProTop!end}\ \\
  This instruction ends the session if executed at the top level.
  \item [stop]\index{ProTop!stop}\ \\
  This instruction ends the session if executed at the top level.
  \item [exit]\index{ProTop!exit}\ \\
  This instruction ends the session if executed at the top level.
  \item [halt]\index{ProTop!halt}\ \\
  This instruction ends the session when executed in an arbitrary context.
\end{description}

\section{Options Controlling \ProTop}

\begin{description}
  \item [ProTop:welcome] |= on|
    \\
    This option controls whether the startup message should be printed.
  \item [ProTop:debug] | = off|
    \\
    This option allows us to get more detailed information on the actions
    performed by \ProTop. This is especially useful when debugging script
    files.
  \item [ProTop:verbose] | = on|
    \\
    This option can be used to control the verbosity of the response given by
    \ProCom.
  \item [ProTop:resource\_file] | = ".protop"|
    \\
    This option specifies a file to be loaded when \ProTop{} is started.
    Usually it is only useful to redefine this option if \ProTop{} is started
    from within Prolog (see section \ref{ProTop:prolog.interface})
  \item [ProTop:path] | = [".","Scripts","~"]|
    \\
    This option specifies the search path where \ProTop\ scripts are searched
    for. It is strongly recommended to include the current directory in this
    list.
  \item [ProTop:script\_extension] | = ["",".pt"]|
    \\
    This option is used to specify the extensions which should be tried when
    searching for \ProTop{} script files.
  \item [ProTop:prover\_file] | = "...prover.pl"|
    \\
    This option specifies the intermediate file name of the prover.
  \item [ProTop:backup] | = on|\label{ProTop:backup}
    \\
    This option controls the action performed when discovering an old report
    file. If this option is turned on then the old report file is moved aside
    by appending some (negative) version number to it. Otherwise the old
    report file is overwritten.
\end{description}

\section{Getting Information About \ProTop}

\begin{description}
  \item [help]\index{ProTop!help}\ 
    \\
    This instruction prints a short information which should help you to carry
    on. The help command can also be given an argument. In this case some
    information on the argument is given. This is the same as executing the
    |status| command with an argument. For details see the description of
    |status|.
  %
\begin{BoxedSample}
  ProTop -> |help.|
  The command `show' can be used to get information on the
  system as a whole or on parts of it.
    show.
  presents the status of some important parts of ProTop.

    show commands.
  gives a list of all commands available.

  To leave ProTop type
    halt.
  ok.
  ProTop ->
\end{BoxedSample}

  \item [status]\index{ProTop!status}\ 
    \\
    This instruction prints the status of various parts of \ProTop{} to the
    scrren. The status of parts can be displayed seperately by the next
    instruction.
  %
\begin{BoxedSample}\def\Prover#1{          #1\par}\let\Filter\Prover
  ProTop -> |status.|
    This is ProTop version \Version
    Filters:
\ProTopFilters
          none
    Provers:
\ProTopProvers
    Matrix:
          No matrix loaded.
    Macros:
          0 macros are defined.
  ok.
\end{BoxedSample}

  \item [status({\em Part})]\index{ProTop!status}\ 
    \\
    This instruction displays the status of the requested part {\em Part} to
    the screen. {\em Part}\/ can also be a list of parts in which case the
    status of all parts given in the list is displayed. The following parts
    can be specified:
    \begin{description}
    \item [version]\ \\ prints the version of \ProTop.
    \item [macros]\ \\  prints a summary of macros defined in \ProTop.
    \item [prover]\ \\  prints a list of provers loaded.
    \item [filter]\ \\  prints a list of filters loaded.
    \item [matrix]\ \\  prints a summary of the matrix currently loaded.
    \item [clauses]\ \\ prints a list of clauses currently stored in the
      matrix.
    \item [contrapositives]\ \\ prints a list of contrapositives currently
      stored in the matrix.
    \item [modules]\ \\ prints a list of the loaded modules and the associated
      documentation strings.
    \item [module({\em Module}\/)]\ \\ prints documentation string for module
      {\em Module}. It fails if {\em Module}\/ is not instanciated properly or
      it contains no \ProTop{} module.
    \item [options]\ \\ prints a list of all options.
    \item [option({\em Option}\/)]\ \\ prints the value of the option {\em
        Option}. 
    \item [file({\em File}\/)]\ \\ displays the contents of the file {\em
        File} on the screen.
    \item [macros]\ \\ prints the list of all macros to the screen.
    \item [macro({\em Macro}\/)]\ \\ displays the definition of all macros for
      which the head unify with {\em Macro}.
    \item [all]\ \\     does {\bf version}, {\bf filter}, {\bf prover}, {\bf
        matrix} and {\bf macros}.
    \end{description}

    This instruction fails if one of the requested parts is not known or it is
    called with a variable argument.

  \item [info]\index{ProTop!info}\ 
    \\
    This is a variant of |status|. For details see above.

  \item [list]\index{ProTop!list}\ 
    \\
    This is a variant of |status|. For details see above.

  \item [show]\index{ProTop!show}\ 
    \\
    This is a variant of |status|. For details see above.

\end{description}


\section{Script Files}

Script files provide a means to keep a sequence of instructions beyond the end
of a single session.

One place where a script file is involved is at the startup of \ProTop. At
this occasion the file {\sf .protop} is sought in the current directory and in
the home directory of the user and instructions from this file are executed if
it is found. The primary intension is to allow the user to specify some
options to adapt the behavior of \ProTop. Nevertheless arbitrary instructions
can be placed there to be executed at startup time.

The following commands are related to script files

\begin{description}
  \item [include({\em File})]\index{ProTop!include}\ 
    \\
    This instruction reads commands from the file {\em File} and executes
    them.  The option |ProTop:path| can be used to specify the search path.
    The option |ProTop:script\_ext| can be used to specify the extensions to
    be appended when searching for the actual file.
  %
\begin{BoxedSample}
  ProTop -> |include("some_script").|
  |---| ProTop script file some\_script not found.
  no.
  ProTop -> |include(".protop").|
  ok.
  ProTop -> 
\end{BoxedSample}

  \item ["{\em File}"]\index{ProTop!include}\ 
    \\
    A file name marked as a string --- i.e. enclosed in double quotes --- Is
    taken as a file name to be included. Thus it is equivalent to
    {\bf include({\em File}\/)}.

  \item [end\_of\_file]\index{ProTop!end\_of\_file}\ \\
  This token is returned by the Prolog reading apparatus upon end of file.
  Thus it can be used in this sense. Upon end of file the session is usually
  ended. If encountered in a script file the rest of the file is ignored.
\end{description}

\section{Output and Input}

For convenience some commands of the underlying Prolog are made available as
\ProTop{} instructions. They can be used to write interactive script files.

\begin{description}
  \item [nl]\index{ProTop!nl}\ \\
  This instruction writes a newline to the screen.
  \item [write({\em Term})]\index{ProTop!write}\ \\
  This instruction writes {\em Term}\/ to the screen. {\em Term} is {\sc not}
  followed by a newline.
  %
\begin{BoxedSample}
  ProTop -> |write("hello world.").|
  hello world.ok.
  ProTop -> 
\end{BoxedSample}

  \item [writeln({\em Term})]\index{ProTop!writeln}\ \\
  This instruction writes {\em Term}\/ to the screen. {\em Term} is followed
  by a newline.
  %
\begin{BoxedSample}
  ProTop -> |writeln("hello world.").|
  hello world.
  ok. 
\end{BoxedSample}

  \item [printf({\em Format}, {\em Arguments})]\index{ProTop!printf}\ \\
  This instructions prints the {\em Arguments} according to the format string
  {\em Format}. {\em Format} is a string where \% is used as escape to specify
  the form of an argument. {\em Arguments} is a list of terms to be printed
  according to the format string. For details see the documentation of the
  \eclipse{} predicate |printf/2|.
  %
\begin{BoxedSample}
  ProTop -> |printf(".....%w.....%w.....\n",[1,2]).|
  .....1.....2.....
  ok.
\end{BoxedSample}

  \item [read\_string({\em String})]\index{ProTop!read\_string}\ \\
  This reads all characters up to the next newline into the string {\em
  String}. The end of file also terminates the reading.
  %
\begin{BoxedSample}
  ProTop -> |read_string(String), writeln(String).|
   |some string terminated by newline|
  some string terminated by newline
  ok.
\end{BoxedSample}

  \item [read\_string({\em Prompt}, {\em String})]\index{ProTop!read\_string}\
  \\
  This instruction writes {\em Prompt} to the screen and reads all characters
  up to the next newline into the string {\em String}. The end of file also
  terminates the reading.
  %
\begin{BoxedSample}
  ProTop -> |read_string("Enter a string:",String).|
  Enter a string: |some string terminated by newline|
  ok.
\end{BoxedSample}

  \item [read({\em Term})]\index{ProTop!read}\ \\
  This instruction reads a prolog term from the standard input and unifies it
  with {\em Term}. If the unification fails then this command fails. The term
  has to be terminated by a colon.
  %
\begin{BoxedSample}
  ProTop -> |read(Term), writeln(Term).|
   |some_term(Var,const).|
  some\_term(Var, const)
  ok.
  ProTop -> |read(term), writeln(Term).|
   |some_term(Var,const).|
  no.
\end{BoxedSample}

  \item [read({\em Prompt}, {\em Term})]\index{ProTop!read}\ \\
  This instruction writes {\em Prompt} to the screen and reads a prolog term
  from the standard input and unifies it with {\em Term}. If the unification
  fails then this command fails. The term has to be terminated by a colon.
  %
\begin{BoxedSample}
  ProTop -> |read("Term: ",Term), writeln(Term).|
  Term:  |some_term(Var,const).|
  some\_term(Var, const)
  ok.
\end{BoxedSample}

\end{description}

\section{Flow of Control}

\begin{description}
  \item [{[]}]\index{ProTop![]}\ \\
  An empty list of instructions simply succeeds.
  %
\begin{BoxedSample}
  ProTop -> |[].|
  ok.
\end{BoxedSample}

  \item [{[{\em Instruction}, \ldots, {\em Instruction}]}]\index{ProTop![...]}\ \\
  A list of instructions is executed like a conjunction (see below).
  %
\begin{BoxedSample}
  ProTop -> |[write(1),write(2),nl].|
  12
  ok.
\end{BoxedSample}

  \item [({\em Instruction}, \ldots, {\em Instruction})]\index{ProTop!(,)}\ \\
  Conjunctions are executed from left to right until one instruction fails.
  In this case the whole conjunction fails. If no instruction fails then
  the conjunction succeeds.
  %
\begin{BoxedSample}
  ProTop -> |write(1),write(2),nl.|
  12
  ok.
\end{BoxedSample}

  \item [({\em Instruction}; \ldots; {\em Instruction})]\index{ProTop!(;)}\ 
    \\
    Disjunctions are executed from left to right until one instruction
    succeeds.  In this case the whole disjunction succeeds. If no instruction
    succeeds then the disjunction fails.
  %
\begin{BoxedSample}
  ProTop -> |(writeln(first),fail); writeln(second).|
  first
  second
  ok.
\end{BoxedSample}

  \item [repeat {\em N}\/ times {\em Command}]\index{ProTop!repeat}\ 
    \\
    This instruction repeats the execution of {\em Command}\/ several times.
    {\em N}\/ is an integer denoting the number of repetitions.
  %
\begin{BoxedSample}
  ProTop -> |repeat 5 times write("----+").|
  ----+----+----+----+----+ok.
  ProTop ->       
\end{BoxedSample}

  \item [for {\em Var}\/ in {\em Spec}\/ do {\em Command}]\index{ProTop!for}\ 
    \\
    This instruction repeats {\em Command}\/ for all elements of {\em Spec}\/
    assigned to {\em Var}. This is done in a failure driven manor, i.e.
    variable bindings are undone after each loop. {\em Spec}\/ is expanded to
    a list before the loop is performed. The following constructs for {\em
      Spec}\/ are supported:
  \begin{description}
    \item [{[{\em ...}]}]\ \\
    A list of any elements is the simplest form of a specification. No
    expansion takes place in this case.
    %
    \begin{BoxedSample}
ProTop -> |for Var in [1,2,3] do writeln(Var).|
1
2
3
ok.
    \end{BoxedSample}

    \item [files({\em Dir})]\ 
      \\
      This specification expands to a list of all files in the directory {\em
        Dir}. Note that the files are not sorted.
    %
\begin{BoxedSample}
  ProTop -> |for F in files(".") do writeln(F).|
  ./COPYING
  ./README
  ./INSTALL
  ./Makefile
  ...
  ok.
\end{BoxedSample}

    \item [files({\em Dir}, {\em Pattern})]\ 
      \\
      This specification expands to a list of all files in the directory {\em
        Dir} which match the pattern {\em Pattern}. The syntax of pattern is
      the syntax allowed for Bourne shell commands.
    %
\begin{BoxedSample}
  ProTop -> |for F in files(".","*.pl") do writeln(F).|
  ./main.pl
  ./...prover.pl
  ok.
\end{BoxedSample}

    \item [directories(Dir)]\ 
      \\
      This specification expands to a list of all directories in the directory
      {\em Dir}.
    %
\begin{BoxedSample}
  ProTop -> |for D in directories(".") do writeln(D).|
  ./Doc
  ./System
  ./ProCom
  ./Reductions
  ./Capri
  ./inputs
  ...
  ok.
\end{BoxedSample}

  \end{description}

  \item [if {\em Condition}\/ then {\em Then}\/ else {\em
      Else}]\index{ProTop!if}\ 
    \\
    This instruction evaluates the commands {\em Condition}. If this
    evaluation succeeds then the commands {\em Then}\/ are evaluated
    afterwards. Otherwise the commands {\em Else}\/ are executed.
    %
\begin{BoxedSample}
  ProTop -> |if exists("/etc/motd")|
            |then show(file("/etc/motd"))|
            |else writeln("Sorry").|
  SunOS Release 4.1.3 (GENERIC) \#3: Mon Jul 27 16:44:16 PDT 1992
  ok. 
\end{BoxedSample}

  \item [if {\em Condition}\/ then {\em Then}]\index{ProTop!if}\ 
    \\
    This instruction evaluates the commands {\em Condition}. If this
    evaluation succeeds then the commands {\em Then}\/ are evaluated
    afterwards. Otherwise it succeeds without performing any other actions.
    %
\begin{BoxedSample}
  ProTop -> |if not(exists("/etc/motd")) then|
              |writeln("Sorry").|
  ok.
\end{BoxedSample}

  \item [true]\index{ProTop!true}\ \\
  This instruction always succeeds and does nothing else.
  %
\begin{BoxedSample}
  ProTop -> |true.|
  ok.
\end{BoxedSample}

  \item [fail]\index{ProTop!fail}\ \\
  This instruction always fails and does nothing else.
  %
\begin{BoxedSample}
  ProTop -> |fail.|
  no.
\end{BoxedSample}

  \item [exists({\em File})]\index{ProTop!exists}\ \\
  This instruction checks the existence of the file {\em File}. {\em File}
  must be a symbol or string. Otherwise a type error is raised and the
  instruction fails.
  %
\begin{BoxedSample}
  ProTop -> |exists("~/.login").|
  ok.
  ProTop -> |exists("~/.protop").|
  no.
\end{BoxedSample}

  \item [not({\em Condition}\/)]\index{ProTop!not}\ 
    \\
    This instruction executes {\em Condition}\/ and reverses the return
    status.  I.e. it succeeds if {\em Condition}\/ fails and vice versa.
  %
\begin{BoxedSample}
  ProTop -> |not(true).|
  no.
  ProTop -> |not(fail).|
  ok.
\end{BoxedSample}

  \item [find\_file({\em File},{\em Path},{\em Extensions},{\em
      FullFile})]\index{ProTop!find\_file}\ 
    \\
    This instruction tries to find the existing file {\em File} in a list of
    directories {\em Path}. A list of extensions {\em Extensions}\/ is used to
    augment the file name with an additional extension. The result is unified
    with {\em FullFile}.
  %
\begin{BoxedSample}
  ProTop -> |find_file(".PROTOP",[".","~"],[""],File),|
            |writeln(File).|
  no.
  ProTop -> |find_file(".protop",[".","~"],[""],File),|
            |writeln(File).|
  ./.protop
  ok.
\end{BoxedSample}
  
  \end{description}

\section{Macros}

Macros are a means to ease live. Tasks which have to be performed repeatedly
can be defined as a macro and executed when necessary. The simplest case is
a macro as abbreviation for some other instructions as in the following
example:
\begin{BoxedSample}
  ProTop -> |my_macro := write("hello "),|
  |            write("world"), write("."), nl.|
  ok.
  ProTop -> |my_macro.|
  hello world.
  ok.
\end{BoxedSample}
Ok, it's rather simple. This macro simply writes |hello world.| followed by a
newline --- an effect one can achieve simpler. The execution is triggered by a
call with its name.

Macros may also have arguments. Prolog variables are used to denote variable
parts of the arguments. Let us again consider out previous example and make it
a little bit more complicated:
\begin{BoxedSample}
  ProTop -> |my_macro(Arg) :=|
  |           write("hello "), write(Arg), write("."), nl.|
  ok.
  ProTop ->
\end{BoxedSample}

Note that this macro can coexist with the previous one since the number of
actual parameters distinguishes them. If we call this macro with the command
|my_macro("world").| we get the same as before. But now we can also specify
another argument, as in
\begin{BoxedSample}
  ProTop -> |my_macro("world").|
  hello world.
  ok.
  ProTop -> |my_macro("friends").|
  hello friends.
  ok.
\end{BoxedSample}

The coexistence of macros goes further. The selection of an appropriate macro
is done via unification of its head. We exploit this fact by further
developing our example:
\begin{BoxedSample}
  ProTop -> |my_macro(all,999) := my_macro("world").|
  ok.
  ProTop -> |my_macro(some,12) := my_macro("friends").|
  ok.
\end{BoxedSample}

These two instructions establish {\em two} definitions of |my_macro/2|. Now we
can test them and see how they work:

\begin{BoxedSample}
  ProTop -> |my_macro(all,X).|
  hello world.
  ok.
  ProTop -> |my_macro(some,X).|
  hello friends.
  ok.
  ProTop -> |my_macro(A,B), writeln('A'=A), writeln('B'=B).|
  hello world.
  A = all
  B = 999
  ok.
\end{BoxedSample}

To conclude we summarize all macro related commands we have already seen and
some others, not mentioned yet.

\begin{description}
  \item [{\em Macro} := {\em Code}]\index{ProTop!:=}\ \\
  This instruction stores {\em Code} as replacement text for {\em Macro}.
  Several clauses for the same macro can be specified.

  \item [delete macro({\em Macro})]\index{ProTop!delete macro}\ \\
  This instruction deletes all macros that unify with {\em Macro}. As a
  consequence we can delete all macros with the command
  %
\begin{BoxedSample}
  ProTop -> |delete macro(_).|
  ok.
\end{BoxedSample}

  Note that the underscore |_| is the anonymous variable like in Prolog.
  A more natural use would be the following one:
  %
\begin{BoxedSample}
  ProTop -> |delete macro(my_private_macro(_,_)).|
  ok.
\end{BoxedSample}

  This example deletes all entries for the macro |my_private_macro| with two
  arguments.

  \item [show macro({\em Macro})]\index{ProTop!show macro}\ \\
  This instruction lists all macros which unify with {\em Macro}. If none is
  found then an informative message is displayed.
  %
\begin{BoxedSample}
  ProTop -> |show macro(my_macro).|
  my\_macro := write(hello ), write(world), write(.), nl
  ok.
  ProTop -> 
\end{BoxedSample}

  \item [defined({\em Macro})]\index{ProTop!defined}\ \\
  This instruction checks for the existence of macros which unify with {\em
  Macro}. The return status indicates success or failure.
  %
\begin{BoxedSample}
  ProTop -> |defined(mac).|
  no.
  ProTop -> |mac := true.|
  ok.
  ProTop -> |defined(mac).|
  ok.
\end{BoxedSample}

  \item [{\em Macro}]\ \\
  A macro is executed when its name is encountered as command. Since the
  built-in commands have precedence over macros it is not possible to redefine
  a built-in.

  \item [op({\em Precedence}, {\em Accosiativity}. {\em Operator})]\ 
    \index{ProTop!op}\\
    This instruction can be used to define operators like in Prolog. This
    feature can be used to make the input to \ProTop{} better readable for
    humans. 
  %
\begin{BoxedSample}
  ProTop -> |op(900,fx,say).|
  ok.
  ProTop -> |say(X) := writeln(X).|
  ok.
  ProTop -> |say hello.|
  hello
  ok.
\end{BoxedSample}
\end{description}

\section{Options Management}

\begin{description}
  \item [{\em Option} = {\em Value}]\index{ProTop!=}\ \\
  This instruction sets or gets the value of the option {\em Option}. If {\em
  Value} is not a variable then this value is stored in {\em Option}.
  Otherwise {\em Value}\/ is unified with the actual value of {\em Option}.
  {\em Option} is required to be a symbol.

\begin{BoxedSample}
  ProTop -> |verbose = off.|
  ok.
  ProTop -> |verbose = VERB, writeln(VERB).|
  off
  ok.
\end{BoxedSample}

  \item [reset\_options]\index{ProTop!reset\_options}\ \\
  This instruction resets all options to their default values. It always
  succeeds and produces no output.

  \item [list options]\index{ProTop!list options}\ \\
  This instruction prints a list of all options together with their values to
  the screen.

  \item [list option({\em Option})]\index{ProTop!list option}\ \\
  This instruction prints a single option {\em Option}\/ and its value to the
  screen.
  %
\begin{BoxedSample}
  ProTop -> |list option prover.|
         prover = procom(extension\_procedure)
  ok.
\end{BoxedSample}

\end{description}

\section{Proving}

\begin{description}

  \item [prove({\em File})]\index{ProTop!prove}\ \\
  This instruction performs compilation and running of the compiled prover.
\end{description}


\section{Report Generation}


\begin{description}
  \item [report:file] |= "report.tex"|\\
  This option names the report file to be generated.

  \item [report:ignore\_options]|= ['run:remove_prover',match('ProTop'),...]|\\
  This option contains a list of specifications for options which should not be
  carried over to the report file. Several types of elements of such a list
  are recognized:
  \begin{description}
    \item [{\em 'symbol'}] A symbol selects the option named.
    \item [match({\em 'symbol'})] A match selects all options contain {\em
    symbol} as substring.
  \end{description}

  \item [report:style\_options] |= ["p-r"]|\\
  This option contains a list of style options (their names as strings) which
  are used when generating a proof report. |"p-r"| is needed to print the
  constructs in the generates \LaTeX{} file. You may consider providing an
  alternative style options file to change the appearance. Otherwise you can
  add options like |"11pt"| to modify the size of the fonts used etc.

  \item [report:style\_path] |= "..."|\\
  This option names the place where additional style files for \LaTeX{} can be
  found. It is automatically set during the installation process to point into
  the installation directory (in fact the {\sf input} subdirectory).
  The value of this option is a string.

  Several paths can be given by separating them with a colon (:).

  \item [report:latex] |= latex|\\
  This option names a command executable in the Bourne shell (sh) to start
  \LaTeX{}. It may be necessary to specify the full path name if the command
  is not on the shell search path (environment variable |$PATH|).%$

  \item [report:flags] |= [preamble,titlepage,text,comment,time,options,matrix,tree]|\\
  This option controlls which parts of the proof report are to be generated.
  It contains a list of the following key words:

  \def\X[#1]{{\bf #1}&}
  \begin{tabular}{lp{.7\textwidth}}
    \X[preamble]                \\
    \X[titlepage]               \\
    \X[text]            \\
    \X[comment]         \\
    \X[time]            \\
    \X[options]         \\
    \X[matrix]          \\
    \X[tree]
  \end{tabular}

  \item [report:sections] |= [comment, text, matrix, options, times, proof]|\\
  This option controlls which subsections to include for each run of the
  prover. The following key words are recognized:

  \begin{tabular}{lp{.7\textwidth}}
    \X[comment] Include the comments as \TeX\ comments. \\
    \X[text]    Include the descriptive text. \\
    \X[matrix]  Include the internal representation of the matrix. \\
    \X[options] Include a list of options into the report. \\
    \X[times]   Include a table of times into the report. \\
    \X[proof]   Include the proof (tree/description) into the report.
  \end{tabular}

  \item [report:tree\_flags] |= [dx(100),dy(100),tree,info]|\\
  This option contains a list of keywords controlling the appearance of the
  proof. The following key words are recognized:

  \begin{tabular}{lp{.7\textwidth}}
    \X[tree]            \\
    \X[info]            \\
    \X[dx($\Delta x$)]          \\
    \X[dy($\Delta y$)]
  \end{tabular}

\end{description}


\begin{description}
  \item [report init({\em ReportFile})]\index{ProTop!report init}\ \\
  This instruction initializes the report file {\em ReportFile}. Subsequent
  actions may leave a message in this file. Depending on the value of the
  option |ProTop:backup| an existing file is saved or overwritten (see page
  \pageref{ProTop:backup}).

  \item [report init]\index{ProTop!report init}\ \\
  This instruction initializes the report file from the option |log_file|. See
  also above.

  \item [report author({\em Text})]\index{ProTop!report author}\ \\
  This instruction writes the text {\em Text} to the report file. This text is
  intended to be used as title of an automatically generated report.

  \item [report title({\em Text})]\index{ProTop!report title}\ \\
  This instruction writes the text {\em Text} to the report file. This text is
  intended to be used as author of an automatically generated report.

  \item [report comment({\em Text})]\index{ProTop!report comment}\ \\
  This instruction writes the text {\em Text} as comment to the report file.
  This report file has to be initialized with |report_init| before this
  instruction is used.

  \item [report text({\em Text})]\index{ProTop!report text}\ \\
  This instruction writes the text {\em Text} as descriptive text to the
  report file. The report file has to be initialized with |report_init|
  before this instruction is used. 

  \item [report section({\em Text})]\index{ProTop!report section}\ \\
  This instruction writes the text {\em Text} as short title of the following
  proof attempt to the report file. It may also be used when tables are
  typeset.  The report file has to be initialized with |report_init|
  before this instruction is used. 

  \item [report label({\em Label})]\index{ProTop!report label} This
  instruction writes the string {\em Label} as identifier for the following
  proof attempt to the report file. It is also used when tables are typeset.
  The report file has to be initialized with |report_init| before this
  instruction is used.

  \item [define\_table({\em Name}, {\em Spec})]\index{ProTop!define\_table}

  \item [remove\_table({\em Name})]\index{ProTop!remove\_table}

  \item [generate\_report]\index{ProTop!generate\_report}\ \\
  This instruction reads the log file and generated the \LaTeX{} source from
  it. This report file can be processed using |latex_report| of processed by
  \LaTeX{} manually.

  {\bf Note:} A \LaTeX{} style file is required to process the generated
  \LaTeX{} code. This style file can be found in the {\sf input} subdirectory
  of the \ProTop{} installation directory.

  \item [generate\_report({\em Files})]\index{ProTop!generate\_report}\ \\
  This instruction reads the file {\em Files} and generated the \LaTeX{}
  source from it.

  \item [latex\_report]\index{ProTop!latex\_report}\ \\
  This instruction tries to run \LaTeX{} on a formerly generated report.

  \item [make\_report]\index{ProTop!make\_report}\ \\
  This instruction generates a report and runs \LaTeX{} on the resulting
  source file. The file given in the option |log_file| is used for analysis.

  \item [make\_report({\em Files})]\index{ProTop!make\_report}\ \\
  This instruction generates a report from the files {\em Files}\ and runs
  \LaTeX{} on the resulting source file.

\end{description}

\section{Misc Instructions}

\begin{description}
  \item [pwd]\index{ProTop!pwd}\ 
    \\
    This instruction prints the current working directory to the standard
    output stream.
  %
\begin{BoxedSample}
  ProTop -> |pwd.|
  /system/ProCom/
  ok.
\end{BoxedSample}

  \item [pwd(Dir)]\index{ProTop!pwd}\ 
    \\
    This instruction unifies it's argument with the current directory. Thus it
    is possible to get your hands on the current directory in script files.
  %
\begin{BoxedSample}
  ProTop -> |pwd(X).|
  ok.
  ProTop -> |pwd(X),writeln(X).|
  /system/ProCom/
  ok.
\end{BoxedSample}

  \item [cd(Dir)]\index{ProTop!cd}\ \\
  This instruction tries to set the current directory to the one given as
  argument. |Dir| has to be a string or atom which corresponds to a valid
  directory name.
  %
\begin{BoxedSample}
  ProTop -> |pwd.|
  /system/ProCom/
  ok.
\end{BoxedSample}

  \item [ls]\index{ProTop!ls}\ \\
  This instruction lists all files in the current directory on the standard
  output stream.


  \item [ls(Dir)]\index{ProTop!ls}\ \\
  This instruction lists all files in the directory |Dir| on the standard
  output stream.

  \item [call({\em Goal})]\index{ProTop!eval}\ \\
  This instruction forwards {\em Goal}\ to the underlying Prolog for execution.
  The Prolog goal |Goal| is called in the module |eclipse|.

  This instruction is strongly discouraged. It may be disabled in a future
  version of \ProTop.
\end{description}

The \ProTop{} instructions shown in this and the previous sections have been
written with parentheses around arguments. For most commands with one argument
ist is also possible to omit the parentheses. This feature is shown in the
following example.

\begin{BoxedSample}
  ProTop -> |cd '/etc'.|
  ok.
  ProTop -> |pwd.|
  /etc
  ok.
  ProTop -> |status version.|
    This is ProTop version \Version
  ok.
\end{BoxedSample}



\section{Command Line Flags of \ProTop}

If \ProTop{} has been proberly installed then the executable\footnote{In fact
  this is an \eclipse{} saved state.} {\sf protop} should exist. This program
can be started like any other program by typing its name in a shell --- if it
is on your shells search path.  In addition the executable accepts the
following command line arguments:

\begin{description}
\item[--a]\ \\
  The backward compatibility file is loaded when this flag is given. The
  loading occurs at once. Thus it is possible to overwrite things later.
\item[--d]\ \\
  The debugging of script files is turned on. This is identical to turning on
  the option |ProTop:debug|.
\item[--f {\em file}]\ \\
  The script file {\em file}\/ is evaluated. This has the same effect as the
  \ProTop{} instruction {\bf include({\em file}\/)}. The success or failure is
  ignored. I.e. the initializing sequence continues even if the file could not
  be found.
\item[--h]\ \\ Write a short summary of the command line options and terminate
  the \ProTop{} process. The exit status 0 is returned to the operating
  system.
\item[--q]\ \\
  Ususally the user's file {\sf .protop} is evaluated after the command line
  arguments. This switch turns off this behaviour.
\item[--R]\ \\
  The user's {\sf .protop} file is loaded at once. The additional evaluation
  after the end of the command line arguments is disabled. No message is
  printed in case it is not found. Thus this switch can be used to suppress
  the message indicating a missing {\sf .protop} file.
\item[--v]\ \\ The version of \ProTop{} is printed and then the \ProTop{}
  process is terminated. The exit status 0 is returned to the operating
  system.
\item[--x]\ \\
  The \ProTop{} process is terminated. The exit status 0 is returned to the
  operating system.
\end{description}

Command line arguments which are not understood are silently ignored.

Consider the following example:

\begin{BoxedSample}
  \$ protop -R -f ancient
\end{BoxedSample}

In this example the user's {\sf .protop} file is searched at once. Then the
script file ancient is loaded. \ProTop{} comes with a script file {\sf
  ancient.pt} This script files defines some instructions which have become
obsolete. It might be neccesary to load those macros in order to run older
scripts. Thus you can see this an compatibility mode of \ProTop.




\section{The Prolog Interface}\label{ProTop:prolog.interface}

\ProTop{} instructions can also be executed from within Prolog. This is simpy
done by passing a list of instructions to the predicate |protop/1|. This
assumes that the module |protop| has been activated.
%
\begin{BoxedSample}\raggedright
\$ |eclipse|
[eclipse 1]: |ensure_loaded('ProTop/protop.pl'),|
             |use_module(protop).|
protop.cfg compiled traceable 4480 bytes in 0.02 seconds
/usr/local/lib/eclipse/3.4.5/lib/lists.pl compiled traceable ...
/usr/local/lib/eclipse/3.4.5/lib/util.pl compiled traceable ...
...
protop.pl  compiled traceable 26208 bytes in 2.65 seconds

yes.
[eclipse 2]: |protop([show(version)]).|
        This is ProTop version \Version

yes.
[eclipse 3]: 
\end{BoxedSample}

\def\PrologFILE{protop}

\Predicate protop/0().

This predicate starts an interactive session with the \ProTop{} interpreter.
Commands are read from the standart input and executed immediately. Messages
may be written to the standart output.

\Predicate protop/1(+Instructions).

This predicate takes a list of \ProTop{} instructions |Instructions| and
executes them in turn. |Instructions| can also be a symbol or string. In this
case it is interpreted as a file name and commands are executed from this
scripts file if it can be found.

This predicate succeeds if and only if the instruction does.




\chapter{Proving with ProTop}%
\begin{figure}
  \begin{center}
    \mbox{\psfig{file=skeleton.eps}}
    \caption{Proving with ProTop}\label{fig:skeleton}
  \end{center}
\end{figure}

The view of ProTop on the proving process is depicted in figure
\ref{fig:skeleton}. The starting point is a problem. This problem is usually
given in normal form as a set of clauses. Before the contents of this file is
parsed and stored in the internal matrix format it is piped through a series
of input filters. These filters provide a powerful means to preprocess the
problem. E.g. a normal form transformation can take place in a filter.

A filter communicates the result via a stream to the next filter. If no filter
is left to process the result is parsed and stored to be used by the further
phases.

One further step is a preprocessing in the classical sense. Here well known
reductions are applied and the matrix is modified accordingly.

The next step is the activation of the selected prover. Provers can be
classified into two major categories: compilers and interpreters. Compilers
generate a stand alone Prolog program which is loaded into the running ProTop
and executed.  The interpreters do not need to load a program into ProTop.
This means that interpreters work directly on the internal representation of
the problem. Alternatively external provers can be attached to ProTop. They
are also treated like interpreters.

An an example for an external prover {\em otter} has been integrated into
\ProTop. External provers are started via a system call after the problem has
been written to a file in a prover specific format. At the end the result has
to be read back in from a file and converted into something that ProTop can
work with.

Most provers can be controlled by a variety of means. Most common are flags
which can turn on or off certain features or set linits etc. The prover
\ProCom{} which was developed in parallel to \ProTop{} offers additionally the
possibility to load prolog modules controlling its behavior dynamically.
Furthermore Prolog code can be fed in by using libraries which are linked to
the compiled code.

%
% Local Variables: 
% mode: latex
% TeX-master: "manual"
% End: 


%%%****************************************************************************
%%% $Id: syntax.tex,v 1.2 1995/01/27 13:45:38 gerd Exp $
%%%============================================================================
%%% 
%%% This file is part of ProCom.
%%% It is distributed under the GNU General Public License.
%%% See the file COPYING for details.
%%% 
%%% (c) Copyright 1995 Gerd Neugebauer
%%% 
%%% Net: gerd@imn.th-leipzig.de
%%% 
%%%****************************************************************************

\def\SYNTAX#1.{\begin{itemize}\item[]\(#1\) \end{itemize}}

%------------------------------------------------------------------------------
\section{The Input Language}

\subsection{Basics}

\ProCom{} is  currently able to process problems  in  clausal form. The chosen
syntax is a superset of pure  Prolog, i.e. without  built-in predicates of any
kind.  Since  the  Prolog  reading  apparatus  is  used  several features  are
inherited directly:
\begin{description}
  \item [Variables\index{variable}] are written with an initial capital letter.

  \item [Functors\index{functor}  and constants\index{constant}] usually start
    with  a lower  case letter.  Additionally  some  predefined  operators are
    defined as infix, e.g. |=|, |<|, |>|, and others.

  \item [Terms\index{term}] are built from functors and variables as usual (in
    Prolog).

  \item [Comments\index{comment}] are anything starting with a |%| up to the
    end  of the line.  Additionally  C-style comments, i.e. starting with |/*|
    and ending in |*/|, can be used.
\end{description}


\subsection{Literals}
\index{literal}

A  literal  is a predicate  with  an optional sign.   Everything  which has an
operator not  treated special is considered  as a predicate. Special operators
are |,|, |;|, |:-|, |<-|, |->|, |?-|, |::|, the list constructor and the empty
list constant |[]|.

A positive literal can  be optionally marked with  the prefix |++|. Internally
the positive sign  is added where required. The  negation sign is |-| or |--|.
Internally |--| is used only.


\subsection{Clauses}
\index{clause}

\SYNTAX |[| L_1 |,| \ldots|,| L_n |]|.
%
The simplest  --- but  hard   to read ---  form  of  a  clause consists of   a
(nonempty) list of literals. They are stored as they are.


\SYNTAX H_1|;|\ldots|;|H_n \,|:-|\, T_1|,|\ldots|,|T_m.
%
This is the general  form of a clause  in  extended Prolog notation.  The head
literals $H_1,\ldots  H_n$  denote  the   negative  literals. Thus  they   are
implicitly  negated when   the internal list  representation is  constructed.
Nevertheless explicit negation using the |-| operator is permitted.

Other degenerate variants --- the head or the tail are  empty --- are provided
as well.


\SYNTAX |:-|\, T_1|,|\ldots|,|T_m.
%
This rule parses the degenerate form of the extended Prolog notation where the
head is empty. Note that  in contrast to Prolog there  is a difference to  the
|?-| operator.


\SYNTAX  H_1|;|\ldots|;|H_n \,|<-|\, T_1|,|\ldots|,|T_m.
%
The forms incorporating the |<-| operator are simply syntactic variants of the
same forms using the |:-| operator.


\SYNTAX |<-|\, T_1|,|\ldots|,|T_m.
%
This form is simply mapped to the corresponding |:-| variant.


\SYNTAX |?-| Goal.
%


\SYNTAX L_1 |;|\ldots|;| L_n.
\SYNTAX L_1 |,|\ldots|,| L_n.
%
A disjunction  is interpreted as  a degenerated left  hand side of an extended
Prolog clause. Thus it is negated while translated into a list.


\subsection{Labels}
\index{label}

\SYNTAX Label |::| Clause.
%
Any clause can have arbitrary information assigned to it.  This is done in two
forms. One form is to mark a clause as goal clause with the |?-| operator. The
more general form implemented here is to write a label  in front of the clause
separated  by the |::| operator.   The label {\em Label}\/  is  stored in  the
Prolog database as |'Label'(|{\em Label}|,|{\em Index}|)|.



%%%****************************************************************************
%%% $Id: filters.tex,v 1.4 1995/04/24 21:29:11 gerd Exp $
%%%============================================================================
%%% 
%%% This file is part of ProCom.
%%% It is distributed under the GNU General Public License.
%%% See the file COPYING for details.
%%% 
%%% (c) Copyright 1995 Gerd Neugebauer
%%% 
%%% Net: gerd@imn.th-leipzig.de
%%% 
%%%****************************************************************************

\section{\ProTop{} Filters}

The first phase of the processing of an input file which is performed by
\ProTop{} consists of the application of a series of input filters. Those
filters take the ASCII representation and transform it. The result is given to
the next filter in the chain.

The filters which are used are set with the option |input_filter| (see
page~\pageref{opt:input_filter}). 






The details of writing your own filters can be found in
appendix~\ref{chap:writing.filters}. 

The following sections describe (some of) the filters provided with \ProTop.


\subsection{The Filter {\tt none}}

The filter {\tt none} is a very simple filter which does simply nothing. It is
mainly provided as a basis for your own development of a filter.


\subsection{The Filter {\tt tee}}

The filter {\tt tee} provides a means to trace or log the data flow in a
filter chain.  If this filter is part of a chain then it acts like the empty
filter.  I.e. the input is passed through. As a side effect it stores
everything passed through also in the file specified by the option
|tee:file|\index{tee:file}.

The value of the option |tee:file| is an atom or string denoting a file name.
Some atoms have a special meaning. |output| and |stdout| denote the output
stream.  |stderr| denotes the standart error stream. If such an reeserved
value is encountered then the associated stream is used instead of opening a
file.  If a file is used then the output is appended to this file.

The option |tee:file| defaults to the value |output|.

Instead of specifying the output file in the option |tee:file| it is possible
to give the name as additional argument to the filter. Thus it is possible to
redirect the output of several incarnations of tee to different files:

The following example logs the input and the output of the filter
|mult_taut_filter| in the two files |file1| and |file2| respectively.

\begin{BoxedSample}
  input\_filter = [tee(file1),mult\_taut\_filter,tee(file2)]
\end{BoxedSample}


\subsection{The Filter {\tt tptp}}

The filter {\tt tptp} provides an interface to the TPTP library (Thousand of
problems for theorem provers). It is written such that files in the TPTP
syntax can be read in. It follows the conventions of the TPTP library. See the
documentation of the TPTP library for details.

This filter is especially useful when you want to experiment with the problems
from the TPTP library. For this purpose this filter should be the first one.

This filter uses the following option:
\begin{description}
\item[tptp:home] This option contains the complete path to the installed TPTP
  library. Usually this options should be set at installation time properly.
\end{description}


\subsection{The Filter {\tt mult\_taut\_filter}}

The filter {\tt mult\_taut\_filter} perform the reductions mult and taut on
the input problem. The filter understands the input syntax of the parser. Thus
it is recommended to use this filter as one of the last ones.

The mult reduction deletes multiple occurences of literals in a clause and
leaves only one instance of such a literal. This filter only removes identical
literals. Unification is not involved. With this restriction this
transormation is equivalence preserving.

The taut reduction deletes clauses which contain a literal as well as its
negated form. Only identical literals --- up to the complementary sign --- are
considered. No unification is involved. With this restriction the
transformation is equivalence preserving.


\subsection{The Filter {\tt mpp}}

The filter {\tt mpp} provides a {\em M}\/acro {\em P}\/re{\em P}\/rocessor
similar to the C preprocessor |cpp|. It is possible to define macros which are
expanded in the remainder of the file. Inclusion of other files and
conditional parts can also be specified. Last but not least it is possible to
define additional operators for syntactic sugar.

The filter {\tt mpp} provides additional meta instructions. Those meta
instructions are evaluated by this filter and noit passed through. The meta
instructions start with the symbol |#|.\footnote{This is defined as a prefix
  operator in \ProTop.} The following meta instructions are evaluated by {\tt
  mpp}: 

\begin{description}
\item [\#include {\em File}.]\ \\
  The contents of file {\em File} is included instead of this instruction as
  if it was there already. Nevertheless only complete clauses can be contained
  in this file. It is not possible to include parts of clauses.

  The included files are searched with the same algorithm as files specified
  as input files. I.e. if they are not absolute then they are searched in the
  directories given in the option
  |input_path|\index{input\_path}. Additionally the extensions stored in the
  option |input_extensions|\index{input\_extensions} are taken into account.

  The following example includes the file {\sf common}:

\begin{BoxedSample}
  \#include "common".
\end{BoxedSample}

\item [\#op({\em Precedence}, {\em Associativity}, {\em Name}\/).]\ \\
  For the remainder of this file the operator {\em Name} with precedence {\em
    Precedence} and associativity {\em Associativity} is defined. See the
  Prolog documentation on |op/3| for details.

  {\bf Note:} This operator is not automatically defined in the next filter!

  It might be a good practice to define a translation rule for any
  operator. Thus it can be translated into a ``normal'' operator not written
  in infix, postfix, or prefix notation. This helps avoiding confusion in the
  following parts of the system.

  The following example defines an operator |==>| which is used
  afterwards. Without the declaration this example would lead to an syntax
  error.

\begin{BoxedSample}
  \#op(900,xfx,(==>)).
  p(X) ==> q(X,f(X)).
  q(11,32).
\end{BoxedSample}

\item [\#define {\em Head} = {\em Tail}.]\ \\
  The macro definition specified in this instruction is stored. This does not
  overwrite previous definitions but is added. In the macro expansion the
  first unifying definition is applied.  The usual Prolog rules for variables
  are used.

  The definitions are applied at every level. This means that they are used at
  the formula level as well as on the term level.

\begin{BoxedSample}
  \#define a = new\_a.
  \#define p(X,new\_a) = pa(X).

  q(A).
  p(X,a).
  a.
\end{BoxedSample}
  The result is
\begin{BoxedSample}
  q(\_g123).
  p(\_g123,new\_a).
  new\_a.
\end{BoxedSample}

  To complete our example started in section on operator declaration we can
  define a translation rule (macro) to translate |==>| terms into the normal
  implication form supported by \ProTop.

\begin{BoxedSample}
  \#op(900,xfx,(==>)).
  \#define (A ==> B) = (B :- A).
  p(X) ==> q(X,f(X)).
\end{BoxedSample}

\item [\#define {\em Head}.]\ \\
  This is the same as defining {\em Head}\/ to 1. This is especially useful in
  combination with conditional inclusion of clauses (see below).

\begin{BoxedSample}
  \#define use\_equality\_axioms
\end{BoxedSample}

\item [\#undef {\em Pattern}.]\ \\
  Since macros are added but not overwritten we need a way to get rid of a
  macro. The undef instruction removes all macros for which the head is
  unifiable with {\em Pattern}. If no matching macro is found nothing is done.

  In the following example two macros are defined and the first one is deleted
  afterwards. 
\begin{BoxedSample}
  \#define mac(1,X) m1(X)
  \#define mac(2,X) m2(X)
  \#undef mac(1,\_)
\end{BoxedSample}
  The following instruction would have deleted both macros:
\begin{BoxedSample}
  \#undef mac(\_,\_)
\end{BoxedSample}

\item [\#if {\em Condition}.]\ \\
  {\em Condition}\/ is evaluated. If it evaluates to true then the block to
  the next matching |#else| or |#endif| is used. The |#else| to |#endif| block
  is ignored. Otherwise the block between the next matching |#else| and the
  |#endif| is used.

\begin{BoxedSample}
  \#if defined(use\_equality\_axioms).
  \#include equality\_axioms.
  \#endif.
\end{BoxedSample}

  They following conditionals are recognized by the |#if| instruction:

  \begin{description}
  \item [defined({\em Pattern}\/)]\ \\
    This condition evaluates to true iff a macro is defined for which the head
    is unifiable with {\em Pattern}.

  \item [not({\em Condition})]\ \\
    This condition evaluates to true iff {\em Condition} does not evaluate to
    true.
  \item [{\em Condition},{\em Condition}]\ \\
    A conjunction evaluates to true iff all conjuncts evaluate to true.

  \item [{\em Condition};{\em Condition}]\ \\
    A disjunction evaluates not to true iff at least one disjunct evaluates
    not to true.
  \end{description}

\item [\#else.]\ \\
  Switch to the alternate block started with a previously matching |#if|,
  |#ifdef|, or |#ifndef|.

\item [\#endif.]\ \\
  This instruction ends a block started with |#if|, |#ifdef|, or |#ifndef|.

\item [\#ifdef {\em Spec}.]\ \\
  This is the same as |#if defined(|{\em Spec}\/|)|.
  
\item [\#ifndef {\em Spec}.]\ \\
  This is the same as |#if not(defined(|{\em Spec}\/|))|.
\end{description}


The filter {\tt mpp} provides the hook |mpp_hook/1| which is called just
before the first clause is read. This hook can be used to add definitions or
include files at startup under the control of a Prolog program. For this
purpose several Prolog procedures are exported. See the implementation
description of {\tt mpp} for details.


\subsection{The Filter {\tt equality\_axioms}}

The filter {\tt equality\_axioms} adds the axioms and axiom schemata needed to
handle the equality. This means that an axiomatization of equality is added
and the proof is carried out without any special knowledge of equality.

The predicate |=/2| is assumed to be the equality predicate. The generated
clauses use this predicate.



\subsection{The Filter {\tt E-flatten}}

The filter {\tt E-flatten} provides an alternative treatment of equality
predicates.


\subsection{The Filter {\tt constraints}}

The filter {\tt constraints} provides a means to divide the matrix into a
matrix with constraints and a Horn-(Prolog) program. An enhanced syntax is
used to specify those parts and their interaction.




\chapter{Using \ProCom}

%%%****************************************************************************
%%% $Id: options.tex,v 1.9 1995/07/03 11:35:12 gerd Exp gerd $
%%%============================================================================
%%% 
%%% This file is part of ProCom.
%%% It is distributed under the GNU General Public License.
%%% See the file COPYING for details.
%%% 
%%% (c) Copyright 1995 Gerd Neugebauer
%%% 
%%% Net: gerd@imn.th-leipzig.de
%%% 
%%%****************************************************************************
% Master File: manual.tex

%------------------------------------------------------------------------------
\section{Options of \ProCom}\label{sec:options}

\ProCom{} can be adapted in it's behavior using a facility called options. A
option corresponds to a register or variable in procedural programming
language. Options of various kinds are used in \ProCom.

In general options can take any Prolog terms as values. Nevertheless there are
usually some kind of restrictions allowing only some kinds of values.

The most common options are boolean options. Boolean options can take the
values |on| and |off| only. In fact anything which is not |on| is interpreted
as |off|.

Another important type of option can take a file name. This is a Prolog string
pointing to a file name.

Options can be set in a options file which read at the beginning of any run of
\ProCom. Usually this file is called {\sf .procom} and is searched in the
current directory and the home directory in this order.

The \ProCom{} options file can contain instructions set options only.
This can be done using instructions of the following form:

|  |{\em Option}| = |{\em Value}|.|

In this instruction {\em Option}\/ denotes the name of an option and {\em
Value}\/ is it's value. Note that {\em Value} has to conform t Prolog
conventions and the terminating point is not optional! Possible options are
described below.


\subsection{General Options}

\begin{description}
  \item [prover] | = procom(extension_procedure)|\label{opt:prover}\\
	This option indicates the prover which has to be activated. The
	framework of \ProCom{} allows arbitrary provers to be integrated. For
	our purpose we describe only the interface to the provers provided
	with \ProCom{} and the interface to user defined provers.
	
	In this special case the value of this option is a Prolog term
	of the form |procom(|{\em Name}|)| where {\em Name}\/ is the
	name of a prover specified in the file {\sf procom\_config.pl}
	(see page \pageref{procom.config}).

  \item [search] | = iterative_deepening(1,1,1)|\\
	This option indicates which search strategy should be used.
	\begin{description}
	  \item [depth\_first]\ \\
		This search strategy uses the unbound depth first search like
		Prolog provides it.
	  \item [iterative\_deepening($\delta_0$, $\alpha$, $\beta$)]\ \\
		This search strategy successively increments a depth bound.
		The search space is searched to this limit. The limit is
		incremented upon failure.
		\\
		The initial depth $d_0$\/ is $\delta_0$.
		The next depth $d_{n+1}$\/ is computed from the previous depth
		$d_n$\/ according to the formula
		\[ d_{n+1} = \alpha\cdot d_n + \beta
		\]
		If $\alpha$\/ is $1$\/ then a linear function is achieved. If
		$\alpha$\/ is greater then $1$\/ then an exponential function
		can be forced.
	  \item [iterative\_deepening($\delta_0$, $\beta$)]\ \\
	        This is the same as 
		iterative\_deepening($\delta_0$, 1, $\beta$).
	  \item [iterative\_deepening($\delta_0$)]\ \\
	        This is the same as iterative\_deepening($\delta_0$, 1, 1).
	  \item [iterative\_deepening]\ \\
	        This is the same as iterative\_deepening(1, 1, 1).
	  \item [iterative\_inferences($\delta_0$, $\alpha$, $\beta$)]
	        This search strategy successively increments a inference bound.
		The search space is searched to this limit. The limit is
		incremented upon failure.
		\\
		The initial depth $d_0$\/ is $\delta_0$.
		The next depth $d_{n+1}$\/ is computed from the previous depth
		$d_n$\/ according to the formula
		\[ d_{n+1} = \alpha\cdot d_n + \beta
		\]
		If $\alpha$\/ is $1$\/ then a linear function is achieved. If
		$\alpha$\/ is greater then $1$\/ then an exponential function
		can be forced.
	  \item [iterative\_inferences($\delta_0$, $\beta$)]\ \\
	        This is the same as 
		iterative\_inferences($\delta_0$, 1, $\beta$).
	  \item [iterative\_inferences($\delta_0$)]\ \\
	        This is the same as iterative\_inferences($\delta_0$, 1, 1).
	  \item [iterative\_inferences]\ \\
	        This is the same as iterative\_inferences(1, 1, 1).
	  \item [iterative\_widening(a,b,c)]\ \\
		This is an experimental mode implementing iterative widening
		according to \cite{ginsberg.harvey:iterative}.

		{\bf Not ready yet!}
	  \item [iterative\_broadening(a,b,c)]\ \\
		This is an experimental mode implementing iterative
		broadening. 
	\end{description}

  \item [prolog] | = eclipse|\label{opt:prolog}\\
	This option indicates the target language which should be
	generated by \ProCom. Currently only a few possibilities are
	supported. The value |eclipse| indicates that \eclipse{} is
	the target language.  |quintus| stands for Quintus
	Prolog. Finally, |default| generates code in standart prolog,
	i.e.\ without special features of any Prolog dialect. In this
	mode several other options may have no effect and the
	resulting code may not be as efficient as in the specific
	modes.

  \item [input\_path] | = ['Samples']|\\
	This option contains a list of directories which are used to find an
	input file. This search path is used {\em after} the given file name
	has been tried as is. Thus the current directory (|.|) need not to be
	on this path.

  \item [verbose] | = on|\\
	This option turns on general verbosity of actions. Several modules may
	decide to use their own verbosity options.

  \item [toplevel:verbose] | = on|\\
	This option controlls the verbosity of the top level.


  \item [input\_filter] | = ''|\label{opt:input_filter}\\
	Filter to be used before the clauses are read. E.g. this filter can
	perform a normal form transformation. Filters may to be defined in
	{\sf config.pl} to be accessible. Otherwise they are dynamically
        loaded. In this case the file containing the filter has to be found on
        the search path for Prolog files.

	The value of this option can also be a list of filters. In this case
	each filter is used in turn. The output of the preceeding filter is
	given as input to the current filter. The initial filter gets the
	input file as input. The result is the result of the last filter.

        If only one filter is specified then the list can be omitted. I.e. a
        symbol or string value of this option is interpreted as a filter to be
        used.

	Non-existing filters are ignored.

	Let us consider an (imaginary) example:

\begin{BoxedSample}
  input\_filter = [nf\_transform,reductions1,reductions2]
\end{BoxedSample}

\end{description}

\subsection{Options Controlling the Prove Phases}

The prover can roughly be divided into two phases. The first phase is called
the preprocessing phase. In this phase several redution techniques can be
applied to reduce the size or complexity of the problem. The second phase is
the theorem prover itself. According to this overall distinction there are
some options controlling the general behaviour.

\begin{description}
\item [prove:red\_goals] |= [complete_goals,connection_graph]|\\
  This option is a list of modules which are invoked in turn to perform their
  tasks during the preprocessing phase. The elements of this list are
  reduction modules. Those modules are either preloaded or they are
  dynamically loaded.
  
\item [prove:red\_path] |= ['.']|\\
  This option contains the path for the dynamic loading of reduction
  modules. During the installation this path is augmented by the directory
  containing \ProTop{} and its files.

\item [prove:path] |= ['.']|\\
  This option contains the path for the dynamic loading of prover modules.
  The prover to be used is determined by the option |prover|.

  During the installation this path is augmented by the directory containing
  \ProTop{} and its files.

\item [prove:extension] |= ['','.pl']|\\
  This option contains a list of extensions used to find a reduction or a
  prover module. The strings or symbols contained in this list are appended to
  the file name before the existence of such a file is checked.

\item [prove:log\_items] |= [prover,matrix,contrapositives]|\\
  This option determines which information is written to the log file just
  after the first phase of the proving, i.e. the preprocessing. The value
  consists of a list of keywords as described below.

  \begin{description}
  \item[contrapositives]\ \\
    The list of contrapositives is written to the log file. The
    contrapositives are enclosed in |begin(contrapositives)| and
    |end(contrapositives)|. This environment contains log terms of the
    type |contrapositive/3|. The first argument is the head of the
    contrapositive. The second argument is the body, i.e. the list of literals
    without the head. The third argument contains the index of the
    contrapositive. 

    It is highly recommended to leave this item in the list |prove:log_items|
    since some postprocessing programs rely on it.
  \item[date]\ \\
    The current date is written to the log file. The date is stored in the log
    term |date/1|. The argument is a string containing the date in the format
    as given by the command |date/1|.
  \item[host]\ \\
    The host information is written to the log file. The host information is
    stored in the log term |host_info/2|. The first argument contains the
    hostname. The second argument contains the host architecture.
  \item[matrix]\ \\
    The list of clauses is written to the log file. The clauses are enclosed
    in |begin(matrix)| and |end(matrix)|. This environment contains log terms
    of the following types:
    \begin{description}
    \item[GoalClause/1] contains the indices of the goal clauses.
    \item[Label/2] contains the labels of the clauses. The first argument is a
      clause index and the second argument is its label.
    \item[Clause/2] contains the  clauses. The first argument is the index of
      this clause. The second argument is the list of literals. Each literal
      is of the form |literal(|$L$|,|$I$|)| where $L$\/ is a signed predicate
      and $I$\/ is its index.
    \end{description}

    It is highly recommended to leave this item in the list |prove:log_items|
    since some postprocessing programs rely on it.
  \item[options]\ \\
    The complete list of options and their values is written to the log file.
  \item[prover]\ \\
    The prover to be used is written to the log file. This information could
    also be extracted from the options, but sometimes this is the only option
    we are interested in.

    The prover name is stored in the log term |prover/1|.
  \item[user]\ \\
    The name of the user is written to the log file. The user information is
    stored in the log term |user/2|. The first argument is the login name of
    the user and the second name is the full name of the user as given in the
    GCOS field of the passwd file.
  \end{description}

\end{description}

\subsubsection{Options of the Reduction Module {\tt complete\_goals}}

\begin{description}

  \item [complete\_goals] | = on|\\
  If this option is turned on then the matrix is checked to contain a
  complete set of goals. If the set of goals is not complete then it is
  completed.

\end{description}

\subsubsection{Options of the Reduction Module {\tt connection\_graph}}

\begin{description}

  \item [find\_connections] | = on|\\
  This option can be used to suppress the generation of the reachability
  graph. \ProCom{} does not assume that this option is turned off. The
  result is unpredictable in this case.

  \item [find\_all\_connections] | = off|\\
  If this option is on then all literal are initially considered when
  constructing the reachability graph. Otherwise only literals in goal
  clauses are considered.

  \item [connect\_weak\_unifiable] | = on|\\
  One of two methods to compute the reachability graph can be selected.  The
  first method is to connect complementary literal which are weak
  unifyalbe. The alternative is to consider the predicate symbol only.

  \item [remove\_unreached\_clauses] | = on|\\
  If this option is on then each unreached clause is removed after the
  reachability graph has been constructed.

\end{description}


\subsection{Automatic Options}

This section contains options which are automatically set. They can be checked
but should not be altered in any way.

\begin{description}
  \item [equality] | = off|\\
  This boolean option is set when an equality predicate (|=/2|) is detected
  in the input problem.

  \item [setvar] | = off|\\
  This boolean option is set when an set variable (|:/2|) is detected in the
  input problem.

  \item [input\_file] | = ''|\\
  This option is automatically set to the current input file name as given
  by the user, i.e. without the automatically appended directory from the
  search path.

  \item [output\_file] | = ''|\\
  This option is set to the output file name as given by the user.

  \item [interactive] | = off|\\
  This option is set when the output is not redirected to a file. Thus an
  interactive prover can act accordingly.

\end{description}



\subsection{\ProCom\ General Options}

\begin{description}

  \item ['ProCom:dynamic\_reordering'] | = on|

  \item ['ProCom:ancestor\_pruning'] | = on|
    \\
    This option enables generation of code performing identical ancestor
    pruning. This means that a branch of the search tree is cut if two
    identical subgoals have been encountered. Only a minor variant may be
    implemented.

  \item ['ProCom::ancestor\_pruning'] | = 'prune.pl'|
    \\
    This option specifies the library which provides the predicates to perform
    the identical ancestor check.

  \item ['ProCom:occurs\_check'] | = on|
    \\
    This option indicates wheter special care should be taken to perform sound
    unification. In this case special caution is taken at the clause level.
    Another variant might be to enable the occurs check globally --- if this
    feature is provided by the Prolog dialect used.

  \item ['ProCom::unify'] | = 'unify.pl'|
    \\
    This option specifies which library should be used to get the predicate
    |unify/2| which performs sound unification. This library is used if the
    option  |ProCom:occurs_check| is |on|.

  \item ['ProCom::unify\_simple'] | = 'unify-no-oc.pl'|
    \\
    This option specifies which library should be used to get the predicate
    |unify/2| which performs unification. This library is used if the option
    |ProCom:occurs_check| is |off|.

  \item ['ProCom::member'] | = 'member.pl'|
    \\
    This option specifies which library should be used to get the predicate
    |member/2| which is a sound member predicate. Ususally this library will
    in turn use the library predicate |unify/2|. Thus the value of the option
    |ProCom:occurs_check| is taken into account.

  \item ['ProCom:literal\_selection'] | = deterministic|
    \\
    This option determines the strategy for literal selection. It can take the
    values |deterministic| or |random|\footnote{Currently not supported.}.

  \item ['ProCom:path'] | = path.pl|\label{opt:ProCom:path}
    \\
    This option determines the library containing the path managment
    predicates.  See section \ref{sec:lib.path} for details.

  \item ['ProCom::lemma'] | = 'no-lemma.pl'|
    \\
    This option specifies the library containing the lemma handling
    predicates.  The standard libraries contain also the library |lemma.pl|
    which is recommended as a basis for your own lemma handling routines.

  \item ['ProCom:lemma'] | = off|
    \\
    This option can be used to suppress the compiling of the calles to
    |lemma/3| which are defined in the lemma library.

  \item ['ProCom:automatic\_put\_on\_path] | = on |
    \\
    This option can be used to suppress the automatic addition of a goal to
    the current path. If youturn this option off you might want to use the
    primitive |put_on_path| in your descriptor to put some liteals on the
    path.

  \item ['ProCom:add\_contrapositives'] | = off|
    \\
    This option enables the generations of contrapositives in the target
    program. In this case the facts |contrapositive/3| are defined.

    The contrapositives may be needed by heuristic functions, e.g. for literal
    selection, which want to inspect the whole problem.

  \item ['ProCom:proof\_limit'] | = off|
    \\
  This option can be used to specify the way to proceed after a proof has been
  found and presented. The following values are currently supported:
  \begin{description}
    \item [interactive]\ 
      \\
      This value forces the generation of code to query the user. This is
      similar to the top level loop of Prolog. The library |ProCom::more| (see
      section \ref{lib:more}) is used in this case.
    \item [all]\ 
      \\
      This value forces the search for further solutions.
    \item [{\em number}]\ 
      \\
      Any natural number forces the search for further solutions until this
      number of solutions are found or all solutions are exhausted. The
      library specified by the option |ProCom::proof_limit| (see section
      \ref{lib:proof_limit}) is used in this case.
  \end{description}

  Any other value will force the termination after the first solution.

\end{description}

\subsection{\ProCom\ Goal Compiler Options}


\begin{description}
  \item ['ProCom:post\_goal\_list'] | = []|
    \\
    This option specifies a list of Prolog predictes which are called after
    the proof of a goal has been successful. As additional arguments the list
    of variable bindings and the proof tree is added to the predicate given.

    Consider the |ProCom:post_goal_list| containing the element |pgl(1,47)|.
    Then the goal |pgl(1,47,Vars,Proof)| is compiled into the prover, where
    |Vars| and |Proof| are instantiated to the variable bindings and the proof
    tree respectively.

    Note that you have to ensure that the predicates given are present in the
    compiled Prolog program. This can be done by specifying appropriate
    libraries containing the Prolog code.

  \item ['ProCom:init\_goal\_list'] | = []|
    \\
    This option specifies a list of Prolog predicates which are called when a
    goal is launched.

    Note that you have to ensure that the predicates given are present in the
    compiled Prolog program. This can be done by specifying appropriate
    libraries containing the Prolog code.

  \item ['ProCom:init\_level\_list'] | = []|
    \\
    This predicate specifies a list of predicates which are called whenever a
    new depth level is started, i.e. right after |set_depth_bound/1| has
    returned a new depth. The depth is added as additional argument to the
    predicates given.

    Note that you have to ensure that the predicates given are present in the
    compiled Prolog program. This can be done by specifying appropriate
    libraries containing the Prolog code.

\end{description}

See also the section on reordering (\ref{procom:reordering}) for further
options to influence the goal compilation.



\subsection{\ProCom\ Linker and Optimizer Options}

The options described in this section influence the behaviour of the linker.
The linker is the last step in the program generation process. Several tasks
are performed here. The main task is to add missing predicates from the
libraries. For this purpose possible libraries have to be named.

Another task perform in this last step is the application of certain
optimizations. The optimizations are mainly unfolding for Prolog predicates
for efficiency. These optimizations drastically reduce the readability of the
generated code. Thus it is recommened to turn them of when first trying to
understand the generated Prolog program.

\begin{description}

\item ['ProCom:link'] | = static|\\ 	This option specifies the type of
  linking. Possible values are
  \begin{description}
  \item [off]\ 
    \\
    This value disbales the linker. This is recommened for development
    purposes only. In this case the generated code is not executable as is.
  \item [static]\ 
    \\
    This is the usual way of linking. Any Prolog code from the libraries is
    copied into the generated Prolog program. Thus the program can be run on
    any appropriate Prolog system.
  \item [dynamic]\ 
    \\
    If this variant is used then for most of the libraries only compile
    instructions are generated in the target program. Thus this program is
    only executable if the used libraries are still accessible under the names
    given during linking.
  \end{description}

\item ['ProCom::link\_path']\label{opt:ProCom::link_path}\ 
  \\
  This option holds a list of directory names where the linker looks for
  appropriate files. The value of the option |prolog| is appended to those
  directories.

\item ['ProCom:optimize'] | = on|
  \\
  The linker has built-in the ability to perform certain optimizations.  For
  testing purposes it might be desirable to disable those optimizations.  In
  general it is not recommended to turn them off.

\item ['ProCom:expand'] | = on|\label{opt:ProCom:expand}
  \\
  This option enables the expansion (i.e. unfolding) of non-recursive
  predicates from the libraries. In the libraries one can specify which
  predicates should be considered (see section~\ref{sec:contents.library}).

\item ['ProCom:expand\_aux'] | = on|
  \\
  This option enables the expansion (i.e. unfolding) of auxiliary predicates.

\item ['ProCom:module'] | = off|
  \\
  This option enables the generation of code which encapsulates the prover in
  a module. This is only possible if a module system is provided by the Prolog
  dialect.

\item ['ProCom::module'] | = 'module.pl'|
  \\
  This option specifies the library which contains the module head.

\item ['ProCom::immediate\_link] | = ['init.pl']|%
  \label{opt:ProCom::immediate_link}\\
  This option specifies a list of libraries which are added to the code
  at the beginning --- possibly right after the module initialization but
  before anything else.

\item ['ProCom::post\_link] | = []|
  \\
  This option specifies a list of libraries which are added to the code at the
  end. This might be used to add libraries which are encapsulated in modules.

\item ['ProCom:ignore\_link\_errors'] | = off|
  \\
  This option can be used to suppress the linker to abort because of missing
  predicates. Usually the linker checks wether all predicates used are also
  defined. Those checks can fail if the predicate is defined in a library file
  or a module loaded with the |ProCom::post_link| option. Thus it can be
  desirable to ignore the linker errors. Nevertheless the error messages are
  displayed but ignored afterwards.
\end{description}

\subsection{\ProCom\ Information Controlling Options}

Information about the proof and the proof process can be of intrest. Thus it
is possible to enable the generation of such information. Since thiese
additional operations cost some time they may be worth turning off, when a
fast proof is required and the information is not essential.

\begin{description}

  \item ['ProCom:verbose'] | = off|
    \\
    This option enables the generation of various comments in the target
    Prolog program. The generated code my be slightly more readable when this
    option is on.

  \item ['ProCom:proof'] | = on|
    \\
    This option enables the generation of code to collect the information
    representing the proof tree. This takes some time and space and should be
    turned off when a fast prover is required. If a proof tree is enabled the
    predicates required are taken from the library given in the next option.

  \item ['ProCom::proof'] | = 'proof.pl'|
    \\
    This option specifies the library which is used for proof tree generation
    and display.

  \item ['ProCom:trace'	] | = off|
    \\
    This option enables code generation for the built-in debugger. This
    debugger can be used to trace the activities of the prover. (see
    \ref{sec:debugger})

  \item ['ProCom::trace'] | = 'debugger.pl'|
    \\
    This option specifies the library which contains the debugger.

  \item ['ProCom:timing'] | = on|
    \\
    This option enables the generation of code to determine the run time of
    the theorem prover. This is only possible in dialects which provide means
    to acess the run time.

  \item ['ProCom::timing'] | = 'time.pl'|
    \\
    This option specifies the library which contains code to set and query the
    timer.

  \item ['ProCom:show\_result'] | = on|
    \\
    This option enables the generation of code to display variable bindings
    and allow the user to ask for additional solutions.

  \item ['ProCom::show\_result'] | = 'show.pl'|
    \\
    This option specifies the library which contains code to display the goal
    and variable bindings after a succesful proof attempt.

\end{description}


\subsection{\ProCom\ Reordering}\label{procom:reordering}

\ProCom{} provides a powerful facility to reorder things before, during and
after the compilation process. This is an addition to the reordering hooks
provided in the preprocessing which are far less expressive.

In contrast to the |reorder_literals|/|clauses| functions which are assumed to
poke around in the Prolog database the \ProCom{} reordering facility provides
an interface which allows the user to specify an order and leave the
reordering to the system. This has the advantage that the user can do no harm
to the data, i.e. alter or delete entities.

The system performs the sorting using a user written comparison predicate. To
allow several sets of comparison predicates to coexist those predicates are
hidden in modules. The options specify the module name in which certain
predicates are expected. If no reordering is wanted then the empty symbol |''|
can be used instead.

The predicates given should succeed if the first element is less or equal than
the second element. These predicates should be deterministic.

\begin{description}
\iffalse
  \item ['ProCom:reorder\_ext']             | = ''|
    \\
    This option provides the name of a module containing a predicate
    |compare_ext/2|. This predicate is used to compare
  \item ['ProCom:reorder\_goal']            | = ''|
    \\
    This option provides the name of a module containing a predicate
    |compare_goal/2|. This predicate is used to compare
\fi
  \item ['ProCom:reorder\_clauses']         | = ''|
    \\
    This option provides the name of a module containing a predicate

    |compare_clauses/2|

    This predicate is used to compare clauses.
  \item ['ProCom:reorder\_goal\_clauses']   | = ''|
    \\
    This option provides the name of a module containing a predicate

    |compare_prolog_clauses/2|

    This predicate is used to compare
  \item ['ProCom:reorder\_prolog\_clauses'] | = ''|
    \\
    This option provides the name of a module containing a predicate

    |compare_prolog_clauses/2|

    This predicate is used to compare Prolog
    clauses of a procedure just after they have been generated and before the
    optimizer has done it's work. Depending on other options the clauses may
    have different forms. E.g. auxiliary predicates may be expanded.
  \item ['ProCom:reorder\_aux\_clauses']    | = ''|
    \\
    This option provides the name of a module containing a predicate

    |compare_aux_clauses/2|

    This predicate is used to compare auxiliary clauses.
\end{description}


%%%****************************************************************************
%%% $Id: capri.tex,v 1.5 1995/03/20 21:24:47 gerd Exp $
%%%============================================================================
%%% 
%%% This file is part of ProCom.
%%% It is distributed under the GNU General Public License.
%%% See the file COPYING for details.
%%% 
%%% (c) Copyright 1995 Gerd Neugebauer
%%% 
%%% Net: gerd@imn.th-leipzig.de
%%% 
%%%****************************************************************************
% Master File: manual.tex

\chapter{Programming \ProCom}

%------------------------------------------------------------------------------
%%%****************************************************************************
%%% $Id: libraries.tex,v 1.5 1995/02/13 20:05:14 gerd Exp $
%%%============================================================================
%%% 
%%% This file is part of ProCom.
%%% It is distributed under the GNU General Public License.
%%% See the file COPYING for details.
%%% 
%%% (c) Copyright 1995 Gerd Neugebauer
%%% 
%%% Net: gerd@intellektik.informatik.th-darmstadt.de
%%% 
%%%****************************************************************************
% Master File: manual.tex

\section{Libraries}

The next stage of modification beyond the adaption of option is to replace
libraries by own code. For this purpose we will describe in detail the
libraries used. Thus it is possible to write own libraries for new Prolog
dialects. Additionally it also enables you to replace the libraries provided
with \ProCom{} by your own versions which have an improved performance or an
enhance functionality.

\paragraph{Note:} Don't modify the libraries provided with \ProCom. Make
modified versions with a different name instead and set the appropriate option
to tell \ProCom\ to use this modified version. It is also possible to exploit
the search mechanism for libraries (see \ref{sec:lib.search}) to place the
modified version in a directory which is searched before the system library.

In the following sections we will describe the libraries and the predicates
they are expected to provide.


\subsection{Library Search}\label{sec:lib.search}
\index{library search}


At the end of the compilation process the linker tries to add libraries to the
code which provide required predicates. For this purpose the linker needs to
know which predicates are required and where to find libraries which might
provide them. Requirements of predicates are specified by the implementor, in
the Capri interpreter, in the Capri modules, or in the libraries.

When the linker is initialized it analyzes the files given to it to find out
which predicates are defined there. Usually the files known to the linker are
initially taken from several option (those containing |::|). The instruction
|link_file/1| in a Capri module can be used to provide additional files.

The link files are searched in the following way. The option
|ProCom:link_path| (see section \ref{opt:ProCom::link_path}) contains a list
of directories which are used to find the libraries. Two subdirectories of the
directories given are also considered. The first subdirectory is named like
the Prolog dialect specified by the option |prolog| (see section
\ref{opt:prolog}). The second subdirectory is called {\sf default}.

The general idea is to place generic code which should run on each Prolog
system in the subdirectoy {\sf default} and the dialect specific modules in
the specific subdirectories, e.g. {\sf eclipse}.

In addition to the intermediate directories the libraries are augmented by the
extension |.pl| during the search. To make it clear which files are considered
let us have a look at an example.

Suppose the option |prolog| has the value |eclipse| and the option
|ProCom:link_path| has the value |[.,ProCom]|. When the linker looks for a
library named |my_lib| the following locations are inspected until an existing
file is found:

{\tt
\begin{tabular}{l}
  ./my\_lib
  \\./my\_lib.pl
  \\./eclipse/my\_lib
  \\./eclipse/my\_lib.pl
  \\./default/my\_lib
  \\./default/my\_lib.pl
  \\ProCom/my\_lib
  \\ProCom/my\_lib.pl
  \\ProCom/eclipse/my\_lib
  \\ProCom/eclipse/my\_lib.pl
  \\ProCom/default/my\_lib
  \\ProCom/default/my\_lib.pl
\end{tabular}
}


\subsection{Contents of Library Files}\label{sec:contents.library}
\index{library file}

Library files are mainly usual Prolog files --- with some exceptions.

You should be very carefully when making a module in a library. Be sure that
you completely understand the translation process and can predict the
resulting Prolog code.

Since the library file is read by the linker each operator declaration must be
known to the linker. Operator declarations in the library are not evaluated.

Most of the instructions are simply passed to the output file. Nevertheless the
linker uses some instructions to control its behavior. The folloing list
describes instructions evaluated by the linker. Each such instruction is
embedded in a single Prolog goal. Accumulating several of them or embedding in
other constructs won't work.

In the following list the expression {\em Pred} always denotes a predicate
specification in the form {\em Functor/Arity}.
\begin{description}
  \item [:- expand\_predicate({\em Pred})]\index{expand\_predicate}\ \\
  This instruction is passed to the optimizer to declare the predicate {\em
  Pred} as expandable. If the option |ProCom:expand| (see section
  \ref{opt:ProCom:expand}) is turned on them expandable predicates are
  replaced by their bodies.

  You should only declare non-recursive, single-clause predicates as
  expandable\footnote{A future version of the linker may determine those
  predicates automatically.}.  Recursive predicates can not be expanded
  properly.  Don't try it. Multi-clause predicates may also cause problems.

  The declaration of expandable predicates must preceed their definition!

  \item [:- provide\_predicate({\em Pred})]\index{provide\_predicate}\ \\
  This instruction tells the linker that the predicate {\em Pred} is defined
  in this library. This should usually be not necessary since the linker
  analyzes the Prolog program to see which predicates are defined.
  Nevertheless it can be necessary to use it to declare a Prolog built-in
  which is required (see below).

  \item [:- require\_predicate({\em Pred})]\index{require\_predicate}\ \\
  This instruction tells the linker that the predicate {\em Pred} is used in
  this library but not defined. The linker makes not a complete analysis of
  used and defined predicates. Instead it evaluates this instruction to get
  the undefined predicates. This can be used to trigger the loading of other
  libraries.
\end{description}


\subsection{The Init Library}\label{sec:lib.init}

For some reasons it may be desirable to add some pieces of code in front of
anything else. This can be used to adjust the behavior of the Prolog system.
A list of libraries is taken from the option |ProCom::immediate_link| to be
linked at the beginning (see section \ref{opt:ProCom::immediate_link}).

The following sample implementations are provided with \ProCom.
\begin{description}
  \item [eclipse/init.pl]\ \\
  This library turns off some debugging features to speed things up.
  Additionally the sound unification is turned on.
  \item [default/init.pl]\ \\
  This library is simply empty. No initializations are needed for a vanilla
  Prolog.
\end{description}


\subsection{The Module Library}\label{sec:lib.module}

The resulting Prolog code can be encapsulated into a module. The head of a
module with a fixed name and export list can be placed in a library named in
the option |ProCom::module|.

The following sample implementation is provided with \ProCom.
\begin{description}
  \item [eclipse/module.pl]\ \\
  This library contains the head of a \eclipse\ module named |'ProCom prover'|
  which exports the predicates |goal/0| and |goal/1|.
\end{description}

\paragraph{Note:} Most of the time it is not desirable to generate a stand
alone module since \ProTop\ wrappes a module around the prover anyway.

\subsection{The Path Management}\label{sec:lib.path}

The data sturcture used for the path is completely encapsulated in a library.
The option |ProCom:path| (see section \ref{opt:ProCom:path}) determines the
library actually used. The following sample implementations are provided with
\ProCom.
\begin{description}
  \item [default/path.pl]
  \item [default/path-simple.pl]
  \item [eclipse/path-regular.pl]\ \\
  This library uses \eclipse\ delayed predicates to implement a variant of
  regularity constraints.
\end{description}

The current implementations use a pair of lists to implement the path. The
first item is the list of positive ancestors and the second one contains the
negative ancestors.  Since the path management is encapsulated in this module
one can replace this kind of path by another data structure. Linear lists are
a simpler case. But also balanced binary trees can be considered. Any
implementation which allows a fast access, e.g. by using sofisticated indexing
mechanisms, might be worth trying it.

The path management consists of the following set of predicates:
\begin{description}
  \item [empty\_path({\em Path})]\index{empty\_path}\ \\
  This predicate should unify its argument with the term representing the
  empty path. It should have at most one solution.

  \item [put\_on\_path({\em Literal}, {\em OldPath}, {\em
  NewPath})]\index{put\_on\_path}\ \\
  This predicate modifies the path {\em OldPath} in such a way that the
  literal {\em Literal} is in it. The result is unified with the new path {\em
  NewPath}.

  \item [put\_on\_path({\em Literal}, {\em Info}, {\em OldPath}, {\em NewPath})]\index{put\_on\_path}

  \item [is\_on\_path({\em Literal}, {\em Path})]\index{is\_on\_path}\ \\
  This predicate tries to unify {\em Literal} with an element on the path {\em
  Path}. Upon backtracking all such candidates are tried.

  \item [is\_on\_path({\em Literal}, {\em Info}, {\em
  Path})]\index{is\_on\_path}

  \item [is\_identical\_on\_path({\em Literal}, {\em
  Path})]\index{is\_identical\_on\_path}\ \\
  This predicate tries to find {\em Literal} on the path {\em Path}. Only
  identical (not only unifyable) occurences are considered. This predicate
  should not be resatisfiable.
\end{description}


\subsection{Sound Unification}

The following sample implementations are provided with \ProCom.
\begin{description}
  \item [eclipse/unify.pl]\ 
    \\
    This library provides a sound |unify/2| predicate.
  \item [default/unify-no-oc.pl]\ 
    \\
    This library provides a |unify/2| predicate not performing the occurs
    check.
\end{description}

The following predicate is provided by those modules:

\begin{description}
  \item [unify({\em Term1}, {\em Term2})]\index{unify}\ 
    \\
    This predicate unifies its arguments.
\end{description}



\subsection{Sound Member Implementation}\label{sec:member}

The following sample implementation is provided with \ProCom.
\begin{description}
  \item [default/member.pl]\ 
    \\
    This library uses the unify/2 predicate to perform unification. Thus it
    depends on the unify library which unification is actually used.
\end{description}

\begin{description}
  \item [sound\_member({\em Element}, {\em List})]\index{sound\_member}\ 
    \\
    This predicate implements the well known member relation. Modification may
    be necessary to use the occurs check when required.
\end{description}


\subsection{Lemmas}\label{sec:lib.lemma}

After the sucessful attempt to solve a subgoal the result can be stored as a
lemma. Thus there exists a lemma library where this feature can be hooked in.
The following sample implementations are provided with \ProCom.
\begin{description}
  \item [default/no-lemma.pl]\ \\
    This library performs no lemma steps. It is just a dummy library to be
    used if nothing else is desirable.
  \item [default/lemma.pl]\ \\
    This library uses the path to store local lemmata. Thus no special
    \CaPrI{} descriptor set is needed to apply the lemmas. They are applied in
    the course of normal reduction steps.
\end{description}

The lemma library has to provide the following predicate:
\begin{description}
  \item [lemma({\em Literal}, {\em PathIn}, {\em PathOut})]\index{lemma}\ \\
    {\em Literal}\/ is the internal representation of the negated 
    solved literal. {\em PathIn}\/ is the current path. It has to be returned
    in {\em PathOut}, possibly enhanced by some additional literals. E.g. the
    predicate |put_on_path/3| might be useful (cf.\
    section~\ref{sec:lib.path}).
\end{description}


\subsection{Literals}\label{sec:lib.literal}

The literals are packed into a predicate to allow manipulations of literals at
link time. Usually a literal is just left alone.
The following sample implementations are provided with \ProCom.
\begin{description}
  \item [default/literal.pl]\ \\
    This library performs nothing. It is just a dummy library to be
    used if nothing else is desirable.
  \item [eclipse/literal\_static.pl]\ \\
    This library restricts the number of solutions of propositional goals to
    one. The test for free variables is performed at compile time (link
    time). Thus it may not detect goals which are propositional after some
    variables have been bound at run time.
  \item [eclipse/literal\_dynamic.pl]\ \\
    This library restricts the number of solutions of propositional goals to
    one. The test for free variables is performed at run time. Thus it may
    cause some additional overhead.
\end{description}

The literal library has to provide the following predicate:
\begin{description}
  \item [literal\_wrapper({\em Literal}, {\em Varlist})]\index{literal\_wrapper}\ \\
    {\em Literal}\/ is the goal representation of the literal to be
    solved. {\em Varlist}\/ is the list of variables ocurring in the literal
    at compile time.
\end{description}


\subsection{Timing}\label{sec:timing}

The time library provides means to access and measure the time elapsed. Since
the plain Prolog does not seem to have any means to perform this task it is
highly dialect specific how this can be done.

The following sample implementation is provided with \ProCom.
\begin{description}
  \item [eclipse/time.pl]\ 
    \\
    This library provides the predicates to manipulate the time for the
    \eclipse{} system.
  \item [eclipse/time.pl]\ 
    \\
    This library provides the dummy predicates for timing. Since there is no
    standard for those predicates they are simply mapped to do nothing.
\end{description}

The following predicates are provided by the time library:

\begin{description}
  \item [set\_time]\index{set\_time}\ 
    \\
    This predicate is used to initialize the system timer.
  \item [reset\_print\_time(File)]\index{reset\_print\_time}\ 
    \\
    This predicate is used to determine the ammount of time elapsed since the
    last call to a timer routine. This time is printed to the standard output.
    If {\em File} is an atom and not |[]| then the time is also appended to
    this log file.
  \item [set\_timeout({\em Limit})]\index{set\_timeout}\ 
    \\
    This predicate arranges things that the execution is interrupted after
    {\em Limit} seconds. If the Prolog dialect has no such capability it
    simply succeeds.
  \item [reset\_timeout]\index{reset\_timeout}\ 
    \\
    This predicate stops the timeout timer or does nothing.
\end{description}


\subsection{Log Files and Proof Presentation}\label{lib:proof}

The following sample implementation is provided with \ProCom.
\begin{description}
  \item [eclipse/proof.pl]\ 
\end{description}

\begin{description}
  \item [save\_bindings({\em Index}, {\em Clause}, {\em Bindings}, {\em
  File})]\index{save\_bindings}\ \\
  This predicate writes the bindings to the log file {\em File}. If {\em
  File} is the empty list |[]| or not an atom then this predicate simply
  succeeds.

  \item [save\_proof({\em Proof}, {\em File})]\index{save\_proof}\ \\
  This predicate saves the proof term {\em Proof}\/ in the log file {\em
  File}. If {\em File}\/ is the empty list |[]| or not an atom then this
  predicate simply succeeds.

  \item [show\_proof({\em Proof})]\index{show\_proof}\ \\
  This predicate displays the proof term {\em Proof} on the standard output.
\end{description}


\paragraph{more.pl}\label{lib:more}

\begin{description}
  \item [more({\em Depth})]\ \\
  This predicate queries the user to continue the search for a next solution
  or to abort. {\em Depth}\/ contains the current depth limit or is an unbound
  variable if no depth limit is used. This predicate succeeds iff {\em no}\/
  more solutions are required.
\end{description}


\paragraph{show.pl}\label{lib:show}

\begin{description}
  \item [list\_bindings({\em GoalIndex},{\em GoalClause},{\em Bindings})]\ \\
    This predicate is called after a solution has been found. It is meant to
    present the solution.
  \item [no\_more\_solutions]\ \\
    This predicate is called when the search tree has been exhausted and no
    more solutions can be found.
  \item [no\_goal]\ \\
    This predicate is called when no goal clause is left.
\end{description}


\paragraph{proof\_limit.pl}\label{lib:proof_limit}

\begin{description}
  \item [init\_proof\_limit]\index{init\_proof\_limit}\ \\
  This predicate initializes the number of proofs already found to 0. This can
  be done by asserting a fact to the Prolog data base or any other method
  which allows the value to survive backtracking.

  \item [check\_proof\_limit({\em Limit})]\index{check\_proof\_limit}\ \\
  This predicate increments the number of proofs already found and compares it
  with {\em Limit}. This predicate succeeds iff the number of proofs found is
  greater or equal to the {\em Limit}.

  \item [get\_proof\_limit({\em Limit})]\index{get\_proof\_limit}\ \\
  This predicate unifies the number of proofs already found with {\em Limit}.
\end{description}




\subsection{The Search Libraries}\label{lib:search}


\begin{description}
  \item [set\_depth\_bound({\em Start}, {\em Factor}, {\em Const}, {\em
  Depth})]\index{set\_depth\_bound}\ \\
  This predicate 

  \item [check\_depth\_bound({\em Depth}, {\em NewDepth}, \_,
  \_)]\index{check\_depth\_bound}\ \\
  This predicate 

  \item [show\_depth({\em Depth})]\index{show\_depth}\ \\
  This predicate 

  \item [choose\_step({\em Step}, {\em Cand}, \_, \_)]\index{choose\_step}\ \\
  This predicate 
\end{description}


\subsection{The Debugger}\label{sec:debugger}






%------------------------------------------------------------------------------
\section{Implementing a Calculus}

To integrate a new calculus into \ProCom{} can be done dynamically or
statically. In the first case you have to write a \CaPrI{} description file
which is dynamically loaded if required. Those files are not taken into
account when an appropriate prover is selected.

Alternatiely you can integrate the new calculus into the precompiled \ProCom{}
executable. In this case it is considered as a possible prover to be used when
none is given or the given one is not appropriate.

To use a dynamic \CaPrI{} description file you have to
\begin{enumerate}
  \item Write a description file as described in the next sections.
  \item Put it on the search path such that dynamic loading can find it.
    The search path is given by the option |ProCom:capri_path|. To find the
    files the extensions from the options |ProCom:capri_extensions| are taken
    into account.
\end{enumerate}


To integrate a new calculus into the compiled \ProCom{} is done in three steps
\begin{enumerate}
  \item Write a description file as described in the next sections.
  \item Add a line in {\sf Makefile} as described in section
    \ref{sec:installation}.
  \item Recompile \ProCom{} as described in section~\ref{sec:recompile}.
\end{enumerate}



%------------------------------------------------------------------------------
\subsection{Configuring \ProCom}\label{procom.config}

The configuration can be performed at several levels. The most desirable place
is the Makefile (see section \ref{sec:installation}. Another file --- which is
automatically generated by the Makefile --- is described next.

The file {\sf ProCom/procom.cfg}\/ is a simple Prolog file which contains
facts for the Prolog predicate |define_prover/1|. Those facts describe which
modules should be loaded into the resulting \ProCom{} executable.

Several provers can be loaded at the same time. The selection for one
compilation is done with the option |prover| (see \pageref{opt:prover}).
Consider the following example of a file {\sf ProCom/procom.cfg}.

\begin{Sample}\index{define\_prover}
\begin{verbatim}
define_prover(extension_procedure).
define_prover(me_paramod).
define_prover(my_own_calculus).
\end{verbatim}
\end{Sample}

In this case three provers are defined. They can be activated with the
settings
\\ |prover = procom(extension_procedure)|,
\\ |prover = procom(me_para)|, or
\\ |prover = procom(my_own_calculus)| respectively.


The prover used is selected according to the following strategy:
\begin{itemize}
\item If the prover specified in the option |prover| is applicable then this
  prover is used. A prover is applicable if the options which are declared to
  be required are set appropriately (see page \pageref{require_option}).

\item If the specified prover is not applicable then each prover given in the
  configuration file {\sf ProCom/procom.cfg}\footnote{This file is
    created automatically by the Makefile.} is tried in turn until one is
  found which is applicable.
\end{itemize}



\subsection{The Framework}

If you want to create a new prover using the CaPrI interface you have to write
a new module. This module has to be compatible with the existing ones.
Basically three distinct pieces of information are involved in this process:
\begin{itemize}
  \item The name of the module.\\
	This name is given in the |module/1| declaration.
  \item The file name of the module.\\
	This file name is used when writing a new module.
  \item The predicate performing the compilation.\\
	This predicate is used implicitely and you don't have to care about
	it, except that you should not use such a predicate yourself.
\end{itemize}

Thus the starting point is a file named {\em my\_own\_calculus}|.pl| which
starts with the following code:
\smallskip

{\bf :- module({\em my\_own\_calculus}).\\
     :- compile(library(capri)).
}
\medskip

The first line starts a new module with the name given. The second directive
ensures that the CaPrI driver routines are included and a predicate {\em
my\_own\_calculus} is defined.

To continue you can consult the next sections to see which instructions can be
used further on.


\subsection{Describing Deduction Steps}

\begin{description}
  \item [descriptor {\em Code}.]\index{descriptor}\ \\
	{\em Code}\/ is a comma seperated list of goals of the form described
	below.
\end{description}

Consider the following example which implements the reduction step of the
extension procedure. This step says that a {\em goal}\/ |Pred| can be solved
by finding the negated literal |-Pred| on the {\em path}. Additionally we can
name this proof step as |reduction(Pred)|. This information is used to
construct a proof tree (if desired).

\begin{BoxedSample}
descriptor
        proof(reduction(Pred)),
        template(Pred,goal),
        template(-Pred,path).
\end{BoxedSample}



\begin{description}
  \item [name({\em Name})]\index{Capri!name}\ 
    \\
    This specification is meant to name a deduction step. It is optional to
    provide it. If none is given then the functor of |proof| (see below) or a
    default is used.

    It can be desirable to distinguish some deduction steps even so they
    generate the same node in the proof tree.

  \item [proof({\em Proof})]\index{Capri!proof}\ 
    \\
    This specification is used to generate the proof tree. The functor of {\em
      Proof}\/ may also be used as the name of a deduction step (see above).
    Thus {\em Proof}\/ is required to be a compound term.

  \item [template({\em Pattern})]\index{Capri!template}\ 
    \\
    This template is a short form to determine a goal pattern. All goal
    patterns leading to one instance of a deduction step are unified. Thus
    various aspects of a goal can be used.

    The following example describes any goal which has a positive sign.

    \begin{BoxedSample}
      templates(++Literal)%
    \end{BoxedSample}

  \item [template({\em Pattern}, goal)]\index{Capri!goal template}\ 
    \\
    This template is used to determine the goal literal to start with.
    Each descriptor must have at least one goal template. Otherwise no 	code
    is generated for it.

    The following example describes any goal which has a positive sign 	again,
    but this time a complete version (no abbreviation) is used.

    \begin{BoxedSample}
      templates(++Literal,goal)%
    \end{BoxedSample}

%  \item [template({\em Pattern}, goal({\em Index}))]

  \item [template({\em Pattern}, path)]\index{Capri!path template}\ 
    \\ 
    This template describes a subgoal which is solved by unifying it with a
    literal on the path.

    The following example can be understood as a part of the extension
    procedure. The negated literal |-Literal| is searched on the path.

    \begin{BoxedSample}
      templates(-Literal,path)%
    \end{BoxedSample}

  \item [template({\em Pattern}, path({\em Info}))]\index{Capri!path
      template}\ 
    \\
    This template describes a subgoal which is solved by unifying it with
    a literal on the path. The additional argument to |path| constitutes a
    way to get hold of information stored about the element on the path.
    This templete will lead to a compiletime/runtime error when the path
    library does not provide the appropriate predicates.

    \begin{BoxedSample}
      templates(-Literal,path(X))%
    \end{BoxedSample}

  \item [template({\em Pattern}, extension({\em
      LiteralIndex}))]\index{Capri!extension template}\ 
    \\
    This template describes a subgoal {\em Pattern}\/ which is solved by
    unifying it with a complementary literal somewhere in the matrix. The
    other subgoals of the clause containing the complemntary literal are left
    as residues. The argument {\em LiteralIndex}\/ is unified with the index
    of the complementary literal.

    \begin{BoxedSample}
      templates(--Literal,extension(C-L))%
    \end{BoxedSample}

  \item [template({\em Pattern}, extension)]\index{Capri!extension template}\ 
    \\
    This template acts like the pattern above, but not literal index is
    used/returned.

	
    \begin{BoxedSample}
      templates(--Literal,extension)%
    \end{BoxedSample}

  \item [template({\em Pattern}, residue)]\index{Capri!residue template}\ 
    \\
    A residue is directly translated into a call to the associated
    procedure. I.e. The goal will be solved with any means provided by the
    calculus in action.

    \begin{BoxedSample}
      templates((A=B),residue)%
    \end{BoxedSample}

  \item [template({\em Pattern}, neg\_residue)]\index{Capri!neg residue
      template}\ 
    \\
    A residue is directly translated into a call to the associated procedure
    after it has been negated. I.e. The goal will be solved with any means
    provided by the calculus in action.

    \begin{BoxedSample}
      templates((A=B),neg\_residue)%
    \end{BoxedSample}

  \item [template({\em Pattern}, {\em ListOfSelectors})]\index{Capri!list of
      templates}\ 
    \\
    This template describes a subgoal which is solved by applyig one of the
    selectors in {\em ListOfSelectors}. This list acts like a disjunction of
    possible selectors. {\em ListOfSelectors} is a list of simple selectors
    described above.

    The following example is used to write two disjoint descriptors --- which
    differ in this one template only --- in a single descriptor.

    \begin{BoxedSample}
      templates(Literal,[extension(E),path])%
    \end{BoxedSample}


  \item [template({\em PatternList}, {\em Selector})]

  \item [template({\em PatternList}, map({\em
      Selector,\ldots,Selector}))]\index{Capri!map template}\ 
    \\
    This template describes a list of patterns. Each pattern may use any of
    the selectors given in the map, but each element of the map is used
    exactly once. Thus |map| has the same number of arguments as {\em
      PatternList} has elements.

    \begin{BoxedSample}
      templates([L1,L2,L3],map([extension,path],goal,residue)%
    \end{BoxedSample}


  \item [some\_templates({\em Pattern},  
    {\em ListOfSelectors})]\index{Capri!some templates}\ 
    \\

    \begin{BoxedSample}
      some\_templates(Literal,[extension(E),path])%
    \end{BoxedSample}

  \item [some\_templates({\em Pattern},  
			 {\em ListOfSelectors},  
			 {\em Predicate})]\index{Capri!some templates}

  \item [some\_templates({\em Pattern}, 
    {\em ListOfSelectors},
    {\em Limit})]\index{Capri!some templates}\ 
    \\

    \begin{BoxedSample}
      some\_templates(Literal,[extension(E),path],23)%
    \end{BoxedSample}
    
    \begin{BoxedSample}
      some\_templates(Literal,[extension(E),path],1-7)%
    \end{BoxedSample}
    
    \begin{BoxedSample}
      some\_templates(Literal,[extension(E),path],at\_least(3))%
    \end{BoxedSample}


  \item [call({\em Prolog\_Code})]\index{Capri!call}\ 
    \\
    The given {\em Prolog\_Code}\/ is executed during the compilation.  This
    feature can be used to check certain conditions and omit the generation of
    code if those conditions do not hold by failing.  On the other side this
    instruction can be used to compute things not accessible in the current
    phase of implementation.
    
    \begin{BoxedSample}
      call(writeln('\%\%\% Checkpoint'))%
    \end{BoxedSample}

  \item [constructor({\em Prolog\_Code})]\index{Capri!constructor} \ 
    \\
    The given {\em Prolog\_Code}\/ is integrated into the target program.

    \begin{BoxedSample}
      constructor(writeln('Hello world.'))%
    \end{BoxedSample}


  \item [get(path, {\em Path})]\index{Capri!get path}\ 
    \\
    {\em Path}\/ is unified with the current path.

    This can be used to get the current path in order to restore it
    later. Such a situation is shown in the next example:

    \begin{BoxedSample}
      get(path,Path),
      template(Literal,extension),
      use\_path(Path),
      template(AnotherLiteral,residue)%
    \end{BoxedSample}
 

  \item [get(depth, {\em Depth})]\index{Capri!get depth}\ 
    \\
    {\em Depth}\/ is unified with the current depth.

    Note that the depth is not neccesarily a number. It may also be a Prolog
    term which is used by the search strategy to perform its task.

  \item [get(functor({\em Pred}), {\em Functor/Arity})]%
    \index{Capri!get functor}\ 
    \\
    This instruction unifies the literal {\em Pred}\/ with the literal
    having the functor {\em Functor}\/ and the arity {\em Arity}. This can be
    used to extract the functor and arity from a literal --- which has an
    additional sign. Alternatively it can be used to construct a literal. In
    this case the sign is |++|.

    The following example will unify |F/A| with |p/2|:
    \begin{BoxedSample}
      get(functor(++p(X,f(23)),F/A))%
    \end{BoxedSample}

    The following example will unify |L| with |++q(_,_,_)|:
    \begin{BoxedSample}
      get(functor(L,q/3))%
    \end{BoxedSample}

  \item [get(literals, {\em ListOfLiterals})]\index{Capri!get literals}\ 
    \\
    {\em ListOfLiterals}\/ is unified with the list of all literals occuring
    in the current matrix. The complete literals --- including sign and
    arguments --- are returned.


  \item [get(predicates, {\em ListOfPredicates})]\index{Capri!get predicates}\
    \\
    {\em ListOfPredicates} is unified with the list of (signed) functors. Each
    element has either the form |(-F)/A| for negative literals or |F/A| for
    positive literals, where |F| is a functor and |A| its arity.

    Consider a matrix containing only positive and negative occurrences of the
    predicate p/1. Then in the following example |PREDS| is unified with {\tt
      [(-p)/1,p/1]}.
    \begin{BoxedSample}
      get(predicates,PREDS)%
    \end{BoxedSample}

%  \item [get(functors, {\em ListOfFunctors})]\index{Capri!get functors}\ 
%    \\
%    {\em ListOfFunctors}

  \item [get(contrapositive({\em Index}), {\em
      Contrapositive})]\index{Capri!get contrapositive}\ \\
    {\em Contrapositive}\/ is unified with the contrapositive having the index
    {\em Index}. This can be used to get a contrapositive with a specified
    index or to get all contrapositives --- upon backtracking.

  \item [get(solved\_goals, {\em ListOfGoals})]\index{Capri!get solved goals}\
    \\
    {\em ListOfGoals}\/ is unified with the list of the goals already solved.

    {\bf NOT YET!}

  \item [get(open\_goals, {\em ListOfGoals})]\index{Capri!get open goals}\ 
    \\
    {\em ListOfGoals}\/ is unified with the list of open goals.

    {\bf NOT YET!}

  \item [put\_on\_path({\em Literal},{\em Info})]\index{Capri!put on path}\ 
    \\
    {\em Literal + Info }\/ is put on the path at this place. To prevent the
    automatic placement of a literal on the path the option
    |ProCom:automatic_put_on_path| can be used.

    \begin{BoxedSample}
      get(path,Path),
      get(depth,Depth),
      put\_on\_path(SomeLiteral,Depth),
      template(Literal,extension),
      use\_path(Path)%
    \end{BoxedSample}
 
  \item [put\_on\_path({\em Literal})]\index{Capri!put on path}\ 
    \\
    {\em Literal}\/ is put on the path at this place. To prevent the automatic
    placement of a literal on the path the option
    |ProCom:automatic_put_on_path| can be used. If no info is given (see
    above) then a new variable is used instead.

    \begin{BoxedSample}
      get(path,Path),
      put\_on\_path(SomeLiteral),
      template(Literal,extension),
      use\_path(Path)%
    \end{BoxedSample}
 
  \item [use\_path({\em Path})]\index{Capri!use path}\ 
    \\
    {\em Path}\/ is used as the path from this time on.  Note that you can not
    rely on any specific format of the path. The only save value for {\em
      Path} is the value of |get/path| or |empty_path|

    \begin{BoxedSample}
      get(path,Path),
      template(Literal,extension),
      use\_path(Path),
      template(AnotherLiteral,residue)%
    \end{BoxedSample}
 
\end{description}

To end this part we will show a complete \CaPrI{} descriptor module. It
implements the well known model elimination calculus. The first descriptor
encodes the reduction step and the second descriptor encodes the extension
step. In both cases the appropriate information for the proof tree is
provided. 

\begin{BoxedSample}
:- module(my\_own\_calculus).
:- compile(capri).

descriptor
        proof(reduction(Pred)),
        template(Pred,goal),
        template(-Pred,path).

descriptor
        proof(extension(-Pred)),
        template(Pred,goal),
        template(-Pred,extension).
\end{BoxedSample}

\iffalse
\subsection{Before and After Descriptors}

\begin{description}
  \item [start\_descriptor]\ \\
	This descriptor is evaluated once before ...

  \item [end\_descriptor]\ \\
	This descriptor is evaluated once after ...
\end{description}

\subsection{Static Reordering}

These features are not implemented yet. 

\begin{description}
  \item [compare\_literals(L1,L2).]\ \\
	Reordering literals

  \item [compare\_clauses(C1,C2).]\ \\
	Reordering clauses
\end{description}

\fi


\subsection{Testing and Modifying Options}

A certain prover may require some options to be set to given values or fit in
a given scheme. E.g. the target language has to be Quintus prolog since a
library is only written for this dialect.

To accomplish this instructions are provided to test and modify options. For a
detailed description which options are predefined see section
\ref{sec:options}.

\begin{description}
  \item [force\_option({\em Option}, {\em
  ListOfValues}).]\index{force\_option}\ \\
	This instruction describes which values of options are allowed. The
	elements os the list {\em ListOfValues} are unified with the actual
	value of {\em Option} until a match is found. 

	If no candidate unifies then the first element of the list is taken as
	the value of {\em Option}.

	This directive is evaluated before the compilation is started.
	I.e. the matrix has already been read and no preprocessing has been
	performed. 

	As an example consider the following instruction:

\begin{BoxedSample}
  force\_option(prolog, [quintus]).
\end{BoxedSample}

	This instruction forces the generation of Prolog code in the target
	language Quintus Prolog.

  \item [require\_option({\em Option}, {\em ListOfValues}).]%
	\label{require_option}\ \\
	This instruction describes requirements for options without the
	possibility to revert the option to a default value. 
	If one of these instructions is incompatible with the actual value of
	an option then no compilation is performed.

	For instance consider the following instruction which forces that the
	option |equality| is off.

\begin{BoxedSample}
  require\_option(equality, [off]).
\end{BoxedSample}

\end{description}


\subsection{Adjusting the Linker}

The linker tries to add definitions for missing Prolog predicates. For this
purpose it scans libraries and adds those libraries containing the appropriate
definitions. 

It can be desirable to replace given libraries, e.g. to use an improved
version of the debugger. This can partially be done by modifying options. This
section provides instructions to perform tasks which can hardly be solved by
modifying options.

\begin{description}
  \item [require\_predicate({\em Predicate}).]\index{require\_predicate}\ \\
	This instruction tells the linker that the predicate {\em Predicate}\/
	is required and a appropriate definition should be linked. {\em
	Predicate}\/ is a predicate specification of the form {\em
	Name/Arity}\/ like in the following example

\begin{BoxedSample}
  require(paramodulate/4).
\end{BoxedSample}

  \item [library\_file({\em File}).]\index{library\_file}\ \\
	This instruction tells the linker to consider the file {\em File} to
	find additional Prolog code.

\begin{BoxedSample}
  library\_file(my\_debugger).
\end{BoxedSample}

  \item [library\_path({\em ListOfDirectories}).]\index{library\_path}\ \\
	This instruction prepends the list of directories {\em
	ListOfDirectories} in front of the search path for libraries as in

\begin{BoxedSample}
  library\_path(['../my\_trash']).
\end{BoxedSample}

  \item [provide\_definition({\em Code}).]\index{provide\_definition}\ \\
	This instruction provides the linker with additional Prolog code which
	may be included. The {\em Code}\/ is a list of Prolog clauses which
	are added to the target program when required.

	This is only recommended for small pieces of Prolog. Consider the use
	of a library instead.
\end{description}



%%*****************************************************************************
%% $Id: otter.tex,v 0.00 1994/06/14 15:34:29 gerd Exp $
%%*****************************************************************************
%%% 
%%% This file is part of ProCom.
%%% It is distributed under the GNU General Public License.
%%% See the file COPYING for details.
%%% 
%%% (c) Copyright 1995 Gerd Neugebauer
%%% 
%%% Net: gerd@imn.th-leipzig.de
%%% 
%%%****************************************************************************

\chapter{Using Otter within \ProTop}


\section{Overview}

\ProTop{} is able to handle internal and external provers. The capability to
handle external provers is demonstrated by the integration of otter. Otter is
a prover written in C which has to be run as a separate process. Otter gets
its problem specification (the matrix) together with the flags (options) in
one single file. The result consists of informative messages, like runtime,
and the final proof or failure indicators. Those results are also collected in
a file.

\ProTop{} has to translate to internal representation of the matrix in a form
acceptable by otter. The external process has to be started and finally the
output has to be analyzed and translated into a form handeled by \ProTop.

To start otter the option |prover| has to be set to the value |otter|. Then
the proof can be started as usual. This is shown in the example in
figure~\ref{fig:otter} which assumes us to be in the \ProTop{} interactive
mode.

\begin{figure}[ht]
\begin{BoxedSample}
  ProTop -> |prover = otter.|
  ok.
  ProTop -> |prove 'eder1-2'.|
  \% No input filter requested
  \% Reading matrix file Samples/eder1-2
  2 clauses read in 17 ms 
  ...

  time("user CPU time       ",20).
  time("system CPU time     ",70).
  time("wall-clock time     ",0).
  time("hyper\_res time      ",0).
  time("for\_sub time        ",0).
  time("back\_sub time       ",0).
  time("conflict time       ",0).
  time("demod time          ",0).
  ok.
  ProTop -> 
\end{BoxedSample}
\caption{A session with otter.}\label{fig:otter}
\end{figure}



\section{Options for Otter}

The module otter provides the following options to adjust its behaviour:

\begin{description}
  \item [otter:flags] | = [auto,-print_given].|
    \\
    This option contains a list of flags for Otter. a detailed description can
    be found in section \ref{otter:flags}.

  \item [otter:executable] | = "otter".|
    \\
    This option contains the executable UNIX command which starts otter. If
    this executable can not be found on the search path of the UNIX shell used
    it might be neccesary to specify the full file name --- including the
    complete path.

    This option is initialized from the valued specified in the Makefile
    during the installation process. Usually it should not need modification.

  \item [otter:tmpdirs] | = [indir,".","/tmp","~"].|
    \\
    This option specifies a list of directories which might be used to create
    intermediate files. The list is tried from left to right to find a
    directory which is writable. The first such directory is used to create
    the intermediate files for otter.

    The elements of this list are interpreted as directories. They can be
    absolute or relative to the current directory. The tilde |~| can be used
    to refer to the home directory of the user. The special symbol |indir| can
    be used to refer to the directory containing the input file.

  \item [otter:remove\_tmp\_files] | = off.|
    \\
    At the end of the proof attempt the intermediate files created for the
    communication with otter are removed if this boolean option is
    |on|. Otherwise they are left unchanged. This can be useful to understand
    how otter is called and get more details what otter has responded.

\end{description}



\section{Flags of Otter}\label{otter:flags}

Otter can be configured with a large number of flags. Most of those
flags\footnote{Only those flags which might disturb \ProTop{} have been
  disabled.} are usable from within \ProTop. For a detailed discussion of the
flags we refer to \cite{mccune:otter}. The following description only
mentiones a few. The main emphasis is on the way how to use them from within
\ProTop.

Otter has two kind of flags: boolean flags and numeric flags. The
specifications for those flags is combined in the option |otter:flags| which
contains a list of those.

The names of the flags are symbols in the sense of Prolog. Thus those names
are used to name the flags in the context of \ProTop{} as well.

\subsection{Boolean Flags}

Boolean flags can be set (on) or reset (off). If a boolean flag {\em flag}
occurs in the list of flags |otter:flags| then this is interpreted as a
positive occurrence and the corresponding Otter flag is turned on.

E.g. if the list in |otter:flags| contains the element |auto| then the Otter
flag |auto| is set.\footnote{The flag |auto| turns on the analysis automatic
  setting of flags according to the properties of the problem given.}

A negated flag in the list |otter:flags| is marked with a preceeding minus
|-|. Flags marked in this way are turned off in Otter.

E.g. if the list in |otter:flags| contains the element |-print_given| then the
Otter flag |print_given| is reset.\footnote{The flag {\tt print\_given} turns
  on the reporting of input clauses in the output.}


\subsection{Numeric Flags}

Numeric flags can take numeric values. Arbitrary numbers can be assigned to
some of the numeric flags. Other can take only values in a specific
range. Those restrictions are checked in the \ProTop{} interface to Otter.

To assign a value {\em n}\/ to a numeric flag {\em flag}\/ simply add an
element {\em flag {\tt=} n} to the list of flags |otter:flags|.

E.g. if the list in |otter:flags| contains the element |max_kept=1000000| then
the numeric flag |max_kept| is set to the value 1000000.\footnote{The flag
  {\tt max\_kept} contains the maximum number of resolvents kept.}


\subsection{List of Flags}

\begin{figure}[t]
{\scriptsize
\catcode`|=\other
\def\OtterFlag#1#2.{\tt #1&#2\\\hline}%
\null\hfill
\begin{tabular}{|p{15em}c|}\hline \bf Flag & \bf Type \\\hline\hline
  \OtterFlag{atom\_wt\_max\_args	}{boolean}.
  \OtterFlag{auto			}{boolean}.
  \OtterFlag{back\_demod		}{boolean}.
  \OtterFlag{back\_sub			}{boolean}.
  \OtterFlag{binary\_res		}{boolean}.
  \OtterFlag{bird\_print		}{default}.
  \OtterFlag{check\_arity		}{boolean}.
  \OtterFlag{control\_memory		}{boolean}.
  \OtterFlag{delete\_identical\_nested\_skolem}{boolean}.
  \OtterFlag{demod\_history		}{boolean}.
  \OtterFlag{demod\_inf			}{boolean}.
  \OtterFlag{demod\_limit		}{integer}.
  \OtterFlag{demod\_linear		}{boolean}.
  \OtterFlag{demod\_out\_in		}{boolean}.
  \OtterFlag{detailed\_history		}{boolean}.
  \OtterFlag{display\_terms		}{default}.
  \OtterFlag{dynamic\_demod		}{boolean}.
  \OtterFlag{dynamic\_demod\_all	}{boolean}.
  \OtterFlag{dynamic\_demod\_lex\_dep	}{boolean}.
  \OtterFlag{echo\_included\_files	}{boolean}.
  \OtterFlag{eq\_units\_both\_ways	}{boolean}.
  \OtterFlag{factor			}{boolean}.
  \OtterFlag{for\_sub			}{boolean}.
  \OtterFlag{for\_sub\_fpa		}{boolean}.
  \OtterFlag{fpa\_literals		}{integer(0,100)}.
  \OtterFlag{fpa\_terms			}{integer(0,100)}.
  \OtterFlag{free\_all\_mem		}{boolean}.
  \OtterFlag{hyper\_res			}{boolean}.
  \OtterFlag{index\_for\_back\_demod 	}{boolean}.
  \OtterFlag{input\_sos\_first		}{boolean}.
  \OtterFlag{interactive\_given		}{default}.
  \OtterFlag{interrupt\_given		}{default}.
  \OtterFlag{knuth\_bendix		}{boolean}.
  \OtterFlag{lex\_order\_vars		}{boolean}.
  \OtterFlag{lrpo			}{boolean}.
  \OtterFlag{max\_distinct\_vars	}{integer}.
  \OtterFlag{max\_gen			}{integer}.
  \OtterFlag{max\_given			}{integer}.
  \OtterFlag{max\_kept			}{integer}.
  \OtterFlag{max\_literals		}{integer}.
  \OtterFlag{max\_mem			}{integer}.
  \OtterFlag{max\_proofs		}{integer}.
  \OtterFlag{max\_seconds		}{integer}.
  \OtterFlag{max\_weight		}{integer}.
  \OtterFlag{min\_bit\_width		}{integer}.
\end{tabular}\hfill
\begin{tabular}{|p{15em}c|}\hline \bf Flag & \bf Type \\\hline\hline
  \OtterFlag{neg\_hyper\_res		}{boolean}.
  \OtterFlag{neg\_weight		}{integer}.
  \OtterFlag{no\_fanl			}{boolean}.
  \OtterFlag{no\_fapl			}{boolean}.
  \OtterFlag{order\_eq			}{boolean}.
  \OtterFlag{order\_history		}{boolean}.
  \OtterFlag{order\_hyper		}{boolean}.
  \OtterFlag{para\_all			}{boolean}.
  \OtterFlag{para\_from			}{boolean}.
  \OtterFlag{para\_from\_left		}{boolean}.
  \OtterFlag{para\_from\_right		}{boolean}.
  \OtterFlag{para\_from\_units\_only 	}{boolean}.
  \OtterFlag{para\_from\_vars		}{boolean}.
  \OtterFlag{para\_into			}{boolean}.
  \OtterFlag{para\_into\_left		}{boolean}.
  \OtterFlag{para\_into\_right		}{boolean}.
  \OtterFlag{para\_into\_units\_only 	}{boolean}.
  \OtterFlag{para\_into\_vars		}{boolean}.
  \OtterFlag{para\_ones\_rule		}{boolean}.
  \OtterFlag{para\_skip\_skolem		}{boolean}.
  \OtterFlag{pick\_given\_ratio		}{integer}.
  \OtterFlag{pretty\_print		}{default}.
  \OtterFlag{pretty\_print\_indent	}{integer(4,16)}.
  \OtterFlag{print\_back\_demod		}{boolean}.
  \OtterFlag{print\_back\_sub		}{boolean}.
  \OtterFlag{print\_given		}{boolean}.
  \OtterFlag{print\_kept		}{clear}.
  \OtterFlag{print\_lists\_at\_end	}{boolean}.
  \OtterFlag{print\_new\_demod		}{boolean}.
  \OtterFlag{print\_proofs		}{default}.
  \OtterFlag{process\_input		}{boolean}.
  \OtterFlag{prolog\_style\_variables	}{set}.
  \OtterFlag{propositional		}{default}.
  \OtterFlag{really\_delete\_clauses	}{default}.
  \OtterFlag{report			}{integer}.
  \OtterFlag{simplify\_fol		}{boolean}.
  \OtterFlag{sort\_literals		}{boolean}.
  \OtterFlag{sos\_queue			}{boolean}.
  \OtterFlag{sos\_stack			}{boolean}.
  \OtterFlag{stats\_level		}{default}.
  \OtterFlag{symbol\_elim		}{boolean}.
  \OtterFlag{term\_wt\_max\_args	}{boolean}.
  \OtterFlag{unit\_deletion		}{boolean}.
  \OtterFlag{ur\_res			}{boolean}.
  \OtterFlag{very\_verbose		}{default}.
\end{tabular}\hfill\null}

  \caption{The Flags of Otter}\label{fig:otter.flags}
\end{figure}

The table \ref{fig:otter.flags} contains a complet list of Otter flags as
understood by the \ProTop/Otter interface.
See the documentation of Otter \cite{mccune:otter} for further details.

{\em Type} can have one of the following values:

\begin{list}{}{\parsep=0pt\itemsep=0pt\labelwidth=9em\leftmargin=10em}
\item [\bf boolean] denotes a boolean flag. it can be turned on or off.
\item [\bf set] denotes a boolean flag which must be turned on.
\item [\bf clear] denotes a boolean flag which must be turned off.
\item [\bf default] denotes a flag which can not be changed.
\item [\bf integer] denotes a numeric flag which can take arbitrary values.
\item [\bf integer({\it min},{\it max})] denotes a numeric flag which can take
  values in the range from {\it min}\/ to {\it max}. 
\end{list}

%%*****************************************************************************
%% $Id$
%%*****************************************************************************
%%% 
%%% This file is part of ProCom.
%%% It is distributed under the GNU General Public License.
%%% See the file COPYING for details.
%%% 
%%% (c) Copyright 1995 Gerd Neugebauer
%%% 
%%% Net: gerd@imn.th-leipzig.de
%%% 
%%%****************************************************************************

\chapter{Using Setheo within \ProTop}


\section{Overview}

\ProTop{} is able to handle internal and external provers. The capability to
handle external provers is demonstrated by the integration of otter. Otter is
a prover written in C which has to be run as a separate process. Setheo
consists of three programs which have to be run one after the other. The input
matrix is given to the first program. Intermediate files are created by Setheo
to hand on the current representation of the problem. Each program can receive
flags in the command line.

\ProTop{} has to translate to internal representation of the matrix in a form
acceptable by Setheo. The external processes have to be started and finally
the output has to be analyzed and translated into a form handeled by \ProTop.

To start Setheo the option |prover| has to be set to the value |setheo|. Then
the proof can be started as usual. This is shown in the example in
figure~\ref{fig:setheo} which assumes us to be in the \ProTop{} interactive
mode.

\begin{figure}[ht]
\begin{BoxedSample}
 ProTop -> |prover = setheo.|
 ok.
 ProTop -> |prove 'eder1-2'.|
 \% No input filter requested
 \% Reading matrix file eder1-2
 2 clauses read in 0 ms 
 Connection graph: 17 ms 
 complete\_goals.pl compiled traceable 1396 bytes in 0.03 seconds
 inwasm V3.2 Copyright TU Munich (March 25, 1994) 
 command line: /home/setheo/bin/inwasm -cons -verbose eder1-2 
 eder1-2.s generated in  0.00 seconds
 wasm V3.2 Copyright TU Munich (March 25, 1994)
 Command line: /home/setheo/bin/wasm -opt -verbose eder1-2 
 Assembler optimization: 16 labels read, 8 labels output
 eder1-2.hex generated in  0.03 seconds
 \% All negative clauses are considered as goals.
 complete\_goals 0 ms 

 \%\char"7C\ Proof with goal clause 1
 \%\char"7C\      ext
 \%\char"7C\      2 - 1
 ...
 ok.
 ProTop -> 
\end{BoxedSample}
\caption{A session with setheo.}\label{fig:setheo}
\end{figure}



\section{Options for Setheo}

The module setheo provides the following options to adjust its behaviour:

\begin{description}
  \item [setheo:flags] | = [cons,opt,verbose,dr].|
    \\
    This option contains a list of flags for Setheo. a detailed description can
    be found in section \ref{setheo:flags}.

  \item [setheo:executable] | = "setheo".|
    \\
    This option contains the executable UNIX command which starts setheo. If
    this executable can not be found on the search path of the UNIX shell used
    it might be neccesary to specify the full file name --- including the
    complete path.

    This option is initialized from the valued specified in the Makefile
    during the installation process. Usually it should not need modification.

  \item [setheo:tmpdirs] | = [indir,".","/tmp","~"].|
    \\
    This option specifies a list of directories which might be used to create
    intermediate files. The list is tried from left to right to find a
    directory which is writable. The first such directory is used to create
    the intermediate files for setheo.

    The elements of this list are interpreted as directories. They can be
    absolute or relative to the current directory. The tilde |~| can be used
    to refer to the home directory of the user. The special symbol |indir| can
    be used to refer to the directory containing the input file.

  \item [setheo:remove\_tmp\_files] | = off.|
    \\
    At the end of the proof attempt the intermediate files created for the
    communication with setheo are removed if this boolean option is |on|.
    Otherwise they are left unchanged. This can be useful to understand how
    setheo is called and get more details what setheo has done.

\end{description}


\section{Flags of Setheo}\label{setheo:flags}

Setheo can be configured with a large number of flags. Most of those
flags\footnote{Only those flags which might disturb \ProTop{} have been
  disabled.} are usable from within \ProTop. For a detailed discussion of the
flags we refer to the Setheo documentation. The following description only
mentiones a few. The main emphasis is on the way how to use them from within
\ProTop.

Setheo has two kind of flags: boolean flags and numeric flags. The
specifications for those flags is combined in the option |setheo:flags| which
contains a list of those.

The names of the flags are symbols in the sense of Prolog. Thus those names
are used to name the flags in the context of \ProTop{} as well. The Setheo
interface known to which program a flag belongs. Thus all relevant flags for
the programs are used and the others omitted.


\subsection{Boolean Flags}

Boolean flags can be turned on by specifying them on the command line. In
absense they are turned off.

E.g. if the list in |setheo:flags| contains the element |dr| then the Setheo
(|sam|) flag |dr| is set.\footnote{The flag |dr| turns on the iterative
  deepening search on the depth of the proof tree.}


\subsection{Numeric Flags}

Numeric flags can take numeric values. Arbitrary numbers can be assigned to
numeric flags.

To assign a value {\em n}\/ to a numeric flag {\em flag}\/ simply add an
element {\em flag {\tt=} n} to the list of flags |setheo:flags|.

E.g. if the list in |setheo:flags| contains the element |trail=1000000| then
the numeric flag |trail| (of |sam|) is set to the value 1000000.


\begin{figure}[t]
{\scriptsize
\catcode`|=\other
\def\SetheoFlag#1#2#3{\tt #1 \sl #2&#3\\\hline}%
\null\hfill\hfill
\begin{tabular}{|p{10em}c|}\hline \bf Flag & \bf Program \\\hline\hline
  \SetheoFlag{purity}{}{inwasm}
  \SetheoFlag{nopurity}{}{inwasm}
  \SetheoFlag{nofan}{}{inwasm}
  \SetheoFlag{reduct}{}{inwasm}
  \SetheoFlag{noreduct}{}{inwasm}
  \SetheoFlag{nosgreord}{}{inwasm}
  \SetheoFlag{noclreord}{}{inwasm}
  \SetheoFlag{notree}{}{inwasm}
  \SetheoFlag{randreord}{}{inwasm}
  \SetheoFlag{eqpred}{}{inwasm}
  \SetheoFlag{all}{}{inwasm}
  \SetheoFlag{reg}{}{inwasm}
  \SetheoFlag{subs}{}{inwasm}
  \SetheoFlag{taut}{}{inwasm}
  \SetheoFlag{cons}{}{inwasm}
  \SetheoFlag{foldup}{}{inwasm}
  \SetheoFlag{foldupx}{}{inwasm}
  \SetheoFlag{folddown}{}{inwasm}
  \SetheoFlag{folddownx}{}{inwasm}
  \SetheoFlag{partialtree}{}{inwasm}
  \SetheoFlag{nopartialtree}{}{inwasm}
  \SetheoFlag{verbose}{= number}{inwasm}
  \SetheoFlag{verbose}{}{inwasm}
  \SetheoFlag{verbose}{}{wasm}
  \SetheoFlag{opt}{}{wasm}
\end{tabular}\hfill
\begin{tabular}{|p{10em}c|}\hline \bf Flag & \bf Program \\\hline\hline
  \SetheoFlag{d}{= number}{sam}
  \SetheoFlag{d}{}{sam}
  \SetheoFlag{i}{= number}{sam}
  \SetheoFlag{i}{}{sam}
  \SetheoFlag{dr}{= number}{sam}
  \SetheoFlag{dr}{}{sam}
  \SetheoFlag{ir}{= number}{sam}
  \SetheoFlag{ir}{}{sam}
  \SetheoFlag{loci}{}{sam}
  \SetheoFlag{locir}{}{sam}
  \SetheoFlag{anl}{}{sam}
  \SetheoFlag{reg}{}{sam}
  \SetheoFlag{st}{}{sam}
  \SetheoFlag{cons}{}{sam}
  \SetheoFlag{code}{= number}{sam}
  \SetheoFlag{stack}{= number}{sam}
  \SetheoFlag{cstack}{= number}{sam}
  \SetheoFlag{heap}{= number}{sam}
  \SetheoFlag{trail}{= number}{sam}
  \SetheoFlag{symbtab}{= number}{sam}
  \SetheoFlag{seed}{= number}{sam}
  \SetheoFlag{v}{= number}{sam}
  \SetheoFlag{v}{}{sam}
  \SetheoFlag{verbose}{= number}{sam}
  \SetheoFlag{verbose}{}{sam}
\end{tabular}\hfill\hfill\null}

  \caption{The Flags of Setheo}\label{fig:setheo.flags}
\end{figure}

\subsection{List of Flags}

The table \ref{fig:setheo.flags} contains a complet list of Setheo flags as
understood by the \ProTop/Setheo interface.  {\em Program}\/ denotes one of
the setheo programs |inwasm|, |wasm|, or |sam|.  If the flag is followed by
``{\em = number}'' then a numeric argument is required. See the documentation
of Setheo for further details.

%
% Local Variables: 
% mode: latex
% TeX-master: nil
% End: 


\part{\ProTop\ for Programmers and Administrators}
%\appendix
%%%****************************************************************************
%%% $Id: filter.tex,v 1.3 1995/03/20 21:24:47 gerd Exp $
%%%============================================================================
%%% 
%%% This file is part of ProCom.
%%% It is distributed under the GNU General Public License.
%%% See the file COPYING for details.
%%% 
%%% (c) Copyright 1995 Gerd Neugebauer
%%% 
%%% Net: gerd@imn.th-leipzig.de
%%% 
%%%****************************************************************************

\chapter{Writing Filters}\label{chap:writing.filters}

This chapter describes how to write an input filter for \ProTop. A filter is a
simple \eclipse{} module which satisfies certain restrictions. The most simple
filter is delivered as the file {\sf Filters/none.pl}. This filter does
nothing but pass everything it reads to the next filter. It gives a general
structure for any special filter.

The first thing we have to take care of is the naming convention for filters.
Three entities which can be used independently are now linked together:
\begin{enumerate}
\item The name of the module. Let's call it {\em my\_filter} for the moment.
\item The name of the file containing the module. This has to be {\em
    my\_filter.pl}, i.e. the name of the module plus an extension of {\tt
    .pl}.
\item One predicate exported by {\em my\_filter}\/ has to be {\em
    my\_filter/A}, where {\em A}\/ is the arity of the filter. The first two
  arguments are occupied by the input and the output stream. Additional
  arguments can be required by the filter. Nevertheless it turned out to be a
  good practice to define the predicate {\em my\_filter/2} which provides
  reasonable defaults for the additional arguments.
\end{enumerate}

In the following we consider a filter without additional arguments, i.e.\ the
filter predicate has arity 2.  Thus the beginning of our filter {\em
  my\_filter} in the file {\em my\_filter.pl} looks as follows:

\begin{BoxedSample}\raggedright\tt
:- module\_interface({\em my\_filter}).
:- export {\em my\_filter}/2.
:- begin\_module({\em my\_filter}).
info(filter,"Vers. 0.1","my\_filter is a filter to do ...").
:- lib(op\_def).
\end{BoxedSample}

The info fact identifies this module to contain a filter. The first argument
is the atom |filter|. The second argument is a string containing a version
identification. The third argument contains a string with a {\em short}\/
description. 

The final instruction in the example above loads the library containing
definition of operators.  Especially the \verb|#| is defined there. The
meaning and use of this functor will be described later.

The predicate {\em my\_filter/2} takes as first argument an input stream and
as second argument an output stream. Opening and closing of these streams is
performed by \ProTop. All {\em my\_filter/2} has to do is to read from the
input stream and write the result to the output stream.

Since many Prolog terms may be waiting to be processed a loop has to be used.
The termination condition is the term \verb|end_of_file|. If this term is read
{\em my\_filter/2} should return with success without leaving a choice point.
{\em my\_filter/2} is not allowed to fail!

The non-failing condition can easily be satisfied by a failure driven loop. The
omission of a choice point is guaranteed by a terminating cut (which is good
practice for a failure driven loop anyhow).

\begin{BoxedSample}\raggedright\tt\obeyspaces
my\_filter(Stream,OutStream) :-
        repeat,
        read(Stream,Term),
        ( Term = end\_of\_file ->
            true
        ;   writeclause(OutStream,Term),
            fail
        ),
        !.
\end{BoxedSample}

This simple example can now be enhanced by replacing the |writeclause|
predicate by a more complicated construct which transforms the term read and
writes the result to |OutStream|.

The predicate |writeclause| is highly recommended since it is guaranteed to
produce output which can be read back into Prolog\footnote{Currently this is
not really true in \eclipse.}.

To allow certain filters to have private informations in the input we have
developed a discipline of programming filters. The |#| is assigned a special
meaning only in filters. |#| is declared as prefix operator.  The following
rules must be honored:

\begin{itemize}
  \item Any |#| instruction not understood by a filter must be passed
  to the next filter unchanged.

  \item The construction |#| {\em Option}|=|{\em Value}|.| is reserved for
  assignments of options.

  \item |#begin(|{\em section}|)| and |#end(|{\em section}|)| serve as
  grouping constructs. Grouping constructs for the same section can not be
  nested. Unknown groups must be passed unchanged to the output stream.
\end{itemize}





%\chapter{File Formats}

%\section{The Proof Tree Format}

%\section{The Report File Format}



\chapter{The Prolog Interface to \ProTop}
%%*****************************************************************************
%% $Id: matrix.tex,v 1.7 1995/04/24 21:29:11 gerd Exp $
%%*****************************************************************************
%% Author: Gerd Neugebauer
%%-----------------------------------------------------------------------------

\section{The Library Matrix}
\def\PrologFILE{System/matrix.pl}

The matrix, which is read and processed with filters, is parsed and stored in
the module {\sf matrix}. The internal organization of this module should not
be relevant for all applications using it. In fact the module {\sf matrix} has
gone through several incarnations where the data has been stored in the Prolog
data base (assert), the indexed data base (record), and the \eclipse{} global
variables. Anybody using the interface predicates only will not even recognize
if a new version is installed --- except for different execution times.


Inside a Prolog program running under the control of \ProTop, it is
sufficient to request the interface with
\begin{BoxedSample}
  :- lib(matrix).
\end{BoxedSample}
It might also turn out as useful to use the library {\sf literal} as well.

The library {\sf matrix} uses hooks wherever possible to allow the user to
perform actions whenever the matrix is changed. The concept and use of hooks
is described in the documentation of the library {\sf hooks}. The hooks are
meant as a means to store additional information in other modules. For other
purposes they may not turn out as appropriate.

The predicates provided by the library {\sf matrix} can be categorized as
follows. Primarily there are predicates to access the information about the
matrix. Additionally there are predicates to add further information to the
matrix or to remove it. Finally printing routines for the matrix are provided.

\subsection{The Data Structures}\label{matrix:data.structures}

The module {\sf matrix} is able to store a single matrix in clause normal
form. Since we are not interested in manipulating more than one matrix of
treat the matrix as a object of its own we have no explicit data structure for
a matrix. Instead we manipulate clauses and literals.


As an illustrative example in the remainder of this section we will use the
following matrix.
\begin{BoxedSample}
       [p(X), -q(f(X),g(X))].
base:: [q(a,X)].
loop:: [q(X,Y), -q(f(X),Y)].
?-     [-p(Z)].
\end{BoxedSample}
This matrix is presented in its external representation. Two clauses have labels
assigned to them and one clause is marked as a goal. For convenience it is
possible to write the negation sign as |-| and to omit the positive sign at all.
Internally we use a more strict regime. The central naming conventions for the
data structures used are discussed next.

\begin{description}
\item[Literal] \ \\
  A literal is a Prolog term of the form |++|{\em Predicate}\/ or |--|{\em
    Predicate}. From our example we can consider the literals in external
  representation |q(a,X)| and |-q(f(X),Y)|. The internal representation would
  be the following two literals
\begin{BoxedSample}
    ++q(a,X)
    --q(f(X),Y)
\end{BoxedSample}

\item[Index]\ \\
  Any literal is uniquely identified by an index. This index is assigned to
  the literals when the clauses or contrapositives are added to the matrix.
  Predicates are provided to manipulate the literal index. Use those
  predicates and do not assume anything about the index except that it
  uniquely points to a literal.

  At the time this is written the index is of the form {\em C|-|L}, where
  {\em C}\/ is a number pointing to the clause and {\em L}\/ is a literal
  pointing to the literal in this clause. When you are reading this, the
  internal representation of an index may have changed already. For reasons of
  efficiency, atoms instead of compound terms may be used as indices.
  
  {\large\bf Attention:} Do {\em not}\/ assume that the index is constructed in
  a special way, i.e. to be of the form {\em C|-|L}. Do {\em not}\/ assume
  that {\em C}\/ or {\em L}\/ are numbers!

\item[Clause Index]\ \\
  A clause index is a unique identifier for a clause. Several predicates are
  provided which take a clause index as argument. Especially one predicate is
  provided to get the clause index from an index.

  Currently the clause index is a number. This may change soon.

  {\large\bf Attention:} Do {\em not}\/ assume that the clause index is a
  number!

\item[Contrapositive]\ \\
  A clause can be accessed through the various literals contained in it. This
  leads to the concept of a contrapositive. A contrapositive is a set of
  literals which (normally) corresponds to the literals of a clause where one
  literal is distinguished. This literal is called ``entry point''. A literal
  can be described by the sign and predicate or by its unique index. Thus a
  contrapositive relates a literal and its index to the remaining set of
  literals.

  For some purposes it seemed to be desirable to manipulate various
  contrapositives of a clause in different ways. One application are Horn
  problems where at most one contrapositive of every clause is needed. For the
  goal no contrapositive is needed at all. Nevertheless the prover has to
  start the prove process and thus it has to get the goal clause.

\item[Clause]\ \\
  A clause is a set of literals. It is uniquely identified by a clause
  index. The clause may have contrapositives assigned to it but this is not
  neccesary. A clause can exist without having any contrapositives.

\item[Connection]\ \\
  Two literals or contrapositives can be related when they are potentially
  conplementary. This relation is also managed in the module {\sf matrix}. The
  relation itself has to be specified externally. The only property maintained
  in this module is that connections can only exist between literals present
  in it.

  Note that the connections managed by this module are directed connections.
\end{description}



\subsection{Accessing the Matrix}

\Predicate Contrapositive/3(?Literal, ?Body, ?Index).

This predicate can be used to get the complete information about a
contrapositive in the matrix.\footnote{To be precise we have to note that
  different entry points to a clause may keep different information.} This
predicate assumes that either the index |Index| is ground or the literal
|Literal| is sufficiently instantiated. In the first case at most one solution
is returned. In the second case all solutions are returned upon backtracking.
If neither the literal nor the index is given all contrapositives are
enumerated. This is the worst case and should be avoided.

|Literal| is unified with a literal (in internal representation; see
\ref{matrix:data.structures}). |Index| is a literal index. |Body| is a list
containing elements of the form {\em literal(L,I)}\/ where {\em L}\/ is a
literal and {\em I}\/ is an index. |Body| contains all literals which are in
the contrapositive of |Literal| except |Literal| itself. The complete clause
represented by this contrapositive would be
\begin{BoxedSample}
  [ literal(Literal,Index) \char"7C Body ]
\end{BoxedSample}


Let us assume that the matrix given earlier is stored in the module {\sf
  matrix}. Then we can access it as shown in the next example.
\begin{BoxedSample}
  ?- 'Contrapositive'(q(A,B),Body,Index).
  A = a
  B = X
  Body = []
  Index = 2-1
  ;
  A = X
  B = Y
  Body = [literal(--q(f(X),Y),3-2)]
  Index = 3-1
  ;
  No more solution
\end{BoxedSample}


\Predicate PositiveClause/1(?ClauseIndex).

This predicate unifies the clause index |ClauseIndex| with those clauses which
contain positive clauses only. All literals currently
in the clause are considered. Even so a single contrapositive may or may not be positive the
whole clause can have an independent property.
\begin{BoxedSample}
  ?- 'PositiveClause'(CI).
  CI = 2
\end{BoxedSample}

\Predicate NegativeClause/1(?ClauseIndex).

This predicate unifies the clause index |ClauseIndex| with those clauses which
contain negative clauses only. All literals are considered which are currently
in the clause. Even so a single contrapositive may or may not be negative the
whole clause can have an independent property.
\begin{BoxedSample}
  ?- 'NegativeClause'(CI).
  CI = 4
\end{BoxedSample}

\Predicate ClauseLength/2(?ClauseIndex,?Length).

This predicate enumerates all pairs of clause indices |ClauseIndex| and the
length |Length| of the clause belonging to |ClauseIndex|. This information is
extracted for the complete clause. Single contrapositives may have a different
length.

\begin{BoxedSample}
  ?- 'ClauseLength(1-1,Len).
  Len = 2
\end{BoxedSample}


\Predicate Clause/1(?ClauseIndex).

This predicate unifies |ClauseIndex| with each index of a clause currently
stored in the module matrix in turn.
\begin{BoxedSample}
  ?- findall( CI, 'Clause'(CI), ClauseList ).
  ClauseList = [1,2,3,4]
\end{BoxedSample}

\Predicate Clause/2(?ClauseIndex, ?Body).

This predicate takes a clause index |ClauseIndex| and unifies |Body| with a
list of the form {\em literal(L,I)}\/ where {\em L}\/ is a literal and {\em
  I}\/ is an index. Body contains all literals left in the specified clause.

The literals in the clause are not modified when a single contrapositive is
changed. Only the deletion of a literal in a clause, building the resolvent,
or deleting a complete clause may change the information returned by this
predicate.

Upon backtracking all unifiable pairs of clause index and literal list are
returned.
\begin{BoxedSample}
  ?- 'Clause'(3,Body).
  Body = [literal(++q(X,Y),3-1),literal(--q(f(X),Y),3-2)]
\end{BoxedSample}

\Predicate GoalClause/1(?ClauseIndex).

This predicate unifies |ClauseIndex| with the index of each clause which is
marked as goal. Several solutions can be returned upon backtracking.
\begin{BoxedSample}
  ?- findall( CI, 'GoalClause'(CI), ClauseList ).
  ClauseList = [4]
\end{BoxedSample}

\Predicate Label/2(?ClauseIndex, ?Label).

This predicate enumerates all pairs consisting of a clause index |ClauseIndex|
and an associated label |Label|. Note that the goal label is a special kind of
label which is not returned with this predicate.
\begin{BoxedSample}
  ?- 'Label'(CI,Label).
  CI = 2
  Label = base
  ;
  CI = 3
  Label = loop
  ;
  No more solution
\end{BoxedSample}

\Predicate ClauseLiteral/2(?ClauseIndex, ?Index).

This predicate establishes a relation between the clause and literal indices.
It can be used to find one or all literals in a clause as well as to find the
clause index of an index.

\begin{BoxedSample}
  ?- 'ClauseLiteral'(1,Index).
  Index = 1-1
  ;
  Index = 1-2
  ;
  No more solution

  ?- 'ClauseLiteral'(CI,3-2).
  CI = 3
\end{BoxedSample}


\Predicate Connection/2(?FromIndex, ?ToIndex).

This predicate relates the indices |FromIndex| and |ToIndex| according to the
connectedness. The connection graph contains an arc from a literal $L$\/ to
another literal $L'$\/ if an extension or reduction from $L$\/ to $L'$\/ can
be part of the final proof. This information is provided by the module {\sf
  connection\_graph}. 


\Predicate new_clause_index/1(-ClauseIndex).

This predicate unifies |ClauseIndex| with a unique new clause index. This
predicate is not resatisfiable.
\begin{BoxedSample}
  ?- new\_clause\_index(CI).
  CI = 5
\end{BoxedSample}


\Predicate new_literal_index/1(+-Index).

This predicate unifies |Index| with a unique new index in the clause
|ClauseIndex|. The index is not reserved in any way. If no associated literal
is added then the new index will be returned upon the next invocation of
|new_literal_index/1| again. This predicate is not resatisfiable.
\begin{BoxedSample}
  ?- new\_literal\_index(3,I).
  I = 3-3
\end{BoxedSample}


\subsection{Adding Information to the Matrix}

\Predicate add_clause/1(+Literals).

This predicate takes a list of literals |Literals| and stores all
contrapositives in the module {\sf matrix}. The hook |add_clause_hook/2| is
evaluated with the index of the new clause and the list of literals |Literals|
as arguments. This hook is evaluated after the literals have been added.

The list of literals |Literals| can be preceded by the goal label |?-|. In
this case the goal label is set also for this clause and the hook
|add_goal_clause_hook/1| is evaluated at the end. The argument is the index of
the goal clause.

The following instructions add all clauses to the module {\sf
  matrix}. Nevertheless the labels and the connections have not been considered yet.
\begin{BoxedSample}
  ?- add\_clause([++p(X),--q(f(X),g(X)]).
  ?- add\_clause([++q(a,X)]).
  ?- add\_clause([++q(X,Y),--q(f(X),Y)]).
  ?- add\_clause((?-[--p(Z)])).
\end{BoxedSample}

\Predicate add_contrapositive/3(+Literal, +Body, +Index).

This predicate adds the contrapositive with the given information to the
matrix. The entry point of the contrapositive is the literal |Literal| which
has the index |Index|. |Body| is a list of elements of the form {\em
  literal(L,I)}\/ where {\em L}\/ is a literal and {\em I}\/ is its index. The
literals of |Body| are the literals in the contrapositive except |Literal|.

\begin{BoxedSample}
  ?- add\_contrapositive(++p(X),[literal(--q(f(X),g(X)),1-2)],1-1).
\end{BoxedSample}

\Predicate add_connection/2(+FromIndex, +ToIndex).

This predicate adds a connection from the literal |FromIndex| to the literal
|ToIndex|. If such a connection already exists then this predicate simply
succeeds. The arguments have to be ground.
\begin{BoxedSample}
  ?- add\_connection(4-1,1-1).
\end{BoxedSample}


\Predicate add_goal_label/1(+ClauseIndex).

This predicate marks the clause with the index |ClauseIndex| as goal. {\em
  No}\/ implicit negation is performed! If the given clause has a goal
label already then the predicate simply succeeds.

The following example adds the goal label to each negative clause.
\begin{BoxedSample}
  ?- 'NegativeClause'(NC), add\_goal\_label(NC), fail; true.
\end{BoxedSample}

\Predicate add_label/2(+ClauseIndex, +Label).

This predicate adds the label |Label| to the clause with the index
|ClauseIndex|. If the label already exists the predicate simply succeeds.

The following goal may be called from within the parser to store the label of
the third clause.
\begin{BoxedSample}
  ?- add\_label(3,loop).
\end{BoxedSample}

\Predicate add_resolvent/3(+Index1, +Index2, +Dest).

This predicate performs a single resolution step. The literals $L_1$\/
identified by |Index1| and the literal $L_2$\/ identified by |Index2| must
have complementary signs. Those signs are stripped and the pure predicates are
unified. This unification uses an occurs check.\footnote{This unification may
  also take into account \eclipse{} metaterms.}

The further operations are determined by the value of |Dest|. The following
possibilities are provided:
\begin{description}
\item[clause({\em ClauseIndex}\/)]\ \\
  If {\em ClauseIndex}\/ is instantiated then the resolvent replaces the clause
  {\em ClauseIndex}. If {\em ClauseIndex}\/ is a variable then a new clause is
  created containing the resolvent. {\em ClauseIndex}\/ is unified with the
  resulting clause.
\item[contrapositive({\em Index}\/)]\ \\
  The resolvent is only performed in the contrapositive {\em Index}. This
  contrapositive is replaced by the result. {\em Index}\/ has to be
  sufficiently instantiated. Additionally {\em Index}\/ has to be in a common
  contrapositive with either |Index1| or |Index2|.
\end{description}




\subsection{Removing Information from the Matrix}


\Predicate delete_matrix/0().

This predicate removes all information stored in the module {\sf matrix}.
Afterwards the information is no longer accessible. Before the matrix is
cleared the hook |delete_matrix_hook/0| is evaluated. At this time the matrix
is still intact. Afterwards the clauses are deleted. This may lead to the
execution of the hook |delete_clause/1|. See the documentation of the
predicate |delete_clause_hook/1| for details.
\begin{BoxedSample}
  ?- delete\_matrix.
\end{BoxedSample}

\Predicate delete_clause/1(?ClauseIndex).

This predicate deletes all clauses --- upon backtracking --- for which the
clause index unifies with |ClauseIndex|.

All information having any relation to the clause being removed is erased.

The hook |delete_clause_hook/1| is evaluated for each clause removed. The
argument is the instantiated clause index of the clause being deleted. The
hook is called before any other information about this clause is removed from
matrix. 

The following example deletes the clause with the index 3.
\begin{BoxedSample}
  ?- delete\_clause(3).
\end{BoxedSample}

\Predicate delete_contrapositive/1(?Index).

This predicate deletes a single contrapositive identified by |Index|. |Index|
may be un-instantiated in which case all contrapositives are removed upon
backtracking. The connections starting from the literal |Index| are also
removed.

The clause as a whole is not affected. Nor are the |ClauseLength| or other
informations related to the whole clause. As a consequence the clause can
continue to exist even though there are no contrapositives left for it.

The hook |delete_contrapositive_hook/1| is evaluated for each contrapositive
being removed. The hook is called before the information is erased from the
matrix. The argument of the hook is the instantiated index of the
contrapositive being deleted.

\begin{BoxedSample}
  ?- delete\_contrapositive(3-2).
\end{BoxedSample}

\Predicate delete_connection/2(?FromIndex, ?ToIndex).

This predicate deletes upon backtracking all connections for which the
starting point unifies with |FromIndex| and the end point unifies with
|ToIndex|. 

The hook |delete_connection_hook/2| is evaluated with the instantiated
|FromIndex| and |ToIndex| as argument for every connection being removed.

The following example deletes all connections stored in the matrix.
\begin{BoxedSample}
  ?- delete\_connection(\_,\_),fail ; true.
\end{BoxedSample}

\Predicate delete_literal/1(+Index).

This predicate deletes the literal identified by |Index| from the matrix. Any
information is abolished. All connections to and from this literal are
deleted. The literal does not occur any longer in any contrapositive or the
whole clause.

The literal has to be sufficiently instantiated. The predicate fails if the
specified literal does not exist.

The hook |delete_literal_hook/1| is evaluated before the literal is removed.
The argument of this hook is the index of the literal to be deleted.

The hooks |delete_contrapositive_hook/1| and |delete_connection_hook/2| may be
called when the respective information is deleted.

The following example deletes the literal |3-2| from the matrix.
\begin{BoxedSample}
  ?- delete\_literal(3-2).
\end{BoxedSample}

\Predicate delete_goal_label/1(?ClauseIndex).

This predicate deletes the goal label of the given clause. |ClauseIndex| is in
turn unified with all clause indices having a goal label. The   predicate is
resatisfiable until no more unifying goal clause is present.
\begin{BoxedSample}
  ?- delete\_goal\_label(CI).
  CI = 4
  ;
  No more solution
\end{BoxedSample}

\Predicate delete_label/2(?ClauseIndex, ?Label).

This predicate deletes the label |Label| of the given clause |ClauseIndex|.
The predicate is resatisfiable until no more labels are present.
\begin{BoxedSample}
  ?- delete\_label(CI,Label).
  CI = 2
  Label = base
  ;
  CI = 3
  Label = loop
  ;
  No more solution
\end{BoxedSample}


\subsection{Printing the Matrix}

\Predicate show_matrix/2(+Prefix, +Stream).

The current matrix is printed onto the stream |Stream|.  Each line written is
preceded by the string |Prefix|.  The format used is suitable for reading it
back in if |Prefix| is the empty string.

|Stream| can also be a symbolic stream of \eclipse. E.g. |output| directs the
matrix to the standard output stream.
\begin{BoxedSample}
  ?- show\_matrix("\%", output).
  \%      [ \% Clause 1
  \%        ++ p(X),
  \%        -- q(f(X),g(X))
  \%      ].
  \% base:: [  \% Clause 2
  \%        ++ q(a,X)
  \%      ].
  \% loop:: [  \% Clause 3
  \%        ++ q(X,Y),
  \%        -- q(f(X),Y)
  \%      ].
  \% ?-   [  \% Clause 4
  \%         -- p(Z)
  \%      ].
\end{BoxedSample}

\Predicate show_contrapositives/2(+Prefix, +Stream).

The current list of contrapositives is printed onto the stream |Stream|.  Each
line written is preceded by the string |Prefix|.

|Stream| can also be a symbolic stream of \eclipse. E.g. |output| directs the
matrix to the standard output stream.
\begin{BoxedSample}
  ?- show\_contrapositives(" ", output).
    ++ p(X) <- \% (1 - 1)
           -- q(f(X),g(X)).
    -- q(f(X),g(X)) <- \% (1 - 2)
           ++ p(X).
    ++ q(a,X). \% (2 - 1)
    ++ q(X,Y) <- \% (3 - 1)
           -- q(f(X),Y).
    -- q(f(X),Y) <- \% (3 - 2)
           ++ q(X,Y).
    -- p(Z). \% (4 - 1)
\end{BoxedSample}

%
% Local Variables: 
% mode: latex
% TeX-master: t
% End: 


%%%****************************************************************************
%%% $Id: install.tex,v 1.3 1995/07/03 11:35:12 gerd Exp gerd $
%%%============================================================================
%%% 
%%% This file is part of ProCom.
%%% It is distributed under the GNU General Public License.
%%% See the file COPYING for details.
%%% 
%%% (c) Copyright 1995 Gerd Neugebauer
%%% 
%%% Net: gerd@imn.th-leipzig.de
%%% 
%%%****************************************************************************

\chapter{Installing \ProCom/\ProTop}

\section{Unpacking \ProCom/\ProTop{} from the Distribution}%
\label{sec:installation}

The \ProCom/\ProTop{} distribution comes in a single gzipped tar file {\sf
  procom-{\em VV}.tar.gz} where {\em VV} stands for a version number. First,
you have to choose a place where to install \ProCom/\ProTop. Change the
current directory to this directory:

| cd |{\em destination}

Execute the shell command

|  gunzip < |{\sf {\em SomeDir/}procom-{\em VV}.tar.Z} {\tt\char"7C} | tar -xvf -|

where {\em SomeDir}\/ is the directory where the distribution file is located.
This command generates a directory named {\sf ProCom} which contains the
distribution files.

Change the current directory to the {\sf ProCom} directory and edit the
Makefile.  Some of the variables might need adjustment.  Run the shell command

|  make config|

to generate the configuration file {\sf config.pl}.
Finally run the shell command

|  make protop|

This will generate an \eclipse{} saved state named {\sf protop}. This file is
executable and should be placed on your search path. Obviously this step
requires \eclipse{} to be installed and accessible under the name |eclipse|
--- or whatever has been configured in the Makefile.

The saved state |protop| will automatically start the \ProTop\ top level
loop.  If it is desirable to use the Prolog interface you can make a saved
state |protop.st| which just contains the preloaded Prolog files without
starting anything automatically. This is done with the command

|  make protop.st|



\section{Recompiling \ProTop}\label{sec:recompile}

Since the \ProTop\ system has been enhanced with the dynamic loading facility
it is not really neccesary to recompile it every now and then. This is only
neccesary when the installation directory is moved or when compiled in modules
need to be exchanged.

When you are recompiling \ProCom{} you have to consider some points.

\begin{itemize}
  \item Change the current directory to the installed \ProCom.
	We assume that \ProCom{} is properly installed.
  \item Edit the file {\sf Makefile}\/ to reflect the changed configuration.
  \item Make sure that the libraries are located by absolute path names only.
	This is strongly recommended to allow any user from any directory to
	use \ProCom{} without problems.
  \item Run the shell command

	|  make PROTOP=|{\em name}

	where {\em name}\/ is the name of the final executable. When this
	command is finished without errors the executable {\em name}\/ can be
	tested.
\end{itemize}



\bibliographystyle{named}
\bibliography{references}

\WithUnderscore{\printindex}

\end{document} %%%%%%%%%%%%%%%%%%%%%%%%%%%%%%%%%%%%%%%%%%%%%%%%%%%%%%%%%%%%%%%%
