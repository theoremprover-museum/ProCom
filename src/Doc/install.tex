%%%****************************************************************************
%%% $Id: install.tex,v 1.3 1995/07/03 11:35:12 gerd Exp gerd $
%%%============================================================================
%%% 
%%% This file is part of ProCom.
%%% It is distributed under the GNU General Public License.
%%% See the file COPYING for details.
%%% 
%%% (c) Copyright 1995 Gerd Neugebauer
%%% 
%%% Net: gerd@imn.th-leipzig.de
%%% 
%%%****************************************************************************

\chapter{Installing \ProCom/\ProTop}

\section{Unpacking \ProCom/\ProTop{} from the Distribution}%
\label{sec:installation}

The \ProCom/\ProTop{} distribution comes in a single gzipped tar file {\sf
  procom-{\em VV}.tar.gz} where {\em VV} stands for a version number. First,
you have to choose a place where to install \ProCom/\ProTop. Change the
current directory to this directory:

| cd |{\em destination}

Execute the shell command

|  gunzip < |{\sf {\em SomeDir/}procom-{\em VV}.tar.Z} {\tt\char"7C} | tar -xvf -|

where {\em SomeDir}\/ is the directory where the distribution file is located.
This command generates a directory named {\sf ProCom} which contains the
distribution files.

Change the current directory to the {\sf ProCom} directory and edit the
Makefile.  Some of the variables might need adjustment.  Run the shell command

|  make config|

to generate the configuration file {\sf config.pl}.
Finally run the shell command

|  make protop|

This will generate an \eclipse{} saved state named {\sf protop}. This file is
executable and should be placed on your search path. Obviously this step
requires \eclipse{} to be installed and accessible under the name |eclipse|
--- or whatever has been configured in the Makefile.

The saved state |protop| will automatically start the \ProTop\ top level
loop.  If it is desirable to use the Prolog interface you can make a saved
state |protop.st| which just contains the preloaded Prolog files without
starting anything automatically. This is done with the command

|  make protop.st|



\section{Recompiling \ProTop}\label{sec:recompile}

Since the \ProTop\ system has been enhanced with the dynamic loading facility
it is not really neccesary to recompile it every now and then. This is only
neccesary when the installation directory is moved or when compiled in modules
need to be exchanged.

When you are recompiling \ProCom{} you have to consider some points.

\begin{itemize}
  \item Change the current directory to the installed \ProCom.
	We assume that \ProCom{} is properly installed.
  \item Edit the file {\sf Makefile}\/ to reflect the changed configuration.
  \item Make sure that the libraries are located by absolute path names only.
	This is strongly recommended to allow any user from any directory to
	use \ProCom{} without problems.
  \item Run the shell command

	|  make PROTOP=|{\em name}

	where {\em name}\/ is the name of the final executable. When this
	command is finished without errors the executable {\em name}\/ can be
	tested.
\end{itemize}

