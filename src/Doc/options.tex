%%%****************************************************************************
%%% $Id: options.tex,v 1.9 1995/07/03 11:35:12 gerd Exp gerd $
%%%============================================================================
%%% 
%%% This file is part of ProCom.
%%% It is distributed under the GNU General Public License.
%%% See the file COPYING for details.
%%% 
%%% (c) Copyright 1995 Gerd Neugebauer
%%% 
%%% Net: gerd@imn.th-leipzig.de
%%% 
%%%****************************************************************************
% Master File: manual.tex

%------------------------------------------------------------------------------
\section{Options of \ProCom}\label{sec:options}

\ProCom{} can be adapted in it's behavior using a facility called options. A
option corresponds to a register or variable in procedural programming
language. Options of various kinds are used in \ProCom.

In general options can take any Prolog terms as values. Nevertheless there are
usually some kind of restrictions allowing only some kinds of values.

The most common options are boolean options. Boolean options can take the
values |on| and |off| only. In fact anything which is not |on| is interpreted
as |off|.

Another important type of option can take a file name. This is a Prolog string
pointing to a file name.

Options can be set in a options file which read at the beginning of any run of
\ProCom. Usually this file is called {\sf .procom} and is searched in the
current directory and the home directory in this order.

The \ProCom{} options file can contain instructions set options only.
This can be done using instructions of the following form:

|  |{\em Option}| = |{\em Value}|.|

In this instruction {\em Option}\/ denotes the name of an option and {\em
Value}\/ is it's value. Note that {\em Value} has to conform t Prolog
conventions and the terminating point is not optional! Possible options are
described below.


\subsection{General Options}

\begin{description}
  \item [prover] | = procom(extension_procedure)|\label{opt:prover}\\
	This option indicates the prover which has to be activated. The
	framework of \ProCom{} allows arbitrary provers to be integrated. For
	our purpose we describe only the interface to the provers provided
	with \ProCom{} and the interface to user defined provers.
	
	In this special case the value of this option is a Prolog term
	of the form |procom(|{\em Name}|)| where {\em Name}\/ is the
	name of a prover specified in the file {\sf procom\_config.pl}
	(see page \pageref{procom.config}).

  \item [search] | = iterative_deepening(1,1,1)|\\
	This option indicates which search strategy should be used.
	\begin{description}
	  \item [depth\_first]\ \\
		This search strategy uses the unbound depth first search like
		Prolog provides it.
	  \item [iterative\_deepening($\delta_0$, $\alpha$, $\beta$)]\ \\
		This search strategy successively increments a depth bound.
		The search space is searched to this limit. The limit is
		incremented upon failure.
		\\
		The initial depth $d_0$\/ is $\delta_0$.
		The next depth $d_{n+1}$\/ is computed from the previous depth
		$d_n$\/ according to the formula
		\[ d_{n+1} = \alpha\cdot d_n + \beta
		\]
		If $\alpha$\/ is $1$\/ then a linear function is achieved. If
		$\alpha$\/ is greater then $1$\/ then an exponential function
		can be forced.
	  \item [iterative\_deepening($\delta_0$, $\beta$)]\ \\
	        This is the same as 
		iterative\_deepening($\delta_0$, 1, $\beta$).
	  \item [iterative\_deepening($\delta_0$)]\ \\
	        This is the same as iterative\_deepening($\delta_0$, 1, 1).
	  \item [iterative\_deepening]\ \\
	        This is the same as iterative\_deepening(1, 1, 1).
	  \item [iterative\_inferences($\delta_0$, $\alpha$, $\beta$)]
	        This search strategy successively increments a inference bound.
		The search space is searched to this limit. The limit is
		incremented upon failure.
		\\
		The initial depth $d_0$\/ is $\delta_0$.
		The next depth $d_{n+1}$\/ is computed from the previous depth
		$d_n$\/ according to the formula
		\[ d_{n+1} = \alpha\cdot d_n + \beta
		\]
		If $\alpha$\/ is $1$\/ then a linear function is achieved. If
		$\alpha$\/ is greater then $1$\/ then an exponential function
		can be forced.
	  \item [iterative\_inferences($\delta_0$, $\beta$)]\ \\
	        This is the same as 
		iterative\_inferences($\delta_0$, 1, $\beta$).
	  \item [iterative\_inferences($\delta_0$)]\ \\
	        This is the same as iterative\_inferences($\delta_0$, 1, 1).
	  \item [iterative\_inferences]\ \\
	        This is the same as iterative\_inferences(1, 1, 1).
	  \item [iterative\_widening(a,b,c)]\ \\
		This is an experimental mode implementing iterative widening
		according to \cite{ginsberg.harvey:iterative}.

		{\bf Not ready yet!}
	  \item [iterative\_broadening(a,b,c)]\ \\
		This is an experimental mode implementing iterative
		broadening. 
	\end{description}

  \item [prolog] | = eclipse|\label{opt:prolog}\\
	This option indicates the target language which should be
	generated by \ProCom. Currently only a few possibilities are
	supported. The value |eclipse| indicates that \eclipse{} is
	the target language.  |quintus| stands for Quintus
	Prolog. Finally, |default| generates code in standart prolog,
	i.e.\ without special features of any Prolog dialect. In this
	mode several other options may have no effect and the
	resulting code may not be as efficient as in the specific
	modes.

  \item [input\_path] | = ['Samples']|\\
	This option contains a list of directories which are used to find an
	input file. This search path is used {\em after} the given file name
	has been tried as is. Thus the current directory (|.|) need not to be
	on this path.

  \item [verbose] | = on|\\
	This option turns on general verbosity of actions. Several modules may
	decide to use their own verbosity options.

  \item [toplevel:verbose] | = on|\\
	This option controlls the verbosity of the top level.


  \item [input\_filter] | = ''|\label{opt:input_filter}\\
	Filter to be used before the clauses are read. E.g. this filter can
	perform a normal form transformation. Filters may to be defined in
	{\sf config.pl} to be accessible. Otherwise they are dynamically
        loaded. In this case the file containing the filter has to be found on
        the search path for Prolog files.

	The value of this option can also be a list of filters. In this case
	each filter is used in turn. The output of the preceeding filter is
	given as input to the current filter. The initial filter gets the
	input file as input. The result is the result of the last filter.

        If only one filter is specified then the list can be omitted. I.e. a
        symbol or string value of this option is interpreted as a filter to be
        used.

	Non-existing filters are ignored.

	Let us consider an (imaginary) example:

\begin{BoxedSample}
  input\_filter = [nf\_transform,reductions1,reductions2]
\end{BoxedSample}

\end{description}

\subsection{Options Controlling the Prove Phases}

The prover can roughly be divided into two phases. The first phase is called
the preprocessing phase. In this phase several redution techniques can be
applied to reduce the size or complexity of the problem. The second phase is
the theorem prover itself. According to this overall distinction there are
some options controlling the general behaviour.

\begin{description}
\item [prove:red\_goals] |= [complete_goals,connection_graph]|\\
  This option is a list of modules which are invoked in turn to perform their
  tasks during the preprocessing phase. The elements of this list are
  reduction modules. Those modules are either preloaded or they are
  dynamically loaded.
  
\item [prove:red\_path] |= ['.']|\\
  This option contains the path for the dynamic loading of reduction
  modules. During the installation this path is augmented by the directory
  containing \ProTop{} and its files.

\item [prove:path] |= ['.']|\\
  This option contains the path for the dynamic loading of prover modules.
  The prover to be used is determined by the option |prover|.

  During the installation this path is augmented by the directory containing
  \ProTop{} and its files.

\item [prove:extension] |= ['','.pl']|\\
  This option contains a list of extensions used to find a reduction or a
  prover module. The strings or symbols contained in this list are appended to
  the file name before the existence of such a file is checked.

\item [prove:log\_items] |= [prover,matrix,contrapositives]|\\
  This option determines which information is written to the log file just
  after the first phase of the proving, i.e. the preprocessing. The value
  consists of a list of keywords as described below.

  \begin{description}
  \item[contrapositives]\ \\
    The list of contrapositives is written to the log file. The
    contrapositives are enclosed in |begin(contrapositives)| and
    |end(contrapositives)|. This environment contains log terms of the
    type |contrapositive/3|. The first argument is the head of the
    contrapositive. The second argument is the body, i.e. the list of literals
    without the head. The third argument contains the index of the
    contrapositive. 

    It is highly recommended to leave this item in the list |prove:log_items|
    since some postprocessing programs rely on it.
  \item[date]\ \\
    The current date is written to the log file. The date is stored in the log
    term |date/1|. The argument is a string containing the date in the format
    as given by the command |date/1|.
  \item[host]\ \\
    The host information is written to the log file. The host information is
    stored in the log term |host_info/2|. The first argument contains the
    hostname. The second argument contains the host architecture.
  \item[matrix]\ \\
    The list of clauses is written to the log file. The clauses are enclosed
    in |begin(matrix)| and |end(matrix)|. This environment contains log terms
    of the following types:
    \begin{description}
    \item[GoalClause/1] contains the indices of the goal clauses.
    \item[Label/2] contains the labels of the clauses. The first argument is a
      clause index and the second argument is its label.
    \item[Clause/2] contains the  clauses. The first argument is the index of
      this clause. The second argument is the list of literals. Each literal
      is of the form |literal(|$L$|,|$I$|)| where $L$\/ is a signed predicate
      and $I$\/ is its index.
    \end{description}

    It is highly recommended to leave this item in the list |prove:log_items|
    since some postprocessing programs rely on it.
  \item[options]\ \\
    The complete list of options and their values is written to the log file.
  \item[prover]\ \\
    The prover to be used is written to the log file. This information could
    also be extracted from the options, but sometimes this is the only option
    we are interested in.

    The prover name is stored in the log term |prover/1|.
  \item[user]\ \\
    The name of the user is written to the log file. The user information is
    stored in the log term |user/2|. The first argument is the login name of
    the user and the second name is the full name of the user as given in the
    GCOS field of the passwd file.
  \end{description}

\end{description}

\subsubsection{Options of the Reduction Module {\tt complete\_goals}}

\begin{description}

  \item [complete\_goals] | = on|\\
  If this option is turned on then the matrix is checked to contain a
  complete set of goals. If the set of goals is not complete then it is
  completed.

\end{description}

\subsubsection{Options of the Reduction Module {\tt connection\_graph}}

\begin{description}

  \item [find\_connections] | = on|\\
  This option can be used to suppress the generation of the reachability
  graph. \ProCom{} does not assume that this option is turned off. The
  result is unpredictable in this case.

  \item [find\_all\_connections] | = off|\\
  If this option is on then all literal are initially considered when
  constructing the reachability graph. Otherwise only literals in goal
  clauses are considered.

  \item [connect\_weak\_unifiable] | = on|\\
  One of two methods to compute the reachability graph can be selected.  The
  first method is to connect complementary literal which are weak
  unifyalbe. The alternative is to consider the predicate symbol only.

  \item [remove\_unreached\_clauses] | = on|\\
  If this option is on then each unreached clause is removed after the
  reachability graph has been constructed.

\end{description}


\subsection{Automatic Options}

This section contains options which are automatically set. They can be checked
but should not be altered in any way.

\begin{description}
  \item [equality] | = off|\\
  This boolean option is set when an equality predicate (|=/2|) is detected
  in the input problem.

  \item [setvar] | = off|\\
  This boolean option is set when an set variable (|:/2|) is detected in the
  input problem.

  \item [input\_file] | = ''|\\
  This option is automatically set to the current input file name as given
  by the user, i.e. without the automatically appended directory from the
  search path.

  \item [output\_file] | = ''|\\
  This option is set to the output file name as given by the user.

  \item [interactive] | = off|\\
  This option is set when the output is not redirected to a file. Thus an
  interactive prover can act accordingly.

\end{description}



\subsection{\ProCom\ General Options}

\begin{description}

  \item ['ProCom:dynamic\_reordering'] | = on|

  \item ['ProCom:ancestor\_pruning'] | = on|
    \\
    This option enables generation of code performing identical ancestor
    pruning. This means that a branch of the search tree is cut if two
    identical subgoals have been encountered. Only a minor variant may be
    implemented.

  \item ['ProCom::ancestor\_pruning'] | = 'prune.pl'|
    \\
    This option specifies the library which provides the predicates to perform
    the identical ancestor check.

  \item ['ProCom:occurs\_check'] | = on|
    \\
    This option indicates wheter special care should be taken to perform sound
    unification. In this case special caution is taken at the clause level.
    Another variant might be to enable the occurs check globally --- if this
    feature is provided by the Prolog dialect used.

  \item ['ProCom::unify'] | = 'unify.pl'|
    \\
    This option specifies which library should be used to get the predicate
    |unify/2| which performs sound unification. This library is used if the
    option  |ProCom:occurs_check| is |on|.

  \item ['ProCom::unify\_simple'] | = 'unify-no-oc.pl'|
    \\
    This option specifies which library should be used to get the predicate
    |unify/2| which performs unification. This library is used if the option
    |ProCom:occurs_check| is |off|.

  \item ['ProCom::member'] | = 'member.pl'|
    \\
    This option specifies which library should be used to get the predicate
    |member/2| which is a sound member predicate. Ususally this library will
    in turn use the library predicate |unify/2|. Thus the value of the option
    |ProCom:occurs_check| is taken into account.

  \item ['ProCom:literal\_selection'] | = deterministic|
    \\
    This option determines the strategy for literal selection. It can take the
    values |deterministic| or |random|\footnote{Currently not supported.}.

  \item ['ProCom:path'] | = path.pl|\label{opt:ProCom:path}
    \\
    This option determines the library containing the path managment
    predicates.  See section \ref{sec:lib.path} for details.

  \item ['ProCom::lemma'] | = 'no-lemma.pl'|
    \\
    This option specifies the library containing the lemma handling
    predicates.  The standard libraries contain also the library |lemma.pl|
    which is recommended as a basis for your own lemma handling routines.

  \item ['ProCom:lemma'] | = off|
    \\
    This option can be used to suppress the compiling of the calles to
    |lemma/3| which are defined in the lemma library.

  \item ['ProCom:automatic\_put\_on\_path] | = on |
    \\
    This option can be used to suppress the automatic addition of a goal to
    the current path. If youturn this option off you might want to use the
    primitive |put_on_path| in your descriptor to put some liteals on the
    path.

  \item ['ProCom:add\_contrapositives'] | = off|
    \\
    This option enables the generations of contrapositives in the target
    program. In this case the facts |contrapositive/3| are defined.

    The contrapositives may be needed by heuristic functions, e.g. for literal
    selection, which want to inspect the whole problem.

  \item ['ProCom:proof\_limit'] | = off|
    \\
  This option can be used to specify the way to proceed after a proof has been
  found and presented. The following values are currently supported:
  \begin{description}
    \item [interactive]\ 
      \\
      This value forces the generation of code to query the user. This is
      similar to the top level loop of Prolog. The library |ProCom::more| (see
      section \ref{lib:more}) is used in this case.
    \item [all]\ 
      \\
      This value forces the search for further solutions.
    \item [{\em number}]\ 
      \\
      Any natural number forces the search for further solutions until this
      number of solutions are found or all solutions are exhausted. The
      library specified by the option |ProCom::proof_limit| (see section
      \ref{lib:proof_limit}) is used in this case.
  \end{description}

  Any other value will force the termination after the first solution.

\end{description}

\subsection{\ProCom\ Goal Compiler Options}


\begin{description}
  \item ['ProCom:post\_goal\_list'] | = []|
    \\
    This option specifies a list of Prolog predictes which are called after
    the proof of a goal has been successful. As additional arguments the list
    of variable bindings and the proof tree is added to the predicate given.

    Consider the |ProCom:post_goal_list| containing the element |pgl(1,47)|.
    Then the goal |pgl(1,47,Vars,Proof)| is compiled into the prover, where
    |Vars| and |Proof| are instantiated to the variable bindings and the proof
    tree respectively.

    Note that you have to ensure that the predicates given are present in the
    compiled Prolog program. This can be done by specifying appropriate
    libraries containing the Prolog code.

  \item ['ProCom:init\_goal\_list'] | = []|
    \\
    This option specifies a list of Prolog predicates which are called when a
    goal is launched.

    Note that you have to ensure that the predicates given are present in the
    compiled Prolog program. This can be done by specifying appropriate
    libraries containing the Prolog code.

  \item ['ProCom:init\_level\_list'] | = []|
    \\
    This predicate specifies a list of predicates which are called whenever a
    new depth level is started, i.e. right after |set_depth_bound/1| has
    returned a new depth. The depth is added as additional argument to the
    predicates given.

    Note that you have to ensure that the predicates given are present in the
    compiled Prolog program. This can be done by specifying appropriate
    libraries containing the Prolog code.

\end{description}

See also the section on reordering (\ref{procom:reordering}) for further
options to influence the goal compilation.



\subsection{\ProCom\ Linker and Optimizer Options}

The options described in this section influence the behaviour of the linker.
The linker is the last step in the program generation process. Several tasks
are performed here. The main task is to add missing predicates from the
libraries. For this purpose possible libraries have to be named.

Another task perform in this last step is the application of certain
optimizations. The optimizations are mainly unfolding for Prolog predicates
for efficiency. These optimizations drastically reduce the readability of the
generated code. Thus it is recommened to turn them of when first trying to
understand the generated Prolog program.

\begin{description}

\item ['ProCom:link'] | = static|\\ 	This option specifies the type of
  linking. Possible values are
  \begin{description}
  \item [off]\ 
    \\
    This value disbales the linker. This is recommened for development
    purposes only. In this case the generated code is not executable as is.
  \item [static]\ 
    \\
    This is the usual way of linking. Any Prolog code from the libraries is
    copied into the generated Prolog program. Thus the program can be run on
    any appropriate Prolog system.
  \item [dynamic]\ 
    \\
    If this variant is used then for most of the libraries only compile
    instructions are generated in the target program. Thus this program is
    only executable if the used libraries are still accessible under the names
    given during linking.
  \end{description}

\item ['ProCom::link\_path']\label{opt:ProCom::link_path}\ 
  \\
  This option holds a list of directory names where the linker looks for
  appropriate files. The value of the option |prolog| is appended to those
  directories.

\item ['ProCom:optimize'] | = on|
  \\
  The linker has built-in the ability to perform certain optimizations.  For
  testing purposes it might be desirable to disable those optimizations.  In
  general it is not recommended to turn them off.

\item ['ProCom:expand'] | = on|\label{opt:ProCom:expand}
  \\
  This option enables the expansion (i.e. unfolding) of non-recursive
  predicates from the libraries. In the libraries one can specify which
  predicates should be considered (see section~\ref{sec:contents.library}).

\item ['ProCom:expand\_aux'] | = on|
  \\
  This option enables the expansion (i.e. unfolding) of auxiliary predicates.

\item ['ProCom:module'] | = off|
  \\
  This option enables the generation of code which encapsulates the prover in
  a module. This is only possible if a module system is provided by the Prolog
  dialect.

\item ['ProCom::module'] | = 'module.pl'|
  \\
  This option specifies the library which contains the module head.

\item ['ProCom::immediate\_link] | = ['init.pl']|%
  \label{opt:ProCom::immediate_link}\\
  This option specifies a list of libraries which are added to the code
  at the beginning --- possibly right after the module initialization but
  before anything else.

\item ['ProCom::post\_link] | = []|
  \\
  This option specifies a list of libraries which are added to the code at the
  end. This might be used to add libraries which are encapsulated in modules.

\item ['ProCom:ignore\_link\_errors'] | = off|
  \\
  This option can be used to suppress the linker to abort because of missing
  predicates. Usually the linker checks wether all predicates used are also
  defined. Those checks can fail if the predicate is defined in a library file
  or a module loaded with the |ProCom::post_link| option. Thus it can be
  desirable to ignore the linker errors. Nevertheless the error messages are
  displayed but ignored afterwards.
\end{description}

\subsection{\ProCom\ Information Controlling Options}

Information about the proof and the proof process can be of intrest. Thus it
is possible to enable the generation of such information. Since thiese
additional operations cost some time they may be worth turning off, when a
fast proof is required and the information is not essential.

\begin{description}

  \item ['ProCom:verbose'] | = off|
    \\
    This option enables the generation of various comments in the target
    Prolog program. The generated code my be slightly more readable when this
    option is on.

  \item ['ProCom:proof'] | = on|
    \\
    This option enables the generation of code to collect the information
    representing the proof tree. This takes some time and space and should be
    turned off when a fast prover is required. If a proof tree is enabled the
    predicates required are taken from the library given in the next option.

  \item ['ProCom::proof'] | = 'proof.pl'|
    \\
    This option specifies the library which is used for proof tree generation
    and display.

  \item ['ProCom:trace'	] | = off|
    \\
    This option enables code generation for the built-in debugger. This
    debugger can be used to trace the activities of the prover. (see
    \ref{sec:debugger})

  \item ['ProCom::trace'] | = 'debugger.pl'|
    \\
    This option specifies the library which contains the debugger.

  \item ['ProCom:timing'] | = on|
    \\
    This option enables the generation of code to determine the run time of
    the theorem prover. This is only possible in dialects which provide means
    to acess the run time.

  \item ['ProCom::timing'] | = 'time.pl'|
    \\
    This option specifies the library which contains code to set and query the
    timer.

  \item ['ProCom:show\_result'] | = on|
    \\
    This option enables the generation of code to display variable bindings
    and allow the user to ask for additional solutions.

  \item ['ProCom::show\_result'] | = 'show.pl'|
    \\
    This option specifies the library which contains code to display the goal
    and variable bindings after a succesful proof attempt.

\end{description}


\subsection{\ProCom\ Reordering}\label{procom:reordering}

\ProCom{} provides a powerful facility to reorder things before, during and
after the compilation process. This is an addition to the reordering hooks
provided in the preprocessing which are far less expressive.

In contrast to the |reorder_literals|/|clauses| functions which are assumed to
poke around in the Prolog database the \ProCom{} reordering facility provides
an interface which allows the user to specify an order and leave the
reordering to the system. This has the advantage that the user can do no harm
to the data, i.e. alter or delete entities.

The system performs the sorting using a user written comparison predicate. To
allow several sets of comparison predicates to coexist those predicates are
hidden in modules. The options specify the module name in which certain
predicates are expected. If no reordering is wanted then the empty symbol |''|
can be used instead.

The predicates given should succeed if the first element is less or equal than
the second element. These predicates should be deterministic.

\begin{description}
\iffalse
  \item ['ProCom:reorder\_ext']             | = ''|
    \\
    This option provides the name of a module containing a predicate
    |compare_ext/2|. This predicate is used to compare
  \item ['ProCom:reorder\_goal']            | = ''|
    \\
    This option provides the name of a module containing a predicate
    |compare_goal/2|. This predicate is used to compare
\fi
  \item ['ProCom:reorder\_clauses']         | = ''|
    \\
    This option provides the name of a module containing a predicate

    |compare_clauses/2|

    This predicate is used to compare clauses.
  \item ['ProCom:reorder\_goal\_clauses']   | = ''|
    \\
    This option provides the name of a module containing a predicate

    |compare_prolog_clauses/2|

    This predicate is used to compare
  \item ['ProCom:reorder\_prolog\_clauses'] | = ''|
    \\
    This option provides the name of a module containing a predicate

    |compare_prolog_clauses/2|

    This predicate is used to compare Prolog
    clauses of a procedure just after they have been generated and before the
    optimizer has done it's work. Depending on other options the clauses may
    have different forms. E.g. auxiliary predicates may be expanded.
  \item ['ProCom:reorder\_aux\_clauses']    | = ''|
    \\
    This option provides the name of a module containing a predicate

    |compare_aux_clauses/2|

    This predicate is used to compare auxiliary clauses.
\end{description}

