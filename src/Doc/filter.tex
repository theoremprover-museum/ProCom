%%%****************************************************************************
%%% $Id: filter.tex,v 1.3 1995/03/20 21:24:47 gerd Exp $
%%%============================================================================
%%% 
%%% This file is part of ProCom.
%%% It is distributed under the GNU General Public License.
%%% See the file COPYING for details.
%%% 
%%% (c) Copyright 1995 Gerd Neugebauer
%%% 
%%% Net: gerd@imn.th-leipzig.de
%%% 
%%%****************************************************************************

\chapter{Writing Filters}\label{chap:writing.filters}

This chapter describes how to write an input filter for \ProTop. A filter is a
simple \eclipse{} module which satisfies certain restrictions. The most simple
filter is delivered as the file {\sf Filters/none.pl}. This filter does
nothing but pass everything it reads to the next filter. It gives a general
structure for any special filter.

The first thing we have to take care of is the naming convention for filters.
Three entities which can be used independently are now linked together:
\begin{enumerate}
\item The name of the module. Let's call it {\em my\_filter} for the moment.
\item The name of the file containing the module. This has to be {\em
    my\_filter.pl}, i.e. the name of the module plus an extension of {\tt
    .pl}.
\item One predicate exported by {\em my\_filter}\/ has to be {\em
    my\_filter/A}, where {\em A}\/ is the arity of the filter. The first two
  arguments are occupied by the input and the output stream. Additional
  arguments can be required by the filter. Nevertheless it turned out to be a
  good practice to define the predicate {\em my\_filter/2} which provides
  reasonable defaults for the additional arguments.
\end{enumerate}

In the following we consider a filter without additional arguments, i.e.\ the
filter predicate has arity 2.  Thus the beginning of our filter {\em
  my\_filter} in the file {\em my\_filter.pl} looks as follows:

\begin{BoxedSample}\raggedright\tt
:- module\_interface({\em my\_filter}).
:- export {\em my\_filter}/2.
:- begin\_module({\em my\_filter}).
info(filter,"Vers. 0.1","my\_filter is a filter to do ...").
:- lib(op\_def).
\end{BoxedSample}

The info fact identifies this module to contain a filter. The first argument
is the atom |filter|. The second argument is a string containing a version
identification. The third argument contains a string with a {\em short}\/
description. 

The final instruction in the example above loads the library containing
definition of operators.  Especially the \verb|#| is defined there. The
meaning and use of this functor will be described later.

The predicate {\em my\_filter/2} takes as first argument an input stream and
as second argument an output stream. Opening and closing of these streams is
performed by \ProTop. All {\em my\_filter/2} has to do is to read from the
input stream and write the result to the output stream.

Since many Prolog terms may be waiting to be processed a loop has to be used.
The termination condition is the term \verb|end_of_file|. If this term is read
{\em my\_filter/2} should return with success without leaving a choice point.
{\em my\_filter/2} is not allowed to fail!

The non-failing condition can easily be satisfied by a failure driven loop. The
omission of a choice point is guaranteed by a terminating cut (which is good
practice for a failure driven loop anyhow).

\begin{BoxedSample}\raggedright\tt\obeyspaces
my\_filter(Stream,OutStream) :-
        repeat,
        read(Stream,Term),
        ( Term = end\_of\_file ->
            true
        ;   writeclause(OutStream,Term),
            fail
        ),
        !.
\end{BoxedSample}

This simple example can now be enhanced by replacing the |writeclause|
predicate by a more complicated construct which transforms the term read and
writes the result to |OutStream|.

The predicate |writeclause| is highly recommended since it is guaranteed to
produce output which can be read back into Prolog\footnote{Currently this is
not really true in \eclipse.}.

To allow certain filters to have private informations in the input we have
developed a discipline of programming filters. The |#| is assigned a special
meaning only in filters. |#| is declared as prefix operator.  The following
rules must be honored:

\begin{itemize}
  \item Any |#| instruction not understood by a filter must be passed
  to the next filter unchanged.

  \item The construction |#| {\em Option}|=|{\em Value}|.| is reserved for
  assignments of options.

  \item |#begin(|{\em section}|)| and |#end(|{\em section}|)| serve as
  grouping constructs. Grouping constructs for the same section can not be
  nested. Unknown groups must be passed unchanged to the output stream.
\end{itemize}





%\chapter{File Formats}

%\section{The Proof Tree Format}

%\section{The Report File Format}

