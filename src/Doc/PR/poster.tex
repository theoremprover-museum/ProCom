%******************************************************************************
%* Author: Gerd Neugebauer   
%******************************************************************************
%* $Id:$
%******************************************************************************

\documentstyle[epic,eepic]{slide}
\pagestyle{empty}

\setlength{\textwidth}{162mm}    \setlength{\textheight}{230mm}
\setlength{\parindent}{1Em}      \setlength{\parskip}{1ex}
\setlength{\topmargin}{0mm}
\setlength{\headheight}{3mm}     \setlength{\headsep}{8mm} 
\setlength{\footskip}{10mm} 
\setlength{\evensidemargin}{0mm} \setlength{\oddsidemargin}{0mm}

\newfont\bfLO{ocmbx12 scaled \magstep5}

\newsavebox\Sbox
\sbox{\Sbox}{\input{capri10.epic}}
\def\CPI{\usebox{\Sbox}}

\newsavebox\SBOX
\sbox{\SBOX}{\input{capri20.epic}}
\def\CAPRI{\usebox{\SBOX}}

\def\Head#1{\CPI\hfill{\bfLO #1}\hfill\CPI}
\begin{document} %%%%%%%%%%%%%%%%%%%%%%%%%%%%%%%%%%%%%%%%%%%%%%%%%%%%%%%%%%%%%%


\begin{Slide}
  {\bf Reif f\"ur die Insel?}
  \AD\AD
  \makebox(0,0){\unitlength=1mm\circle{50}}
  \makebox(0,0){%
    \begin{minipage}{.4\textwidth}
      \begin{center}
	{\bfLO ProCom\\[1.2ex]CaPrI}    
      \end{center}
    \end{minipage}}
  \AD\AD
  {\Large\bf Calculi Programming Interface}
  \rule{\textwidth}{2pt}
  \input{lines.latex}
  \vspace{-100pt}
  \CAPRI\hfill\CAPRI

\end{Slide}
%------------------------------------------------------------------------------
\begin{Slide}[\Head{ProCom}]
  \large\bf

  \SPACE
  Prolog-Technologie-\"Ubersetzung
  \SPACE
  \begin{Itemize}[.6\textwidth]
    \item auf Klauselebene
    \item auf Prozedurebene
    \item auf Goal-Ebene
  \end{Itemize}

  \SPACE
\end{Slide}
%------------------------------------------------------------------------------
\begin{Slide}[\Head{Framework}]
  \large\bf

  \SPACE
  Rahmen f\"ur Normalformbeweiser
  \SPACE
  \begin{Itemize}[.7\textwidth]
    \item Parsing
    \item Matrix-Management
    \item Optionen-Verwaltung
    \item Beweiser-Ansteuerung
  \end{Itemize}

  \SPACE
\end{Slide}
%------------------------------------------------------------------------------
\begin{Slide}[\Head{CaPrI}]
  \large\bf

  \SPACE
  Calculi Programming Interface
  \SPACE
  \begin{Itemize}[.6\textwidth]
    \item Spezifikation der \"Ubersetzung
    \item Anforderungen spezifizierbar
    \item Interaktion mit ProCom
  \end{Itemize}

  \SPACE
\end{Slide}
%------------------------------------------------------------------------------
\begin{Slide}[\Head{CaPrI --- Beispiel(1)}]
  \large\bf

  {\Large\bf Extensionsverfahren} (1)
  \SPACE
\begin{minipage}{.9\textwidth}
\begin{verbatim}
:- module(me1).
:- compile(library(capri)).

require_option(equality,[off]).

descriptor
    template(Pred,goal),
    template(-Pred,
             [path,extension]).
\end{verbatim}
\end{minipage}

\end{Slide}
%------------------------------------------------------------------------------
\begin{Slide}[\Head{CaPrI --- Beispiel(2)}]
  \large\bf

\begin{minipage}{.9\textwidth}
\begin{verbatim}
:- module(me2).
:- compile(library(capri)).

require_option(equality,[off]).

descriptor
    proof(reduction(Pred)),
    template(Pred,goal),
    template(-Pred,path).

descriptor
    proof(extension(-Pred)),
    template(Pred,goal),
    template(-Pred,extension).
\end{verbatim}
\end{minipage}

\end{Slide}
%------------------------------------------------------------------------------
\begin{Slide}[\Head{CaPrI --- Beispiel(3)}]
  \large\bf
  \SPACE
\begin{minipage}{\textwidth}
\begin{verbatim}
:- module(me3).
:- compile(library(capri)).

require_option(equality,[off]).
force_option(
    'ProCom:automatic_put_on_path',
    [off]).

descriptor
    proof(reduction(Pred)),
    template(Pred,goal),
    template(-Pred,path).

descriptor
    proof(extension(-Pred)),
    template(Pred,goal),
    put_on_path(Pred),
    template(-Pred,extension).
\end{verbatim}
\end{minipage}

\end{Slide}
%------------------------------------------------------------------------------
\begin{Slide}[\Head{CaPrI}]
  \large\bf

  \SPACE
  Kalk\"ule
  \SPACE
  \begin{Itemize}[.8\textwidth]
    \item Extensionsverfahren
    \item --- mit Paramodulation
    \item --- mit S-Modifikation
    \item --- mit E-Modifikation
    \item Restart-ME mit rigid-E
  \end{Itemize}

  \SPACE
\end{Slide}
%------------------------------------------------------------------------------



\end{document} %%%%%%%%%%%%%%%%%%%%%%%%%%%%%%%%%%%%%%%%%%%%%%%%%%%%%%%%%%%%%%%%
