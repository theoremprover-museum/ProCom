%%%****************************************************************************
%%% $Id: search.tex,v 1.4 1995/01/16 21:03:12 gerd Exp $
%%%============================================================================
%%% 
%%% This file is part of ProCom.
%%% It is distributed under the GNU General Public License.
%%% See the file COPYING for details.
%%% 
%%% (c) Copyright 1995 Gerd Neugebauer
%%% 
%%% Net: gerd@imn.th-leipzig.de
%%% 
%%%****************************************************************************

%------------------------------------------------------------------------------
\section{Implementing Search Strategies}


The selection of the search strategy for \ProCom{} can be done with the option
|search|. When  a  search strategy is   selected this way this  influences the
execution  of  an initialization predicate.   This initialization predicate is
called

\Predicate init\_search/0().

This predicate is used from the  module |p__|{\em search}\/ where {\em search}
is the functor of the search strategy.

Let us consider an example. Suppose the option |search| is
|iterative_deepening(1,2,3)|.  This setting is evaluated at the time when
\ProCom{} is compiling a matrix.  The functor of this option is
|iterative_deepening|.  Thus the predicate |init_search/0| is called in the
module with the prefix |p__| added, i.e.\ |p__iterative_deepening|.

This predicate has reading and writing access to all options and some internal
global variables which can be used to adjust the compilation process
accordingly.

\begin{description}
\item [Hint:] Since the arity and the arguments of the option |search| are not
  taken  into account in any way,  the search strategy can  use them for their
  own purposes.  Especially a scheme can  be implemented to use default values
  for  missing arguments,  or issue  an  warning message  when the appropriate
  values are not given.
\end{description}

Each search strategy is declared with  the predicate |define_search/1| in {\sf
  procom.cfg}. This  configuration  file is automatically  generated  from the
{\sf Makefile}.

A checklist for adding a new search strategy is given below.
\begin{itemize}
\item  Write   a   module  |p__|{\em search}     which exports  the  predicate
  |init_search/0|. Place  this  module file  on  the  search  path  of Prolog.

\item  Modify the variable |SUBDIRS| in {\sf Makefile}.

\item Declare the search strategy in {\sf Makefile}. For this purpose the
  value of the Makefile variable |PROCOM_SEARCH| must be adapted.

\item Recompile the system.
\end{itemize}
