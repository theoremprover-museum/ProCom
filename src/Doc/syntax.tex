%%%****************************************************************************
%%% $Id: syntax.tex,v 1.2 1995/01/27 13:45:38 gerd Exp $
%%%============================================================================
%%% 
%%% This file is part of ProCom.
%%% It is distributed under the GNU General Public License.
%%% See the file COPYING for details.
%%% 
%%% (c) Copyright 1995 Gerd Neugebauer
%%% 
%%% Net: gerd@imn.th-leipzig.de
%%% 
%%%****************************************************************************

\def\SYNTAX#1.{\begin{itemize}\item[]\(#1\) \end{itemize}}

%------------------------------------------------------------------------------
\section{The Input Language}

\subsection{Basics}

\ProCom{} is  currently able to process problems  in  clausal form. The chosen
syntax is a superset of pure  Prolog, i.e. without  built-in predicates of any
kind.  Since  the  Prolog  reading  apparatus  is  used  several features  are
inherited directly:
\begin{description}
  \item [Variables\index{variable}] are written with an initial capital letter.

  \item [Functors\index{functor}  and constants\index{constant}] usually start
    with  a lower  case letter.  Additionally  some  predefined  operators are
    defined as infix, e.g. |=|, |<|, |>|, and others.

  \item [Terms\index{term}] are built from functors and variables as usual (in
    Prolog).

  \item [Comments\index{comment}] are anything starting with a |%| up to the
    end  of the line.  Additionally  C-style comments, i.e. starting with |/*|
    and ending in |*/|, can be used.
\end{description}


\subsection{Literals}
\index{literal}

A  literal  is a predicate  with  an optional sign.   Everything  which has an
operator not  treated special is considered  as a predicate. Special operators
are |,|, |;|, |:-|, |<-|, |->|, |?-|, |::|, the list constructor and the empty
list constant |[]|.

A positive literal can  be optionally marked with  the prefix |++|. Internally
the positive sign  is added where required. The  negation sign is |-| or |--|.
Internally |--| is used only.


\subsection{Clauses}
\index{clause}

\SYNTAX |[| L_1 |,| \ldots|,| L_n |]|.
%
The simplest  --- but  hard   to read ---  form  of  a  clause consists of   a
(nonempty) list of literals. They are stored as they are.


\SYNTAX H_1|;|\ldots|;|H_n \,|:-|\, T_1|,|\ldots|,|T_m.
%
This is the general  form of a clause  in  extended Prolog notation.  The head
literals $H_1,\ldots  H_n$  denote  the   negative  literals. Thus  they   are
implicitly  negated when   the internal list  representation is  constructed.
Nevertheless explicit negation using the |-| operator is permitted.

Other degenerate variants --- the head or the tail are  empty --- are provided
as well.


\SYNTAX |:-|\, T_1|,|\ldots|,|T_m.
%
This rule parses the degenerate form of the extended Prolog notation where the
head is empty. Note that  in contrast to Prolog there  is a difference to  the
|?-| operator.


\SYNTAX  H_1|;|\ldots|;|H_n \,|<-|\, T_1|,|\ldots|,|T_m.
%
The forms incorporating the |<-| operator are simply syntactic variants of the
same forms using the |:-| operator.


\SYNTAX |<-|\, T_1|,|\ldots|,|T_m.
%
This form is simply mapped to the corresponding |:-| variant.


\SYNTAX |?-| Goal.
%


\SYNTAX L_1 |;|\ldots|;| L_n.
\SYNTAX L_1 |,|\ldots|,| L_n.
%
A disjunction  is interpreted as  a degenerated left  hand side of an extended
Prolog clause. Thus it is negated while translated into a list.


\subsection{Labels}
\index{label}

\SYNTAX Label |::| Clause.
%
Any clause can have arbitrary information assigned to it.  This is done in two
forms. One form is to mark a clause as goal clause with the |?-| operator. The
more general form implemented here is to write a label  in front of the clause
separated  by the |::| operator.   The label {\em Label}\/  is  stored in  the
Prolog database as |'Label'(|{\em Label}|,|{\em Index}|)|.

