%%%****************************************************************************
%%% $Id: libraries.tex,v 1.5 1995/02/13 20:05:14 gerd Exp $
%%%============================================================================
%%% 
%%% This file is part of ProCom.
%%% It is distributed under the GNU General Public License.
%%% See the file COPYING for details.
%%% 
%%% (c) Copyright 1995 Gerd Neugebauer
%%% 
%%% Net: gerd@intellektik.informatik.th-darmstadt.de
%%% 
%%%****************************************************************************
% Master File: manual.tex

\section{Libraries}

The next stage of modification beyond the adaption of option is to replace
libraries by own code. For this purpose we will describe in detail the
libraries used. Thus it is possible to write own libraries for new Prolog
dialects. Additionally it also enables you to replace the libraries provided
with \ProCom{} by your own versions which have an improved performance or an
enhance functionality.

\paragraph{Note:} Don't modify the libraries provided with \ProCom. Make
modified versions with a different name instead and set the appropriate option
to tell \ProCom\ to use this modified version. It is also possible to exploit
the search mechanism for libraries (see \ref{sec:lib.search}) to place the
modified version in a directory which is searched before the system library.

In the following sections we will describe the libraries and the predicates
they are expected to provide.


\subsection{Library Search}\label{sec:lib.search}
\index{library search}


At the end of the compilation process the linker tries to add libraries to the
code which provide required predicates. For this purpose the linker needs to
know which predicates are required and where to find libraries which might
provide them. Requirements of predicates are specified by the implementor, in
the Capri interpreter, in the Capri modules, or in the libraries.

When the linker is initialized it analyzes the files given to it to find out
which predicates are defined there. Usually the files known to the linker are
initially taken from several option (those containing |::|). The instruction
|link_file/1| in a Capri module can be used to provide additional files.

The link files are searched in the following way. The option
|ProCom:link_path| (see section \ref{opt:ProCom::link_path}) contains a list
of directories which are used to find the libraries. Two subdirectories of the
directories given are also considered. The first subdirectory is named like
the Prolog dialect specified by the option |prolog| (see section
\ref{opt:prolog}). The second subdirectory is called {\sf default}.

The general idea is to place generic code which should run on each Prolog
system in the subdirectoy {\sf default} and the dialect specific modules in
the specific subdirectories, e.g. {\sf eclipse}.

In addition to the intermediate directories the libraries are augmented by the
extension |.pl| during the search. To make it clear which files are considered
let us have a look at an example.

Suppose the option |prolog| has the value |eclipse| and the option
|ProCom:link_path| has the value |[.,ProCom]|. When the linker looks for a
library named |my_lib| the following locations are inspected until an existing
file is found:

{\tt
\begin{tabular}{l}
  ./my\_lib
  \\./my\_lib.pl
  \\./eclipse/my\_lib
  \\./eclipse/my\_lib.pl
  \\./default/my\_lib
  \\./default/my\_lib.pl
  \\ProCom/my\_lib
  \\ProCom/my\_lib.pl
  \\ProCom/eclipse/my\_lib
  \\ProCom/eclipse/my\_lib.pl
  \\ProCom/default/my\_lib
  \\ProCom/default/my\_lib.pl
\end{tabular}
}


\subsection{Contents of Library Files}\label{sec:contents.library}
\index{library file}

Library files are mainly usual Prolog files --- with some exceptions.

You should be very carefully when making a module in a library. Be sure that
you completely understand the translation process and can predict the
resulting Prolog code.

Since the library file is read by the linker each operator declaration must be
known to the linker. Operator declarations in the library are not evaluated.

Most of the instructions are simply passed to the output file. Nevertheless the
linker uses some instructions to control its behavior. The folloing list
describes instructions evaluated by the linker. Each such instruction is
embedded in a single Prolog goal. Accumulating several of them or embedding in
other constructs won't work.

In the following list the expression {\em Pred} always denotes a predicate
specification in the form {\em Functor/Arity}.
\begin{description}
  \item [:- expand\_predicate({\em Pred})]\index{expand\_predicate}\ \\
  This instruction is passed to the optimizer to declare the predicate {\em
  Pred} as expandable. If the option |ProCom:expand| (see section
  \ref{opt:ProCom:expand}) is turned on them expandable predicates are
  replaced by their bodies.

  You should only declare non-recursive, single-clause predicates as
  expandable\footnote{A future version of the linker may determine those
  predicates automatically.}.  Recursive predicates can not be expanded
  properly.  Don't try it. Multi-clause predicates may also cause problems.

  The declaration of expandable predicates must preceed their definition!

  \item [:- provide\_predicate({\em Pred})]\index{provide\_predicate}\ \\
  This instruction tells the linker that the predicate {\em Pred} is defined
  in this library. This should usually be not necessary since the linker
  analyzes the Prolog program to see which predicates are defined.
  Nevertheless it can be necessary to use it to declare a Prolog built-in
  which is required (see below).

  \item [:- require\_predicate({\em Pred})]\index{require\_predicate}\ \\
  This instruction tells the linker that the predicate {\em Pred} is used in
  this library but not defined. The linker makes not a complete analysis of
  used and defined predicates. Instead it evaluates this instruction to get
  the undefined predicates. This can be used to trigger the loading of other
  libraries.
\end{description}


\subsection{The Init Library}\label{sec:lib.init}

For some reasons it may be desirable to add some pieces of code in front of
anything else. This can be used to adjust the behavior of the Prolog system.
A list of libraries is taken from the option |ProCom::immediate_link| to be
linked at the beginning (see section \ref{opt:ProCom::immediate_link}).

The following sample implementations are provided with \ProCom.
\begin{description}
  \item [eclipse/init.pl]\ \\
  This library turns off some debugging features to speed things up.
  Additionally the sound unification is turned on.
  \item [default/init.pl]\ \\
  This library is simply empty. No initializations are needed for a vanilla
  Prolog.
\end{description}


\subsection{The Module Library}\label{sec:lib.module}

The resulting Prolog code can be encapsulated into a module. The head of a
module with a fixed name and export list can be placed in a library named in
the option |ProCom::module|.

The following sample implementation is provided with \ProCom.
\begin{description}
  \item [eclipse/module.pl]\ \\
  This library contains the head of a \eclipse\ module named |'ProCom prover'|
  which exports the predicates |goal/0| and |goal/1|.
\end{description}

\paragraph{Note:} Most of the time it is not desirable to generate a stand
alone module since \ProTop\ wrappes a module around the prover anyway.

\subsection{The Path Management}\label{sec:lib.path}

The data sturcture used for the path is completely encapsulated in a library.
The option |ProCom:path| (see section \ref{opt:ProCom:path}) determines the
library actually used. The following sample implementations are provided with
\ProCom.
\begin{description}
  \item [default/path.pl]
  \item [default/path-simple.pl]
  \item [eclipse/path-regular.pl]\ \\
  This library uses \eclipse\ delayed predicates to implement a variant of
  regularity constraints.
\end{description}

The current implementations use a pair of lists to implement the path. The
first item is the list of positive ancestors and the second one contains the
negative ancestors.  Since the path management is encapsulated in this module
one can replace this kind of path by another data structure. Linear lists are
a simpler case. But also balanced binary trees can be considered. Any
implementation which allows a fast access, e.g. by using sofisticated indexing
mechanisms, might be worth trying it.

The path management consists of the following set of predicates:
\begin{description}
  \item [empty\_path({\em Path})]\index{empty\_path}\ \\
  This predicate should unify its argument with the term representing the
  empty path. It should have at most one solution.

  \item [put\_on\_path({\em Literal}, {\em OldPath}, {\em
  NewPath})]\index{put\_on\_path}\ \\
  This predicate modifies the path {\em OldPath} in such a way that the
  literal {\em Literal} is in it. The result is unified with the new path {\em
  NewPath}.

  \item [put\_on\_path({\em Literal}, {\em Info}, {\em OldPath}, {\em NewPath})]\index{put\_on\_path}

  \item [is\_on\_path({\em Literal}, {\em Path})]\index{is\_on\_path}\ \\
  This predicate tries to unify {\em Literal} with an element on the path {\em
  Path}. Upon backtracking all such candidates are tried.

  \item [is\_on\_path({\em Literal}, {\em Info}, {\em
  Path})]\index{is\_on\_path}

  \item [is\_identical\_on\_path({\em Literal}, {\em
  Path})]\index{is\_identical\_on\_path}\ \\
  This predicate tries to find {\em Literal} on the path {\em Path}. Only
  identical (not only unifyable) occurences are considered. This predicate
  should not be resatisfiable.
\end{description}


\subsection{Sound Unification}

The following sample implementations are provided with \ProCom.
\begin{description}
  \item [eclipse/unify.pl]\ 
    \\
    This library provides a sound |unify/2| predicate.
  \item [default/unify-no-oc.pl]\ 
    \\
    This library provides a |unify/2| predicate not performing the occurs
    check.
\end{description}

The following predicate is provided by those modules:

\begin{description}
  \item [unify({\em Term1}, {\em Term2})]\index{unify}\ 
    \\
    This predicate unifies its arguments.
\end{description}



\subsection{Sound Member Implementation}\label{sec:member}

The following sample implementation is provided with \ProCom.
\begin{description}
  \item [default/member.pl]\ 
    \\
    This library uses the unify/2 predicate to perform unification. Thus it
    depends on the unify library which unification is actually used.
\end{description}

\begin{description}
  \item [sound\_member({\em Element}, {\em List})]\index{sound\_member}\ 
    \\
    This predicate implements the well known member relation. Modification may
    be necessary to use the occurs check when required.
\end{description}


\subsection{Lemmas}\label{sec:lib.lemma}

After the sucessful attempt to solve a subgoal the result can be stored as a
lemma. Thus there exists a lemma library where this feature can be hooked in.
The following sample implementations are provided with \ProCom.
\begin{description}
  \item [default/no-lemma.pl]\ \\
    This library performs no lemma steps. It is just a dummy library to be
    used if nothing else is desirable.
  \item [default/lemma.pl]\ \\
    This library uses the path to store local lemmata. Thus no special
    \CaPrI{} descriptor set is needed to apply the lemmas. They are applied in
    the course of normal reduction steps.
\end{description}

The lemma library has to provide the following predicate:
\begin{description}
  \item [lemma({\em Literal}, {\em PathIn}, {\em PathOut})]\index{lemma}\ \\
    {\em Literal}\/ is the internal representation of the negated 
    solved literal. {\em PathIn}\/ is the current path. It has to be returned
    in {\em PathOut}, possibly enhanced by some additional literals. E.g. the
    predicate |put_on_path/3| might be useful (cf.\
    section~\ref{sec:lib.path}).
\end{description}


\subsection{Literals}\label{sec:lib.literal}

The literals are packed into a predicate to allow manipulations of literals at
link time. Usually a literal is just left alone.
The following sample implementations are provided with \ProCom.
\begin{description}
  \item [default/literal.pl]\ \\
    This library performs nothing. It is just a dummy library to be
    used if nothing else is desirable.
  \item [eclipse/literal\_static.pl]\ \\
    This library restricts the number of solutions of propositional goals to
    one. The test for free variables is performed at compile time (link
    time). Thus it may not detect goals which are propositional after some
    variables have been bound at run time.
  \item [eclipse/literal\_dynamic.pl]\ \\
    This library restricts the number of solutions of propositional goals to
    one. The test for free variables is performed at run time. Thus it may
    cause some additional overhead.
\end{description}

The literal library has to provide the following predicate:
\begin{description}
  \item [literal\_wrapper({\em Literal}, {\em Varlist})]\index{literal\_wrapper}\ \\
    {\em Literal}\/ is the goal representation of the literal to be
    solved. {\em Varlist}\/ is the list of variables ocurring in the literal
    at compile time.
\end{description}


\subsection{Timing}\label{sec:timing}

The time library provides means to access and measure the time elapsed. Since
the plain Prolog does not seem to have any means to perform this task it is
highly dialect specific how this can be done.

The following sample implementation is provided with \ProCom.
\begin{description}
  \item [eclipse/time.pl]\ 
    \\
    This library provides the predicates to manipulate the time for the
    \eclipse{} system.
  \item [eclipse/time.pl]\ 
    \\
    This library provides the dummy predicates for timing. Since there is no
    standard for those predicates they are simply mapped to do nothing.
\end{description}

The following predicates are provided by the time library:

\begin{description}
  \item [set\_time]\index{set\_time}\ 
    \\
    This predicate is used to initialize the system timer.
  \item [reset\_print\_time(File)]\index{reset\_print\_time}\ 
    \\
    This predicate is used to determine the ammount of time elapsed since the
    last call to a timer routine. This time is printed to the standard output.
    If {\em File} is an atom and not |[]| then the time is also appended to
    this log file.
  \item [set\_timeout({\em Limit})]\index{set\_timeout}\ 
    \\
    This predicate arranges things that the execution is interrupted after
    {\em Limit} seconds. If the Prolog dialect has no such capability it
    simply succeeds.
  \item [reset\_timeout]\index{reset\_timeout}\ 
    \\
    This predicate stops the timeout timer or does nothing.
\end{description}


\subsection{Log Files and Proof Presentation}\label{lib:proof}

The following sample implementation is provided with \ProCom.
\begin{description}
  \item [eclipse/proof.pl]\ 
\end{description}

\begin{description}
  \item [save\_bindings({\em Index}, {\em Clause}, {\em Bindings}, {\em
  File})]\index{save\_bindings}\ \\
  This predicate writes the bindings to the log file {\em File}. If {\em
  File} is the empty list |[]| or not an atom then this predicate simply
  succeeds.

  \item [save\_proof({\em Proof}, {\em File})]\index{save\_proof}\ \\
  This predicate saves the proof term {\em Proof}\/ in the log file {\em
  File}. If {\em File}\/ is the empty list |[]| or not an atom then this
  predicate simply succeeds.

  \item [show\_proof({\em Proof})]\index{show\_proof}\ \\
  This predicate displays the proof term {\em Proof} on the standard output.
\end{description}


\paragraph{more.pl}\label{lib:more}

\begin{description}
  \item [more({\em Depth})]\ \\
  This predicate queries the user to continue the search for a next solution
  or to abort. {\em Depth}\/ contains the current depth limit or is an unbound
  variable if no depth limit is used. This predicate succeeds iff {\em no}\/
  more solutions are required.
\end{description}


\paragraph{show.pl}\label{lib:show}

\begin{description}
  \item [list\_bindings({\em GoalIndex},{\em GoalClause},{\em Bindings})]\ \\
    This predicate is called after a solution has been found. It is meant to
    present the solution.
  \item [no\_more\_solutions]\ \\
    This predicate is called when the search tree has been exhausted and no
    more solutions can be found.
  \item [no\_goal]\ \\
    This predicate is called when no goal clause is left.
\end{description}


\paragraph{proof\_limit.pl}\label{lib:proof_limit}

\begin{description}
  \item [init\_proof\_limit]\index{init\_proof\_limit}\ \\
  This predicate initializes the number of proofs already found to 0. This can
  be done by asserting a fact to the Prolog data base or any other method
  which allows the value to survive backtracking.

  \item [check\_proof\_limit({\em Limit})]\index{check\_proof\_limit}\ \\
  This predicate increments the number of proofs already found and compares it
  with {\em Limit}. This predicate succeeds iff the number of proofs found is
  greater or equal to the {\em Limit}.

  \item [get\_proof\_limit({\em Limit})]\index{get\_proof\_limit}\ \\
  This predicate unifies the number of proofs already found with {\em Limit}.
\end{description}




\subsection{The Search Libraries}\label{lib:search}


\begin{description}
  \item [set\_depth\_bound({\em Start}, {\em Factor}, {\em Const}, {\em
  Depth})]\index{set\_depth\_bound}\ \\
  This predicate 

  \item [check\_depth\_bound({\em Depth}, {\em NewDepth}, \_,
  \_)]\index{check\_depth\_bound}\ \\
  This predicate 

  \item [show\_depth({\em Depth})]\index{show\_depth}\ \\
  This predicate 

  \item [choose\_step({\em Step}, {\em Cand}, \_, \_)]\index{choose\_step}\ \\
  This predicate 
\end{description}


\subsection{The Debugger}\label{sec:debugger}



