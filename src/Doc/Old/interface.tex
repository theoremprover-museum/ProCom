%%%****************************************************************************
%%% $Id: interface.tex,v 1.1 1994/10/18 18:42:27 gerd Exp $
%%%============================================================================
%%% 
%%% This file is part of ProCom.
%%% It is distributed under the GNU General Public License.
%%% See the file COPYING for details.
%%% 
%%% (c) Copyright 1994 Gerd Neugebauer
%%% 
%%% Net: gerd@imn.th-leipzig.de
%%% 
%%%****************************************************************************
% Master File: use-capri.tex

%------------------------------------------------------------------------------
\section{Getting Started}

In this section we assume that \ProCom{} has been installed as described in
section \ref{sec:installation} and \ProCom{} is accessible as the command
|procom|.

In this section we will see how \ProCom{} can be used as a stand-alone theorem
prover. 

\subsection{The TTY and X11 Interfaces}

The tty based interface has evolved to the \ProTop\ shell. It is described in
a separate documentation.

The X11 Interface is under construction. The documentation will be completed
when the interface is stable.


\subsection{The Prolog-Interface}

In a testing phase it can be desirable to avoid the generation of a
stand-alone version of \ProCom{} as described in section \ref{sec:recompile}.
Thus we will describe shortly how to proceed.

We assume that you are in the \ProCom{} installation directory {\sf ProCom}
and \ProCom{} has been installed properly as described in section
\ref{sec:installation}.

|% xeclipse|

|?- [main]|

|?- run(|{\em InFile}|, |{\em OutFile}|).|

|?- compile(|{\em OutFile}|).|

|?- goal.|


Alternately you can use the following scheme:


|% xeclipse|

|?- [main]|

|?- prove(|{\em InFile}|).|