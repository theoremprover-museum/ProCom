%%%****************************************************************************
%%% $Id: use-capri.tex,v 1.1 1994/12/02 08:58:35 gerd Exp $
%%%============================================================================
%%% 
%%% This file is part of ProCom.
%%% It is distributed under the GNU General Public License.
%%% See the file COPYING for details.
%%% 
%%% (c) Copyright 1994 Gerd Neugebauer
%%% 
%%% Net: gerd@imn.th-leipzig.de
%%% 
%%%****************************************************************************
\def\RCSrevision{\RCSstrip$Revision: 1.1 $}
\def\RCSdate{\RCSstrip$Date: 1994/12/02 08:58:35 $}

\documentstyle [11pt,%.......... 
		fleqn,%......... smash mathematics leftward
		floatfig,%...... 
		psfig,%......... 
		verbatim,%...... use input of files
		pcode,%......... obviously Prolog code
		gntitle,%....... 
		old-fonts,%..... 
		makeidx,%....... yes, we are making an index
		named,%......... 
		rcs,%........... 
		psulhead,%...... use underlined heading
		idtt,%.......... use | | for verbatim
		epic%...........
		]{book}%..... 
\pagestyle{ulheadings}

\setlength{\textwidth}{160mm}    \setlength{\textheight}{230mm}
\setlength{\parindent}{0Em}      \setlength{\parskip}{1ex}
\setlength{\topmargin}{0mm}
\setlength{\headheight}{6mm}
\setlength{\headsep}{8mm} 
\setlength{\footskip}{8mm} 
\setlength{\evensidemargin}{0mm} \setlength{\oddsidemargin}{0mm}


\author{Gerd Neugebauer}
\title{\ProCom/CaPrI \\\rlap{Calculi Programming Interface}}
\subtitle{User's Guide}
\version{\Version}
\edition{\RCSrevision}
\date{\today}
\address{IMN, HTWK Leipzig
\\	 Postfach 66
\\	 04251 Leipzig (Germany)
\\	 Net: gerd@imn.th-leipzig.de
}
%% ATTENTION: This file has been generated automatically.
%%            Changes to this file may be overwritten.
%%            Consult INSTALL and Makefile for details.
%%
%% Last created: Thu Jul 6 11:04:26 MET DST 1995
%%      by user: gerd
%%      on host: upsilon
%%
%----------------------------------------------------------
\gdef\Version{1.69}

\gdef\ProTopFilters{%{\catcode`\_=12%
  \Filter{tptp}%
  \Filter{mult_taut_filter}%
  \Filter{mpp}%
  \Filter{tee}%
  \Filter{equality_axioms}%
  \Filter{E_flatten}%
  \Filter{constraints}%
}%}

\gdef\ProTopProvers{%{\catcode`\_=12%
  \Prover{procom}%
  \Prover{pool}%
  \Prover{otter}%
  \Prover{setheo}%
}%}



\newenvironment{Sample}{%
	\begin{center}
	  \begin{minipage}{.5\textwidth}
	    \rule{\textwidth}{.1pt}\vspace{-1ex}%
	    }{%
	    \vspace{-2ex}\rule{\textwidth}{.1pt}
	  \end{minipage}
	\end{center}%
	}

\newcommand\ProTop{ProTop}
\newcommand\ProCom{ProCom}
\newcommand\eclipse{ECL$^i$PS$^e$}

\let\PrologFILE=\relax
\let\PredicateFileExtension=\relax

\makeindex
\begin{document} %%%%%%%%%%%%%%%%%%%%%%%%%%%%%%%%%%%%%%%%%%%%%%%%%%%%%%%%%%%%%%
\initfloatingfigs

\maketitle
\null\vfill
Copyright {\copyright} 1994 Gerd Neugebauer
\medskip

Permission is granted to make and distribute verbatim copies of this manual
provided the copyright notice and this permission notice are preserved on all
copies.

\newpage %---------------------------------------------------------------------
\tableofcontents
\newpage %---------------------------------------------------------------------

\chapter{Using \ProCom}

%------------------------------------------------------------------------------
\section{Introduction}

\begin{floatingfigure}{.4\textwidth}
  \begin{center}
    \mbox{\psfig{file=overview.eps}}
    \caption{Overview}\label{overview}
  \end{center}
\end{floatingfigure}

\ProCom{} is a family of theorem provers which can be used out of the box as
well as a test bed for developing own theorem provers. The techniques uses in
\ProCom{} are close to PTTP-like theorem provers \cite{stickel:prolog}. We
will not assume any knowledge on these techniques but it might help
understanding some of the details.

Implementing this paradigm means to translate a given problem into a Prolog
program which behaves like a theorem prover. A overview of this idea can be
seen in figure~\ref{overview}.

The first phase of a proof attempt takes a problem --- typically in the
language of first order logic --- and translates it into Prolog clauses. This
task is performed by the \ProCom{} system. The result is a Prolog program.
This Prolog program may contain specific constructs of the target Prolog
dialect.\footnote{Currently only \eclipse\ and Quintus Prolog are supported.}

Even so it is not necessary we will sketch the way \ProCom{} works.  After the
problem has been read and stored in matrix form, a preprocessor is applied
which performs several reductions. The reduced matrix is translated into
Prolog code in four phases. One phase takes the clauses of the matrix,
clusters them in procedures and generates code for those procedures. A second
phase translates the set of goals. Goals deserve a special treatment which is
different from usual procedurs. A third phase is adds Prolog code from a
library. The final phase consists of an optimizer which tries to remove
redundencies from the generated Prolog code.

\newpage
Having in mind this scheme simply using \ProCom{} goes as follows. First, a
problem is specified. This means it is written into a file in an extended
Prolog notation. Another file containing various options to controll the
behaviour of \ProCom{} may be used. Those input files are given to \ProCom.
The result is a stand-alone Prolog program. This Prolog program can be
executed. 

As a minimal result the answer {\tt yes} or {\tt no} is given (if any).
Additional the presentation of a answer substitution or even a complete proof
tree can be achieved.

%------------------------------------------------------------------------------
%%%****************************************************************************
%%% $Id: interface.tex,v 1.1 1994/10/18 18:42:27 gerd Exp $
%%%============================================================================
%%% 
%%% This file is part of ProCom.
%%% It is distributed under the GNU General Public License.
%%% See the file COPYING for details.
%%% 
%%% (c) Copyright 1994 Gerd Neugebauer
%%% 
%%% Net: gerd@imn.th-leipzig.de
%%% 
%%%****************************************************************************
% Master File: use-capri.tex

%------------------------------------------------------------------------------
\section{Getting Started}

In this section we assume that \ProCom{} has been installed as described in
section \ref{sec:installation} and \ProCom{} is accessible as the command
|procom|.

In this section we will see how \ProCom{} can be used as a stand-alone theorem
prover. 

\subsection{The TTY and X11 Interfaces}

The tty based interface has evolved to the \ProTop\ shell. It is described in
a separate documentation.

The X11 Interface is under construction. The documentation will be completed
when the interface is stable.


\subsection{The Prolog-Interface}

In a testing phase it can be desirable to avoid the generation of a
stand-alone version of \ProCom{} as described in section \ref{sec:recompile}.
Thus we will describe shortly how to proceed.

We assume that you are in the \ProCom{} installation directory {\sf ProCom}
and \ProCom{} has been installed properly as described in section
\ref{sec:installation}.

|% xeclipse|

|?- [main]|

|?- run(|{\em InFile}|, |{\em OutFile}|).|

|?- compile(|{\em OutFile}|).|

|?- goal.|


Alternately you can use the following scheme:


|% xeclipse|

|?- [main]|

|?- prove(|{\em InFile}|).|
%%%****************************************************************************
%%% $Id: syntax.tex,v 1.2 1995/01/27 13:45:38 gerd Exp $
%%%============================================================================
%%% 
%%% This file is part of ProCom.
%%% It is distributed under the GNU General Public License.
%%% See the file COPYING for details.
%%% 
%%% (c) Copyright 1995 Gerd Neugebauer
%%% 
%%% Net: gerd@imn.th-leipzig.de
%%% 
%%%****************************************************************************

\def\SYNTAX#1.{\begin{itemize}\item[]\(#1\) \end{itemize}}

%------------------------------------------------------------------------------
\section{The Input Language}

\subsection{Basics}

\ProCom{} is  currently able to process problems  in  clausal form. The chosen
syntax is a superset of pure  Prolog, i.e. without  built-in predicates of any
kind.  Since  the  Prolog  reading  apparatus  is  used  several features  are
inherited directly:
\begin{description}
  \item [Variables\index{variable}] are written with an initial capital letter.

  \item [Functors\index{functor}  and constants\index{constant}] usually start
    with  a lower  case letter.  Additionally  some  predefined  operators are
    defined as infix, e.g. |=|, |<|, |>|, and others.

  \item [Terms\index{term}] are built from functors and variables as usual (in
    Prolog).

  \item [Comments\index{comment}] are anything starting with a |%| up to the
    end  of the line.  Additionally  C-style comments, i.e. starting with |/*|
    and ending in |*/|, can be used.
\end{description}


\subsection{Literals}
\index{literal}

A  literal  is a predicate  with  an optional sign.   Everything  which has an
operator not  treated special is considered  as a predicate. Special operators
are |,|, |;|, |:-|, |<-|, |->|, |?-|, |::|, the list constructor and the empty
list constant |[]|.

A positive literal can  be optionally marked with  the prefix |++|. Internally
the positive sign  is added where required. The  negation sign is |-| or |--|.
Internally |--| is used only.


\subsection{Clauses}
\index{clause}

\SYNTAX |[| L_1 |,| \ldots|,| L_n |]|.
%
The simplest  --- but  hard   to read ---  form  of  a  clause consists of   a
(nonempty) list of literals. They are stored as they are.


\SYNTAX H_1|;|\ldots|;|H_n \,|:-|\, T_1|,|\ldots|,|T_m.
%
This is the general  form of a clause  in  extended Prolog notation.  The head
literals $H_1,\ldots  H_n$  denote  the   negative  literals. Thus  they   are
implicitly  negated when   the internal list  representation is  constructed.
Nevertheless explicit negation using the |-| operator is permitted.

Other degenerate variants --- the head or the tail are  empty --- are provided
as well.


\SYNTAX |:-|\, T_1|,|\ldots|,|T_m.
%
This rule parses the degenerate form of the extended Prolog notation where the
head is empty. Note that  in contrast to Prolog there  is a difference to  the
|?-| operator.


\SYNTAX  H_1|;|\ldots|;|H_n \,|<-|\, T_1|,|\ldots|,|T_m.
%
The forms incorporating the |<-| operator are simply syntactic variants of the
same forms using the |:-| operator.


\SYNTAX |<-|\, T_1|,|\ldots|,|T_m.
%
This form is simply mapped to the corresponding |:-| variant.


\SYNTAX |?-| Goal.
%


\SYNTAX L_1 |;|\ldots|;| L_n.
\SYNTAX L_1 |,|\ldots|,| L_n.
%
A disjunction  is interpreted as  a degenerated left  hand side of an extended
Prolog clause. Thus it is negated while translated into a list.


\subsection{Labels}
\index{label}

\SYNTAX Label |::| Clause.
%
Any clause can have arbitrary information assigned to it.  This is done in two
forms. One form is to mark a clause as goal clause with the |?-| operator. The
more general form implemented here is to write a label  in front of the clause
separated  by the |::| operator.   The label {\em Label}\/  is  stored in  the
Prolog database as |'Label'(|{\em Label}|,|{\em Index}|)|.


%%%****************************************************************************
%%% $Id: options.tex,v 1.9 1995/07/03 11:35:12 gerd Exp gerd $
%%%============================================================================
%%% 
%%% This file is part of ProCom.
%%% It is distributed under the GNU General Public License.
%%% See the file COPYING for details.
%%% 
%%% (c) Copyright 1995 Gerd Neugebauer
%%% 
%%% Net: gerd@imn.th-leipzig.de
%%% 
%%%****************************************************************************
% Master File: manual.tex

%------------------------------------------------------------------------------
\section{Options of \ProCom}\label{sec:options}

\ProCom{} can be adapted in it's behavior using a facility called options. A
option corresponds to a register or variable in procedural programming
language. Options of various kinds are used in \ProCom.

In general options can take any Prolog terms as values. Nevertheless there are
usually some kind of restrictions allowing only some kinds of values.

The most common options are boolean options. Boolean options can take the
values |on| and |off| only. In fact anything which is not |on| is interpreted
as |off|.

Another important type of option can take a file name. This is a Prolog string
pointing to a file name.

Options can be set in a options file which read at the beginning of any run of
\ProCom. Usually this file is called {\sf .procom} and is searched in the
current directory and the home directory in this order.

The \ProCom{} options file can contain instructions set options only.
This can be done using instructions of the following form:

|  |{\em Option}| = |{\em Value}|.|

In this instruction {\em Option}\/ denotes the name of an option and {\em
Value}\/ is it's value. Note that {\em Value} has to conform t Prolog
conventions and the terminating point is not optional! Possible options are
described below.


\subsection{General Options}

\begin{description}
  \item [prover] | = procom(extension_procedure)|\label{opt:prover}\\
	This option indicates the prover which has to be activated. The
	framework of \ProCom{} allows arbitrary provers to be integrated. For
	our purpose we describe only the interface to the provers provided
	with \ProCom{} and the interface to user defined provers.
	
	In this special case the value of this option is a Prolog term
	of the form |procom(|{\em Name}|)| where {\em Name}\/ is the
	name of a prover specified in the file {\sf procom\_config.pl}
	(see page \pageref{procom.config}).

  \item [search] | = iterative_deepening(1,1,1)|\\
	This option indicates which search strategy should be used.
	\begin{description}
	  \item [depth\_first]\ \\
		This search strategy uses the unbound depth first search like
		Prolog provides it.
	  \item [iterative\_deepening($\delta_0$, $\alpha$, $\beta$)]\ \\
		This search strategy successively increments a depth bound.
		The search space is searched to this limit. The limit is
		incremented upon failure.
		\\
		The initial depth $d_0$\/ is $\delta_0$.
		The next depth $d_{n+1}$\/ is computed from the previous depth
		$d_n$\/ according to the formula
		\[ d_{n+1} = \alpha\cdot d_n + \beta
		\]
		If $\alpha$\/ is $1$\/ then a linear function is achieved. If
		$\alpha$\/ is greater then $1$\/ then an exponential function
		can be forced.
	  \item [iterative\_deepening($\delta_0$, $\beta$)]\ \\
	        This is the same as 
		iterative\_deepening($\delta_0$, 1, $\beta$).
	  \item [iterative\_deepening($\delta_0$)]\ \\
	        This is the same as iterative\_deepening($\delta_0$, 1, 1).
	  \item [iterative\_deepening]\ \\
	        This is the same as iterative\_deepening(1, 1, 1).
	  \item [iterative\_inferences($\delta_0$, $\alpha$, $\beta$)]
	        This search strategy successively increments a inference bound.
		The search space is searched to this limit. The limit is
		incremented upon failure.
		\\
		The initial depth $d_0$\/ is $\delta_0$.
		The next depth $d_{n+1}$\/ is computed from the previous depth
		$d_n$\/ according to the formula
		\[ d_{n+1} = \alpha\cdot d_n + \beta
		\]
		If $\alpha$\/ is $1$\/ then a linear function is achieved. If
		$\alpha$\/ is greater then $1$\/ then an exponential function
		can be forced.
	  \item [iterative\_inferences($\delta_0$, $\beta$)]\ \\
	        This is the same as 
		iterative\_inferences($\delta_0$, 1, $\beta$).
	  \item [iterative\_inferences($\delta_0$)]\ \\
	        This is the same as iterative\_inferences($\delta_0$, 1, 1).
	  \item [iterative\_inferences]\ \\
	        This is the same as iterative\_inferences(1, 1, 1).
	  \item [iterative\_widening(a,b,c)]\ \\
		This is an experimental mode implementing iterative widening
		according to \cite{ginsberg.harvey:iterative}.

		{\bf Not ready yet!}
	  \item [iterative\_broadening(a,b,c)]\ \\
		This is an experimental mode implementing iterative
		broadening. 
	\end{description}

  \item [prolog] | = eclipse|\label{opt:prolog}\\
	This option indicates the target language which should be
	generated by \ProCom. Currently only a few possibilities are
	supported. The value |eclipse| indicates that \eclipse{} is
	the target language.  |quintus| stands for Quintus
	Prolog. Finally, |default| generates code in standart prolog,
	i.e.\ without special features of any Prolog dialect. In this
	mode several other options may have no effect and the
	resulting code may not be as efficient as in the specific
	modes.

  \item [input\_path] | = ['Samples']|\\
	This option contains a list of directories which are used to find an
	input file. This search path is used {\em after} the given file name
	has been tried as is. Thus the current directory (|.|) need not to be
	on this path.

  \item [verbose] | = on|\\
	This option turns on general verbosity of actions. Several modules may
	decide to use their own verbosity options.

  \item [toplevel:verbose] | = on|\\
	This option controlls the verbosity of the top level.


  \item [input\_filter] | = ''|\label{opt:input_filter}\\
	Filter to be used before the clauses are read. E.g. this filter can
	perform a normal form transformation. Filters may to be defined in
	{\sf config.pl} to be accessible. Otherwise they are dynamically
        loaded. In this case the file containing the filter has to be found on
        the search path for Prolog files.

	The value of this option can also be a list of filters. In this case
	each filter is used in turn. The output of the preceeding filter is
	given as input to the current filter. The initial filter gets the
	input file as input. The result is the result of the last filter.

        If only one filter is specified then the list can be omitted. I.e. a
        symbol or string value of this option is interpreted as a filter to be
        used.

	Non-existing filters are ignored.

	Let us consider an (imaginary) example:

\begin{BoxedSample}
  input\_filter = [nf\_transform,reductions1,reductions2]
\end{BoxedSample}

\end{description}

\subsection{Options Controlling the Prove Phases}

The prover can roughly be divided into two phases. The first phase is called
the preprocessing phase. In this phase several redution techniques can be
applied to reduce the size or complexity of the problem. The second phase is
the theorem prover itself. According to this overall distinction there are
some options controlling the general behaviour.

\begin{description}
\item [prove:red\_goals] |= [complete_goals,connection_graph]|\\
  This option is a list of modules which are invoked in turn to perform their
  tasks during the preprocessing phase. The elements of this list are
  reduction modules. Those modules are either preloaded or they are
  dynamically loaded.
  
\item [prove:red\_path] |= ['.']|\\
  This option contains the path for the dynamic loading of reduction
  modules. During the installation this path is augmented by the directory
  containing \ProTop{} and its files.

\item [prove:path] |= ['.']|\\
  This option contains the path for the dynamic loading of prover modules.
  The prover to be used is determined by the option |prover|.

  During the installation this path is augmented by the directory containing
  \ProTop{} and its files.

\item [prove:extension] |= ['','.pl']|\\
  This option contains a list of extensions used to find a reduction or a
  prover module. The strings or symbols contained in this list are appended to
  the file name before the existence of such a file is checked.

\item [prove:log\_items] |= [prover,matrix,contrapositives]|\\
  This option determines which information is written to the log file just
  after the first phase of the proving, i.e. the preprocessing. The value
  consists of a list of keywords as described below.

  \begin{description}
  \item[contrapositives]\ \\
    The list of contrapositives is written to the log file. The
    contrapositives are enclosed in |begin(contrapositives)| and
    |end(contrapositives)|. This environment contains log terms of the
    type |contrapositive/3|. The first argument is the head of the
    contrapositive. The second argument is the body, i.e. the list of literals
    without the head. The third argument contains the index of the
    contrapositive. 

    It is highly recommended to leave this item in the list |prove:log_items|
    since some postprocessing programs rely on it.
  \item[date]\ \\
    The current date is written to the log file. The date is stored in the log
    term |date/1|. The argument is a string containing the date in the format
    as given by the command |date/1|.
  \item[host]\ \\
    The host information is written to the log file. The host information is
    stored in the log term |host_info/2|. The first argument contains the
    hostname. The second argument contains the host architecture.
  \item[matrix]\ \\
    The list of clauses is written to the log file. The clauses are enclosed
    in |begin(matrix)| and |end(matrix)|. This environment contains log terms
    of the following types:
    \begin{description}
    \item[GoalClause/1] contains the indices of the goal clauses.
    \item[Label/2] contains the labels of the clauses. The first argument is a
      clause index and the second argument is its label.
    \item[Clause/2] contains the  clauses. The first argument is the index of
      this clause. The second argument is the list of literals. Each literal
      is of the form |literal(|$L$|,|$I$|)| where $L$\/ is a signed predicate
      and $I$\/ is its index.
    \end{description}

    It is highly recommended to leave this item in the list |prove:log_items|
    since some postprocessing programs rely on it.
  \item[options]\ \\
    The complete list of options and their values is written to the log file.
  \item[prover]\ \\
    The prover to be used is written to the log file. This information could
    also be extracted from the options, but sometimes this is the only option
    we are interested in.

    The prover name is stored in the log term |prover/1|.
  \item[user]\ \\
    The name of the user is written to the log file. The user information is
    stored in the log term |user/2|. The first argument is the login name of
    the user and the second name is the full name of the user as given in the
    GCOS field of the passwd file.
  \end{description}

\end{description}

\subsubsection{Options of the Reduction Module {\tt complete\_goals}}

\begin{description}

  \item [complete\_goals] | = on|\\
  If this option is turned on then the matrix is checked to contain a
  complete set of goals. If the set of goals is not complete then it is
  completed.

\end{description}

\subsubsection{Options of the Reduction Module {\tt connection\_graph}}

\begin{description}

  \item [find\_connections] | = on|\\
  This option can be used to suppress the generation of the reachability
  graph. \ProCom{} does not assume that this option is turned off. The
  result is unpredictable in this case.

  \item [find\_all\_connections] | = off|\\
  If this option is on then all literal are initially considered when
  constructing the reachability graph. Otherwise only literals in goal
  clauses are considered.

  \item [connect\_weak\_unifiable] | = on|\\
  One of two methods to compute the reachability graph can be selected.  The
  first method is to connect complementary literal which are weak
  unifyalbe. The alternative is to consider the predicate symbol only.

  \item [remove\_unreached\_clauses] | = on|\\
  If this option is on then each unreached clause is removed after the
  reachability graph has been constructed.

\end{description}


\subsection{Automatic Options}

This section contains options which are automatically set. They can be checked
but should not be altered in any way.

\begin{description}
  \item [equality] | = off|\\
  This boolean option is set when an equality predicate (|=/2|) is detected
  in the input problem.

  \item [setvar] | = off|\\
  This boolean option is set when an set variable (|:/2|) is detected in the
  input problem.

  \item [input\_file] | = ''|\\
  This option is automatically set to the current input file name as given
  by the user, i.e. without the automatically appended directory from the
  search path.

  \item [output\_file] | = ''|\\
  This option is set to the output file name as given by the user.

  \item [interactive] | = off|\\
  This option is set when the output is not redirected to a file. Thus an
  interactive prover can act accordingly.

\end{description}



\subsection{\ProCom\ General Options}

\begin{description}

  \item ['ProCom:dynamic\_reordering'] | = on|

  \item ['ProCom:ancestor\_pruning'] | = on|
    \\
    This option enables generation of code performing identical ancestor
    pruning. This means that a branch of the search tree is cut if two
    identical subgoals have been encountered. Only a minor variant may be
    implemented.

  \item ['ProCom::ancestor\_pruning'] | = 'prune.pl'|
    \\
    This option specifies the library which provides the predicates to perform
    the identical ancestor check.

  \item ['ProCom:occurs\_check'] | = on|
    \\
    This option indicates wheter special care should be taken to perform sound
    unification. In this case special caution is taken at the clause level.
    Another variant might be to enable the occurs check globally --- if this
    feature is provided by the Prolog dialect used.

  \item ['ProCom::unify'] | = 'unify.pl'|
    \\
    This option specifies which library should be used to get the predicate
    |unify/2| which performs sound unification. This library is used if the
    option  |ProCom:occurs_check| is |on|.

  \item ['ProCom::unify\_simple'] | = 'unify-no-oc.pl'|
    \\
    This option specifies which library should be used to get the predicate
    |unify/2| which performs unification. This library is used if the option
    |ProCom:occurs_check| is |off|.

  \item ['ProCom::member'] | = 'member.pl'|
    \\
    This option specifies which library should be used to get the predicate
    |member/2| which is a sound member predicate. Ususally this library will
    in turn use the library predicate |unify/2|. Thus the value of the option
    |ProCom:occurs_check| is taken into account.

  \item ['ProCom:literal\_selection'] | = deterministic|
    \\
    This option determines the strategy for literal selection. It can take the
    values |deterministic| or |random|\footnote{Currently not supported.}.

  \item ['ProCom:path'] | = path.pl|\label{opt:ProCom:path}
    \\
    This option determines the library containing the path managment
    predicates.  See section \ref{sec:lib.path} for details.

  \item ['ProCom::lemma'] | = 'no-lemma.pl'|
    \\
    This option specifies the library containing the lemma handling
    predicates.  The standard libraries contain also the library |lemma.pl|
    which is recommended as a basis for your own lemma handling routines.

  \item ['ProCom:lemma'] | = off|
    \\
    This option can be used to suppress the compiling of the calles to
    |lemma/3| which are defined in the lemma library.

  \item ['ProCom:automatic\_put\_on\_path] | = on |
    \\
    This option can be used to suppress the automatic addition of a goal to
    the current path. If youturn this option off you might want to use the
    primitive |put_on_path| in your descriptor to put some liteals on the
    path.

  \item ['ProCom:add\_contrapositives'] | = off|
    \\
    This option enables the generations of contrapositives in the target
    program. In this case the facts |contrapositive/3| are defined.

    The contrapositives may be needed by heuristic functions, e.g. for literal
    selection, which want to inspect the whole problem.

  \item ['ProCom:proof\_limit'] | = off|
    \\
  This option can be used to specify the way to proceed after a proof has been
  found and presented. The following values are currently supported:
  \begin{description}
    \item [interactive]\ 
      \\
      This value forces the generation of code to query the user. This is
      similar to the top level loop of Prolog. The library |ProCom::more| (see
      section \ref{lib:more}) is used in this case.
    \item [all]\ 
      \\
      This value forces the search for further solutions.
    \item [{\em number}]\ 
      \\
      Any natural number forces the search for further solutions until this
      number of solutions are found or all solutions are exhausted. The
      library specified by the option |ProCom::proof_limit| (see section
      \ref{lib:proof_limit}) is used in this case.
  \end{description}

  Any other value will force the termination after the first solution.

\end{description}

\subsection{\ProCom\ Goal Compiler Options}


\begin{description}
  \item ['ProCom:post\_goal\_list'] | = []|
    \\
    This option specifies a list of Prolog predictes which are called after
    the proof of a goal has been successful. As additional arguments the list
    of variable bindings and the proof tree is added to the predicate given.

    Consider the |ProCom:post_goal_list| containing the element |pgl(1,47)|.
    Then the goal |pgl(1,47,Vars,Proof)| is compiled into the prover, where
    |Vars| and |Proof| are instantiated to the variable bindings and the proof
    tree respectively.

    Note that you have to ensure that the predicates given are present in the
    compiled Prolog program. This can be done by specifying appropriate
    libraries containing the Prolog code.

  \item ['ProCom:init\_goal\_list'] | = []|
    \\
    This option specifies a list of Prolog predicates which are called when a
    goal is launched.

    Note that you have to ensure that the predicates given are present in the
    compiled Prolog program. This can be done by specifying appropriate
    libraries containing the Prolog code.

  \item ['ProCom:init\_level\_list'] | = []|
    \\
    This predicate specifies a list of predicates which are called whenever a
    new depth level is started, i.e. right after |set_depth_bound/1| has
    returned a new depth. The depth is added as additional argument to the
    predicates given.

    Note that you have to ensure that the predicates given are present in the
    compiled Prolog program. This can be done by specifying appropriate
    libraries containing the Prolog code.

\end{description}

See also the section on reordering (\ref{procom:reordering}) for further
options to influence the goal compilation.



\subsection{\ProCom\ Linker and Optimizer Options}

The options described in this section influence the behaviour of the linker.
The linker is the last step in the program generation process. Several tasks
are performed here. The main task is to add missing predicates from the
libraries. For this purpose possible libraries have to be named.

Another task perform in this last step is the application of certain
optimizations. The optimizations are mainly unfolding for Prolog predicates
for efficiency. These optimizations drastically reduce the readability of the
generated code. Thus it is recommened to turn them of when first trying to
understand the generated Prolog program.

\begin{description}

\item ['ProCom:link'] | = static|\\ 	This option specifies the type of
  linking. Possible values are
  \begin{description}
  \item [off]\ 
    \\
    This value disbales the linker. This is recommened for development
    purposes only. In this case the generated code is not executable as is.
  \item [static]\ 
    \\
    This is the usual way of linking. Any Prolog code from the libraries is
    copied into the generated Prolog program. Thus the program can be run on
    any appropriate Prolog system.
  \item [dynamic]\ 
    \\
    If this variant is used then for most of the libraries only compile
    instructions are generated in the target program. Thus this program is
    only executable if the used libraries are still accessible under the names
    given during linking.
  \end{description}

\item ['ProCom::link\_path']\label{opt:ProCom::link_path}\ 
  \\
  This option holds a list of directory names where the linker looks for
  appropriate files. The value of the option |prolog| is appended to those
  directories.

\item ['ProCom:optimize'] | = on|
  \\
  The linker has built-in the ability to perform certain optimizations.  For
  testing purposes it might be desirable to disable those optimizations.  In
  general it is not recommended to turn them off.

\item ['ProCom:expand'] | = on|\label{opt:ProCom:expand}
  \\
  This option enables the expansion (i.e. unfolding) of non-recursive
  predicates from the libraries. In the libraries one can specify which
  predicates should be considered (see section~\ref{sec:contents.library}).

\item ['ProCom:expand\_aux'] | = on|
  \\
  This option enables the expansion (i.e. unfolding) of auxiliary predicates.

\item ['ProCom:module'] | = off|
  \\
  This option enables the generation of code which encapsulates the prover in
  a module. This is only possible if a module system is provided by the Prolog
  dialect.

\item ['ProCom::module'] | = 'module.pl'|
  \\
  This option specifies the library which contains the module head.

\item ['ProCom::immediate\_link] | = ['init.pl']|%
  \label{opt:ProCom::immediate_link}\\
  This option specifies a list of libraries which are added to the code
  at the beginning --- possibly right after the module initialization but
  before anything else.

\item ['ProCom::post\_link] | = []|
  \\
  This option specifies a list of libraries which are added to the code at the
  end. This might be used to add libraries which are encapsulated in modules.

\item ['ProCom:ignore\_link\_errors'] | = off|
  \\
  This option can be used to suppress the linker to abort because of missing
  predicates. Usually the linker checks wether all predicates used are also
  defined. Those checks can fail if the predicate is defined in a library file
  or a module loaded with the |ProCom::post_link| option. Thus it can be
  desirable to ignore the linker errors. Nevertheless the error messages are
  displayed but ignored afterwards.
\end{description}

\subsection{\ProCom\ Information Controlling Options}

Information about the proof and the proof process can be of intrest. Thus it
is possible to enable the generation of such information. Since thiese
additional operations cost some time they may be worth turning off, when a
fast proof is required and the information is not essential.

\begin{description}

  \item ['ProCom:verbose'] | = off|
    \\
    This option enables the generation of various comments in the target
    Prolog program. The generated code my be slightly more readable when this
    option is on.

  \item ['ProCom:proof'] | = on|
    \\
    This option enables the generation of code to collect the information
    representing the proof tree. This takes some time and space and should be
    turned off when a fast prover is required. If a proof tree is enabled the
    predicates required are taken from the library given in the next option.

  \item ['ProCom::proof'] | = 'proof.pl'|
    \\
    This option specifies the library which is used for proof tree generation
    and display.

  \item ['ProCom:trace'	] | = off|
    \\
    This option enables code generation for the built-in debugger. This
    debugger can be used to trace the activities of the prover. (see
    \ref{sec:debugger})

  \item ['ProCom::trace'] | = 'debugger.pl'|
    \\
    This option specifies the library which contains the debugger.

  \item ['ProCom:timing'] | = on|
    \\
    This option enables the generation of code to determine the run time of
    the theorem prover. This is only possible in dialects which provide means
    to acess the run time.

  \item ['ProCom::timing'] | = 'time.pl'|
    \\
    This option specifies the library which contains code to set and query the
    timer.

  \item ['ProCom:show\_result'] | = on|
    \\
    This option enables the generation of code to display variable bindings
    and allow the user to ask for additional solutions.

  \item ['ProCom::show\_result'] | = 'show.pl'|
    \\
    This option specifies the library which contains code to display the goal
    and variable bindings after a succesful proof attempt.

\end{description}


\subsection{\ProCom\ Reordering}\label{procom:reordering}

\ProCom{} provides a powerful facility to reorder things before, during and
after the compilation process. This is an addition to the reordering hooks
provided in the preprocessing which are far less expressive.

In contrast to the |reorder_literals|/|clauses| functions which are assumed to
poke around in the Prolog database the \ProCom{} reordering facility provides
an interface which allows the user to specify an order and leave the
reordering to the system. This has the advantage that the user can do no harm
to the data, i.e. alter or delete entities.

The system performs the sorting using a user written comparison predicate. To
allow several sets of comparison predicates to coexist those predicates are
hidden in modules. The options specify the module name in which certain
predicates are expected. If no reordering is wanted then the empty symbol |''|
can be used instead.

The predicates given should succeed if the first element is less or equal than
the second element. These predicates should be deterministic.

\begin{description}
\iffalse
  \item ['ProCom:reorder\_ext']             | = ''|
    \\
    This option provides the name of a module containing a predicate
    |compare_ext/2|. This predicate is used to compare
  \item ['ProCom:reorder\_goal']            | = ''|
    \\
    This option provides the name of a module containing a predicate
    |compare_goal/2|. This predicate is used to compare
\fi
  \item ['ProCom:reorder\_clauses']         | = ''|
    \\
    This option provides the name of a module containing a predicate

    |compare_clauses/2|

    This predicate is used to compare clauses.
  \item ['ProCom:reorder\_goal\_clauses']   | = ''|
    \\
    This option provides the name of a module containing a predicate

    |compare_prolog_clauses/2|

    This predicate is used to compare
  \item ['ProCom:reorder\_prolog\_clauses'] | = ''|
    \\
    This option provides the name of a module containing a predicate

    |compare_prolog_clauses/2|

    This predicate is used to compare Prolog
    clauses of a procedure just after they have been generated and before the
    optimizer has done it's work. Depending on other options the clauses may
    have different forms. E.g. auxiliary predicates may be expanded.
  \item ['ProCom:reorder\_aux\_clauses']    | = ''|
    \\
    This option provides the name of a module containing a predicate

    |compare_aux_clauses/2|

    This predicate is used to compare auxiliary clauses.
\end{description}



%%%****************************************************************************
%%% $Id: capri.tex,v 1.5 1995/03/20 21:24:47 gerd Exp $
%%%============================================================================
%%% 
%%% This file is part of ProCom.
%%% It is distributed under the GNU General Public License.
%%% See the file COPYING for details.
%%% 
%%% (c) Copyright 1995 Gerd Neugebauer
%%% 
%%% Net: gerd@imn.th-leipzig.de
%%% 
%%%****************************************************************************
% Master File: manual.tex

\chapter{Programming \ProCom}

%------------------------------------------------------------------------------
%%%****************************************************************************
%%% $Id: libraries.tex,v 1.5 1995/02/13 20:05:14 gerd Exp $
%%%============================================================================
%%% 
%%% This file is part of ProCom.
%%% It is distributed under the GNU General Public License.
%%% See the file COPYING for details.
%%% 
%%% (c) Copyright 1995 Gerd Neugebauer
%%% 
%%% Net: gerd@intellektik.informatik.th-darmstadt.de
%%% 
%%%****************************************************************************
% Master File: manual.tex

\section{Libraries}

The next stage of modification beyond the adaption of option is to replace
libraries by own code. For this purpose we will describe in detail the
libraries used. Thus it is possible to write own libraries for new Prolog
dialects. Additionally it also enables you to replace the libraries provided
with \ProCom{} by your own versions which have an improved performance or an
enhance functionality.

\paragraph{Note:} Don't modify the libraries provided with \ProCom. Make
modified versions with a different name instead and set the appropriate option
to tell \ProCom\ to use this modified version. It is also possible to exploit
the search mechanism for libraries (see \ref{sec:lib.search}) to place the
modified version in a directory which is searched before the system library.

In the following sections we will describe the libraries and the predicates
they are expected to provide.


\subsection{Library Search}\label{sec:lib.search}
\index{library search}


At the end of the compilation process the linker tries to add libraries to the
code which provide required predicates. For this purpose the linker needs to
know which predicates are required and where to find libraries which might
provide them. Requirements of predicates are specified by the implementor, in
the Capri interpreter, in the Capri modules, or in the libraries.

When the linker is initialized it analyzes the files given to it to find out
which predicates are defined there. Usually the files known to the linker are
initially taken from several option (those containing |::|). The instruction
|link_file/1| in a Capri module can be used to provide additional files.

The link files are searched in the following way. The option
|ProCom:link_path| (see section \ref{opt:ProCom::link_path}) contains a list
of directories which are used to find the libraries. Two subdirectories of the
directories given are also considered. The first subdirectory is named like
the Prolog dialect specified by the option |prolog| (see section
\ref{opt:prolog}). The second subdirectory is called {\sf default}.

The general idea is to place generic code which should run on each Prolog
system in the subdirectoy {\sf default} and the dialect specific modules in
the specific subdirectories, e.g. {\sf eclipse}.

In addition to the intermediate directories the libraries are augmented by the
extension |.pl| during the search. To make it clear which files are considered
let us have a look at an example.

Suppose the option |prolog| has the value |eclipse| and the option
|ProCom:link_path| has the value |[.,ProCom]|. When the linker looks for a
library named |my_lib| the following locations are inspected until an existing
file is found:

{\tt
\begin{tabular}{l}
  ./my\_lib
  \\./my\_lib.pl
  \\./eclipse/my\_lib
  \\./eclipse/my\_lib.pl
  \\./default/my\_lib
  \\./default/my\_lib.pl
  \\ProCom/my\_lib
  \\ProCom/my\_lib.pl
  \\ProCom/eclipse/my\_lib
  \\ProCom/eclipse/my\_lib.pl
  \\ProCom/default/my\_lib
  \\ProCom/default/my\_lib.pl
\end{tabular}
}


\subsection{Contents of Library Files}\label{sec:contents.library}
\index{library file}

Library files are mainly usual Prolog files --- with some exceptions.

You should be very carefully when making a module in a library. Be sure that
you completely understand the translation process and can predict the
resulting Prolog code.

Since the library file is read by the linker each operator declaration must be
known to the linker. Operator declarations in the library are not evaluated.

Most of the instructions are simply passed to the output file. Nevertheless the
linker uses some instructions to control its behavior. The folloing list
describes instructions evaluated by the linker. Each such instruction is
embedded in a single Prolog goal. Accumulating several of them or embedding in
other constructs won't work.

In the following list the expression {\em Pred} always denotes a predicate
specification in the form {\em Functor/Arity}.
\begin{description}
  \item [:- expand\_predicate({\em Pred})]\index{expand\_predicate}\ \\
  This instruction is passed to the optimizer to declare the predicate {\em
  Pred} as expandable. If the option |ProCom:expand| (see section
  \ref{opt:ProCom:expand}) is turned on them expandable predicates are
  replaced by their bodies.

  You should only declare non-recursive, single-clause predicates as
  expandable\footnote{A future version of the linker may determine those
  predicates automatically.}.  Recursive predicates can not be expanded
  properly.  Don't try it. Multi-clause predicates may also cause problems.

  The declaration of expandable predicates must preceed their definition!

  \item [:- provide\_predicate({\em Pred})]\index{provide\_predicate}\ \\
  This instruction tells the linker that the predicate {\em Pred} is defined
  in this library. This should usually be not necessary since the linker
  analyzes the Prolog program to see which predicates are defined.
  Nevertheless it can be necessary to use it to declare a Prolog built-in
  which is required (see below).

  \item [:- require\_predicate({\em Pred})]\index{require\_predicate}\ \\
  This instruction tells the linker that the predicate {\em Pred} is used in
  this library but not defined. The linker makes not a complete analysis of
  used and defined predicates. Instead it evaluates this instruction to get
  the undefined predicates. This can be used to trigger the loading of other
  libraries.
\end{description}


\subsection{The Init Library}\label{sec:lib.init}

For some reasons it may be desirable to add some pieces of code in front of
anything else. This can be used to adjust the behavior of the Prolog system.
A list of libraries is taken from the option |ProCom::immediate_link| to be
linked at the beginning (see section \ref{opt:ProCom::immediate_link}).

The following sample implementations are provided with \ProCom.
\begin{description}
  \item [eclipse/init.pl]\ \\
  This library turns off some debugging features to speed things up.
  Additionally the sound unification is turned on.
  \item [default/init.pl]\ \\
  This library is simply empty. No initializations are needed for a vanilla
  Prolog.
\end{description}


\subsection{The Module Library}\label{sec:lib.module}

The resulting Prolog code can be encapsulated into a module. The head of a
module with a fixed name and export list can be placed in a library named in
the option |ProCom::module|.

The following sample implementation is provided with \ProCom.
\begin{description}
  \item [eclipse/module.pl]\ \\
  This library contains the head of a \eclipse\ module named |'ProCom prover'|
  which exports the predicates |goal/0| and |goal/1|.
\end{description}

\paragraph{Note:} Most of the time it is not desirable to generate a stand
alone module since \ProTop\ wrappes a module around the prover anyway.

\subsection{The Path Management}\label{sec:lib.path}

The data sturcture used for the path is completely encapsulated in a library.
The option |ProCom:path| (see section \ref{opt:ProCom:path}) determines the
library actually used. The following sample implementations are provided with
\ProCom.
\begin{description}
  \item [default/path.pl]
  \item [default/path-simple.pl]
  \item [eclipse/path-regular.pl]\ \\
  This library uses \eclipse\ delayed predicates to implement a variant of
  regularity constraints.
\end{description}

The current implementations use a pair of lists to implement the path. The
first item is the list of positive ancestors and the second one contains the
negative ancestors.  Since the path management is encapsulated in this module
one can replace this kind of path by another data structure. Linear lists are
a simpler case. But also balanced binary trees can be considered. Any
implementation which allows a fast access, e.g. by using sofisticated indexing
mechanisms, might be worth trying it.

The path management consists of the following set of predicates:
\begin{description}
  \item [empty\_path({\em Path})]\index{empty\_path}\ \\
  This predicate should unify its argument with the term representing the
  empty path. It should have at most one solution.

  \item [put\_on\_path({\em Literal}, {\em OldPath}, {\em
  NewPath})]\index{put\_on\_path}\ \\
  This predicate modifies the path {\em OldPath} in such a way that the
  literal {\em Literal} is in it. The result is unified with the new path {\em
  NewPath}.

  \item [put\_on\_path({\em Literal}, {\em Info}, {\em OldPath}, {\em NewPath})]\index{put\_on\_path}

  \item [is\_on\_path({\em Literal}, {\em Path})]\index{is\_on\_path}\ \\
  This predicate tries to unify {\em Literal} with an element on the path {\em
  Path}. Upon backtracking all such candidates are tried.

  \item [is\_on\_path({\em Literal}, {\em Info}, {\em
  Path})]\index{is\_on\_path}

  \item [is\_identical\_on\_path({\em Literal}, {\em
  Path})]\index{is\_identical\_on\_path}\ \\
  This predicate tries to find {\em Literal} on the path {\em Path}. Only
  identical (not only unifyable) occurences are considered. This predicate
  should not be resatisfiable.
\end{description}


\subsection{Sound Unification}

The following sample implementations are provided with \ProCom.
\begin{description}
  \item [eclipse/unify.pl]\ 
    \\
    This library provides a sound |unify/2| predicate.
  \item [default/unify-no-oc.pl]\ 
    \\
    This library provides a |unify/2| predicate not performing the occurs
    check.
\end{description}

The following predicate is provided by those modules:

\begin{description}
  \item [unify({\em Term1}, {\em Term2})]\index{unify}\ 
    \\
    This predicate unifies its arguments.
\end{description}



\subsection{Sound Member Implementation}\label{sec:member}

The following sample implementation is provided with \ProCom.
\begin{description}
  \item [default/member.pl]\ 
    \\
    This library uses the unify/2 predicate to perform unification. Thus it
    depends on the unify library which unification is actually used.
\end{description}

\begin{description}
  \item [sound\_member({\em Element}, {\em List})]\index{sound\_member}\ 
    \\
    This predicate implements the well known member relation. Modification may
    be necessary to use the occurs check when required.
\end{description}


\subsection{Lemmas}\label{sec:lib.lemma}

After the sucessful attempt to solve a subgoal the result can be stored as a
lemma. Thus there exists a lemma library where this feature can be hooked in.
The following sample implementations are provided with \ProCom.
\begin{description}
  \item [default/no-lemma.pl]\ \\
    This library performs no lemma steps. It is just a dummy library to be
    used if nothing else is desirable.
  \item [default/lemma.pl]\ \\
    This library uses the path to store local lemmata. Thus no special
    \CaPrI{} descriptor set is needed to apply the lemmas. They are applied in
    the course of normal reduction steps.
\end{description}

The lemma library has to provide the following predicate:
\begin{description}
  \item [lemma({\em Literal}, {\em PathIn}, {\em PathOut})]\index{lemma}\ \\
    {\em Literal}\/ is the internal representation of the negated 
    solved literal. {\em PathIn}\/ is the current path. It has to be returned
    in {\em PathOut}, possibly enhanced by some additional literals. E.g. the
    predicate |put_on_path/3| might be useful (cf.\
    section~\ref{sec:lib.path}).
\end{description}


\subsection{Literals}\label{sec:lib.literal}

The literals are packed into a predicate to allow manipulations of literals at
link time. Usually a literal is just left alone.
The following sample implementations are provided with \ProCom.
\begin{description}
  \item [default/literal.pl]\ \\
    This library performs nothing. It is just a dummy library to be
    used if nothing else is desirable.
  \item [eclipse/literal\_static.pl]\ \\
    This library restricts the number of solutions of propositional goals to
    one. The test for free variables is performed at compile time (link
    time). Thus it may not detect goals which are propositional after some
    variables have been bound at run time.
  \item [eclipse/literal\_dynamic.pl]\ \\
    This library restricts the number of solutions of propositional goals to
    one. The test for free variables is performed at run time. Thus it may
    cause some additional overhead.
\end{description}

The literal library has to provide the following predicate:
\begin{description}
  \item [literal\_wrapper({\em Literal}, {\em Varlist})]\index{literal\_wrapper}\ \\
    {\em Literal}\/ is the goal representation of the literal to be
    solved. {\em Varlist}\/ is the list of variables ocurring in the literal
    at compile time.
\end{description}


\subsection{Timing}\label{sec:timing}

The time library provides means to access and measure the time elapsed. Since
the plain Prolog does not seem to have any means to perform this task it is
highly dialect specific how this can be done.

The following sample implementation is provided with \ProCom.
\begin{description}
  \item [eclipse/time.pl]\ 
    \\
    This library provides the predicates to manipulate the time for the
    \eclipse{} system.
  \item [eclipse/time.pl]\ 
    \\
    This library provides the dummy predicates for timing. Since there is no
    standard for those predicates they are simply mapped to do nothing.
\end{description}

The following predicates are provided by the time library:

\begin{description}
  \item [set\_time]\index{set\_time}\ 
    \\
    This predicate is used to initialize the system timer.
  \item [reset\_print\_time(File)]\index{reset\_print\_time}\ 
    \\
    This predicate is used to determine the ammount of time elapsed since the
    last call to a timer routine. This time is printed to the standard output.
    If {\em File} is an atom and not |[]| then the time is also appended to
    this log file.
  \item [set\_timeout({\em Limit})]\index{set\_timeout}\ 
    \\
    This predicate arranges things that the execution is interrupted after
    {\em Limit} seconds. If the Prolog dialect has no such capability it
    simply succeeds.
  \item [reset\_timeout]\index{reset\_timeout}\ 
    \\
    This predicate stops the timeout timer or does nothing.
\end{description}


\subsection{Log Files and Proof Presentation}\label{lib:proof}

The following sample implementation is provided with \ProCom.
\begin{description}
  \item [eclipse/proof.pl]\ 
\end{description}

\begin{description}
  \item [save\_bindings({\em Index}, {\em Clause}, {\em Bindings}, {\em
  File})]\index{save\_bindings}\ \\
  This predicate writes the bindings to the log file {\em File}. If {\em
  File} is the empty list |[]| or not an atom then this predicate simply
  succeeds.

  \item [save\_proof({\em Proof}, {\em File})]\index{save\_proof}\ \\
  This predicate saves the proof term {\em Proof}\/ in the log file {\em
  File}. If {\em File}\/ is the empty list |[]| or not an atom then this
  predicate simply succeeds.

  \item [show\_proof({\em Proof})]\index{show\_proof}\ \\
  This predicate displays the proof term {\em Proof} on the standard output.
\end{description}


\paragraph{more.pl}\label{lib:more}

\begin{description}
  \item [more({\em Depth})]\ \\
  This predicate queries the user to continue the search for a next solution
  or to abort. {\em Depth}\/ contains the current depth limit or is an unbound
  variable if no depth limit is used. This predicate succeeds iff {\em no}\/
  more solutions are required.
\end{description}


\paragraph{show.pl}\label{lib:show}

\begin{description}
  \item [list\_bindings({\em GoalIndex},{\em GoalClause},{\em Bindings})]\ \\
    This predicate is called after a solution has been found. It is meant to
    present the solution.
  \item [no\_more\_solutions]\ \\
    This predicate is called when the search tree has been exhausted and no
    more solutions can be found.
  \item [no\_goal]\ \\
    This predicate is called when no goal clause is left.
\end{description}


\paragraph{proof\_limit.pl}\label{lib:proof_limit}

\begin{description}
  \item [init\_proof\_limit]\index{init\_proof\_limit}\ \\
  This predicate initializes the number of proofs already found to 0. This can
  be done by asserting a fact to the Prolog data base or any other method
  which allows the value to survive backtracking.

  \item [check\_proof\_limit({\em Limit})]\index{check\_proof\_limit}\ \\
  This predicate increments the number of proofs already found and compares it
  with {\em Limit}. This predicate succeeds iff the number of proofs found is
  greater or equal to the {\em Limit}.

  \item [get\_proof\_limit({\em Limit})]\index{get\_proof\_limit}\ \\
  This predicate unifies the number of proofs already found with {\em Limit}.
\end{description}




\subsection{The Search Libraries}\label{lib:search}


\begin{description}
  \item [set\_depth\_bound({\em Start}, {\em Factor}, {\em Const}, {\em
  Depth})]\index{set\_depth\_bound}\ \\
  This predicate 

  \item [check\_depth\_bound({\em Depth}, {\em NewDepth}, \_,
  \_)]\index{check\_depth\_bound}\ \\
  This predicate 

  \item [show\_depth({\em Depth})]\index{show\_depth}\ \\
  This predicate 

  \item [choose\_step({\em Step}, {\em Cand}, \_, \_)]\index{choose\_step}\ \\
  This predicate 
\end{description}


\subsection{The Debugger}\label{sec:debugger}






%------------------------------------------------------------------------------
\section{Implementing a Calculus}

To integrate a new calculus into \ProCom{} can be done dynamically or
statically. In the first case you have to write a \CaPrI{} description file
which is dynamically loaded if required. Those files are not taken into
account when an appropriate prover is selected.

Alternatiely you can integrate the new calculus into the precompiled \ProCom{}
executable. In this case it is considered as a possible prover to be used when
none is given or the given one is not appropriate.

To use a dynamic \CaPrI{} description file you have to
\begin{enumerate}
  \item Write a description file as described in the next sections.
  \item Put it on the search path such that dynamic loading can find it.
    The search path is given by the option |ProCom:capri_path|. To find the
    files the extensions from the options |ProCom:capri_extensions| are taken
    into account.
\end{enumerate}


To integrate a new calculus into the compiled \ProCom{} is done in three steps
\begin{enumerate}
  \item Write a description file as described in the next sections.
  \item Add a line in {\sf Makefile} as described in section
    \ref{sec:installation}.
  \item Recompile \ProCom{} as described in section~\ref{sec:recompile}.
\end{enumerate}



%------------------------------------------------------------------------------
\subsection{Configuring \ProCom}\label{procom.config}

The configuration can be performed at several levels. The most desirable place
is the Makefile (see section \ref{sec:installation}. Another file --- which is
automatically generated by the Makefile --- is described next.

The file {\sf ProCom/procom.cfg}\/ is a simple Prolog file which contains
facts for the Prolog predicate |define_prover/1|. Those facts describe which
modules should be loaded into the resulting \ProCom{} executable.

Several provers can be loaded at the same time. The selection for one
compilation is done with the option |prover| (see \pageref{opt:prover}).
Consider the following example of a file {\sf ProCom/procom.cfg}.

\begin{Sample}\index{define\_prover}
\begin{verbatim}
define_prover(extension_procedure).
define_prover(me_paramod).
define_prover(my_own_calculus).
\end{verbatim}
\end{Sample}

In this case three provers are defined. They can be activated with the
settings
\\ |prover = procom(extension_procedure)|,
\\ |prover = procom(me_para)|, or
\\ |prover = procom(my_own_calculus)| respectively.


The prover used is selected according to the following strategy:
\begin{itemize}
\item If the prover specified in the option |prover| is applicable then this
  prover is used. A prover is applicable if the options which are declared to
  be required are set appropriately (see page \pageref{require_option}).

\item If the specified prover is not applicable then each prover given in the
  configuration file {\sf ProCom/procom.cfg}\footnote{This file is
    created automatically by the Makefile.} is tried in turn until one is
  found which is applicable.
\end{itemize}



\subsection{The Framework}

If you want to create a new prover using the CaPrI interface you have to write
a new module. This module has to be compatible with the existing ones.
Basically three distinct pieces of information are involved in this process:
\begin{itemize}
  \item The name of the module.\\
	This name is given in the |module/1| declaration.
  \item The file name of the module.\\
	This file name is used when writing a new module.
  \item The predicate performing the compilation.\\
	This predicate is used implicitely and you don't have to care about
	it, except that you should not use such a predicate yourself.
\end{itemize}

Thus the starting point is a file named {\em my\_own\_calculus}|.pl| which
starts with the following code:
\smallskip

{\bf :- module({\em my\_own\_calculus}).\\
     :- compile(library(capri)).
}
\medskip

The first line starts a new module with the name given. The second directive
ensures that the CaPrI driver routines are included and a predicate {\em
my\_own\_calculus} is defined.

To continue you can consult the next sections to see which instructions can be
used further on.


\subsection{Describing Deduction Steps}

\begin{description}
  \item [descriptor {\em Code}.]\index{descriptor}\ \\
	{\em Code}\/ is a comma seperated list of goals of the form described
	below.
\end{description}

Consider the following example which implements the reduction step of the
extension procedure. This step says that a {\em goal}\/ |Pred| can be solved
by finding the negated literal |-Pred| on the {\em path}. Additionally we can
name this proof step as |reduction(Pred)|. This information is used to
construct a proof tree (if desired).

\begin{BoxedSample}
descriptor
        proof(reduction(Pred)),
        template(Pred,goal),
        template(-Pred,path).
\end{BoxedSample}



\begin{description}
  \item [name({\em Name})]\index{Capri!name}\ 
    \\
    This specification is meant to name a deduction step. It is optional to
    provide it. If none is given then the functor of |proof| (see below) or a
    default is used.

    It can be desirable to distinguish some deduction steps even so they
    generate the same node in the proof tree.

  \item [proof({\em Proof})]\index{Capri!proof}\ 
    \\
    This specification is used to generate the proof tree. The functor of {\em
      Proof}\/ may also be used as the name of a deduction step (see above).
    Thus {\em Proof}\/ is required to be a compound term.

  \item [template({\em Pattern})]\index{Capri!template}\ 
    \\
    This template is a short form to determine a goal pattern. All goal
    patterns leading to one instance of a deduction step are unified. Thus
    various aspects of a goal can be used.

    The following example describes any goal which has a positive sign.

    \begin{BoxedSample}
      templates(++Literal)%
    \end{BoxedSample}

  \item [template({\em Pattern}, goal)]\index{Capri!goal template}\ 
    \\
    This template is used to determine the goal literal to start with.
    Each descriptor must have at least one goal template. Otherwise no 	code
    is generated for it.

    The following example describes any goal which has a positive sign 	again,
    but this time a complete version (no abbreviation) is used.

    \begin{BoxedSample}
      templates(++Literal,goal)%
    \end{BoxedSample}

%  \item [template({\em Pattern}, goal({\em Index}))]

  \item [template({\em Pattern}, path)]\index{Capri!path template}\ 
    \\ 
    This template describes a subgoal which is solved by unifying it with a
    literal on the path.

    The following example can be understood as a part of the extension
    procedure. The negated literal |-Literal| is searched on the path.

    \begin{BoxedSample}
      templates(-Literal,path)%
    \end{BoxedSample}

  \item [template({\em Pattern}, path({\em Info}))]\index{Capri!path
      template}\ 
    \\
    This template describes a subgoal which is solved by unifying it with
    a literal on the path. The additional argument to |path| constitutes a
    way to get hold of information stored about the element on the path.
    This templete will lead to a compiletime/runtime error when the path
    library does not provide the appropriate predicates.

    \begin{BoxedSample}
      templates(-Literal,path(X))%
    \end{BoxedSample}

  \item [template({\em Pattern}, extension({\em
      LiteralIndex}))]\index{Capri!extension template}\ 
    \\
    This template describes a subgoal {\em Pattern}\/ which is solved by
    unifying it with a complementary literal somewhere in the matrix. The
    other subgoals of the clause containing the complemntary literal are left
    as residues. The argument {\em LiteralIndex}\/ is unified with the index
    of the complementary literal.

    \begin{BoxedSample}
      templates(--Literal,extension(C-L))%
    \end{BoxedSample}

  \item [template({\em Pattern}, extension)]\index{Capri!extension template}\ 
    \\
    This template acts like the pattern above, but not literal index is
    used/returned.

	
    \begin{BoxedSample}
      templates(--Literal,extension)%
    \end{BoxedSample}

  \item [template({\em Pattern}, residue)]\index{Capri!residue template}\ 
    \\
    A residue is directly translated into a call to the associated
    procedure. I.e. The goal will be solved with any means provided by the
    calculus in action.

    \begin{BoxedSample}
      templates((A=B),residue)%
    \end{BoxedSample}

  \item [template({\em Pattern}, neg\_residue)]\index{Capri!neg residue
      template}\ 
    \\
    A residue is directly translated into a call to the associated procedure
    after it has been negated. I.e. The goal will be solved with any means
    provided by the calculus in action.

    \begin{BoxedSample}
      templates((A=B),neg\_residue)%
    \end{BoxedSample}

  \item [template({\em Pattern}, {\em ListOfSelectors})]\index{Capri!list of
      templates}\ 
    \\
    This template describes a subgoal which is solved by applyig one of the
    selectors in {\em ListOfSelectors}. This list acts like a disjunction of
    possible selectors. {\em ListOfSelectors} is a list of simple selectors
    described above.

    The following example is used to write two disjoint descriptors --- which
    differ in this one template only --- in a single descriptor.

    \begin{BoxedSample}
      templates(Literal,[extension(E),path])%
    \end{BoxedSample}


  \item [template({\em PatternList}, {\em Selector})]

  \item [template({\em PatternList}, map({\em
      Selector,\ldots,Selector}))]\index{Capri!map template}\ 
    \\
    This template describes a list of patterns. Each pattern may use any of
    the selectors given in the map, but each element of the map is used
    exactly once. Thus |map| has the same number of arguments as {\em
      PatternList} has elements.

    \begin{BoxedSample}
      templates([L1,L2,L3],map([extension,path],goal,residue)%
    \end{BoxedSample}


  \item [some\_templates({\em Pattern},  
    {\em ListOfSelectors})]\index{Capri!some templates}\ 
    \\

    \begin{BoxedSample}
      some\_templates(Literal,[extension(E),path])%
    \end{BoxedSample}

  \item [some\_templates({\em Pattern},  
			 {\em ListOfSelectors},  
			 {\em Predicate})]\index{Capri!some templates}

  \item [some\_templates({\em Pattern}, 
    {\em ListOfSelectors},
    {\em Limit})]\index{Capri!some templates}\ 
    \\

    \begin{BoxedSample}
      some\_templates(Literal,[extension(E),path],23)%
    \end{BoxedSample}
    
    \begin{BoxedSample}
      some\_templates(Literal,[extension(E),path],1-7)%
    \end{BoxedSample}
    
    \begin{BoxedSample}
      some\_templates(Literal,[extension(E),path],at\_least(3))%
    \end{BoxedSample}


  \item [call({\em Prolog\_Code})]\index{Capri!call}\ 
    \\
    The given {\em Prolog\_Code}\/ is executed during the compilation.  This
    feature can be used to check certain conditions and omit the generation of
    code if those conditions do not hold by failing.  On the other side this
    instruction can be used to compute things not accessible in the current
    phase of implementation.
    
    \begin{BoxedSample}
      call(writeln('\%\%\% Checkpoint'))%
    \end{BoxedSample}

  \item [constructor({\em Prolog\_Code})]\index{Capri!constructor} \ 
    \\
    The given {\em Prolog\_Code}\/ is integrated into the target program.

    \begin{BoxedSample}
      constructor(writeln('Hello world.'))%
    \end{BoxedSample}


  \item [get(path, {\em Path})]\index{Capri!get path}\ 
    \\
    {\em Path}\/ is unified with the current path.

    This can be used to get the current path in order to restore it
    later. Such a situation is shown in the next example:

    \begin{BoxedSample}
      get(path,Path),
      template(Literal,extension),
      use\_path(Path),
      template(AnotherLiteral,residue)%
    \end{BoxedSample}
 

  \item [get(depth, {\em Depth})]\index{Capri!get depth}\ 
    \\
    {\em Depth}\/ is unified with the current depth.

    Note that the depth is not neccesarily a number. It may also be a Prolog
    term which is used by the search strategy to perform its task.

  \item [get(functor({\em Pred}), {\em Functor/Arity})]%
    \index{Capri!get functor}\ 
    \\
    This instruction unifies the literal {\em Pred}\/ with the literal
    having the functor {\em Functor}\/ and the arity {\em Arity}. This can be
    used to extract the functor and arity from a literal --- which has an
    additional sign. Alternatively it can be used to construct a literal. In
    this case the sign is |++|.

    The following example will unify |F/A| with |p/2|:
    \begin{BoxedSample}
      get(functor(++p(X,f(23)),F/A))%
    \end{BoxedSample}

    The following example will unify |L| with |++q(_,_,_)|:
    \begin{BoxedSample}
      get(functor(L,q/3))%
    \end{BoxedSample}

  \item [get(literals, {\em ListOfLiterals})]\index{Capri!get literals}\ 
    \\
    {\em ListOfLiterals}\/ is unified with the list of all literals occuring
    in the current matrix. The complete literals --- including sign and
    arguments --- are returned.


  \item [get(predicates, {\em ListOfPredicates})]\index{Capri!get predicates}\
    \\
    {\em ListOfPredicates} is unified with the list of (signed) functors. Each
    element has either the form |(-F)/A| for negative literals or |F/A| for
    positive literals, where |F| is a functor and |A| its arity.

    Consider a matrix containing only positive and negative occurrences of the
    predicate p/1. Then in the following example |PREDS| is unified with {\tt
      [(-p)/1,p/1]}.
    \begin{BoxedSample}
      get(predicates,PREDS)%
    \end{BoxedSample}

%  \item [get(functors, {\em ListOfFunctors})]\index{Capri!get functors}\ 
%    \\
%    {\em ListOfFunctors}

  \item [get(contrapositive({\em Index}), {\em
      Contrapositive})]\index{Capri!get contrapositive}\ \\
    {\em Contrapositive}\/ is unified with the contrapositive having the index
    {\em Index}. This can be used to get a contrapositive with a specified
    index or to get all contrapositives --- upon backtracking.

  \item [get(solved\_goals, {\em ListOfGoals})]\index{Capri!get solved goals}\
    \\
    {\em ListOfGoals}\/ is unified with the list of the goals already solved.

    {\bf NOT YET!}

  \item [get(open\_goals, {\em ListOfGoals})]\index{Capri!get open goals}\ 
    \\
    {\em ListOfGoals}\/ is unified with the list of open goals.

    {\bf NOT YET!}

  \item [put\_on\_path({\em Literal},{\em Info})]\index{Capri!put on path}\ 
    \\
    {\em Literal + Info }\/ is put on the path at this place. To prevent the
    automatic placement of a literal on the path the option
    |ProCom:automatic_put_on_path| can be used.

    \begin{BoxedSample}
      get(path,Path),
      get(depth,Depth),
      put\_on\_path(SomeLiteral,Depth),
      template(Literal,extension),
      use\_path(Path)%
    \end{BoxedSample}
 
  \item [put\_on\_path({\em Literal})]\index{Capri!put on path}\ 
    \\
    {\em Literal}\/ is put on the path at this place. To prevent the automatic
    placement of a literal on the path the option
    |ProCom:automatic_put_on_path| can be used. If no info is given (see
    above) then a new variable is used instead.

    \begin{BoxedSample}
      get(path,Path),
      put\_on\_path(SomeLiteral),
      template(Literal,extension),
      use\_path(Path)%
    \end{BoxedSample}
 
  \item [use\_path({\em Path})]\index{Capri!use path}\ 
    \\
    {\em Path}\/ is used as the path from this time on.  Note that you can not
    rely on any specific format of the path. The only save value for {\em
      Path} is the value of |get/path| or |empty_path|

    \begin{BoxedSample}
      get(path,Path),
      template(Literal,extension),
      use\_path(Path),
      template(AnotherLiteral,residue)%
    \end{BoxedSample}
 
\end{description}

To end this part we will show a complete \CaPrI{} descriptor module. It
implements the well known model elimination calculus. The first descriptor
encodes the reduction step and the second descriptor encodes the extension
step. In both cases the appropriate information for the proof tree is
provided. 

\begin{BoxedSample}
:- module(my\_own\_calculus).
:- compile(capri).

descriptor
        proof(reduction(Pred)),
        template(Pred,goal),
        template(-Pred,path).

descriptor
        proof(extension(-Pred)),
        template(Pred,goal),
        template(-Pred,extension).
\end{BoxedSample}

\iffalse
\subsection{Before and After Descriptors}

\begin{description}
  \item [start\_descriptor]\ \\
	This descriptor is evaluated once before ...

  \item [end\_descriptor]\ \\
	This descriptor is evaluated once after ...
\end{description}

\subsection{Static Reordering}

These features are not implemented yet. 

\begin{description}
  \item [compare\_literals(L1,L2).]\ \\
	Reordering literals

  \item [compare\_clauses(C1,C2).]\ \\
	Reordering clauses
\end{description}

\fi


\subsection{Testing and Modifying Options}

A certain prover may require some options to be set to given values or fit in
a given scheme. E.g. the target language has to be Quintus prolog since a
library is only written for this dialect.

To accomplish this instructions are provided to test and modify options. For a
detailed description which options are predefined see section
\ref{sec:options}.

\begin{description}
  \item [force\_option({\em Option}, {\em
  ListOfValues}).]\index{force\_option}\ \\
	This instruction describes which values of options are allowed. The
	elements os the list {\em ListOfValues} are unified with the actual
	value of {\em Option} until a match is found. 

	If no candidate unifies then the first element of the list is taken as
	the value of {\em Option}.

	This directive is evaluated before the compilation is started.
	I.e. the matrix has already been read and no preprocessing has been
	performed. 

	As an example consider the following instruction:

\begin{BoxedSample}
  force\_option(prolog, [quintus]).
\end{BoxedSample}

	This instruction forces the generation of Prolog code in the target
	language Quintus Prolog.

  \item [require\_option({\em Option}, {\em ListOfValues}).]%
	\label{require_option}\ \\
	This instruction describes requirements for options without the
	possibility to revert the option to a default value. 
	If one of these instructions is incompatible with the actual value of
	an option then no compilation is performed.

	For instance consider the following instruction which forces that the
	option |equality| is off.

\begin{BoxedSample}
  require\_option(equality, [off]).
\end{BoxedSample}

\end{description}


\subsection{Adjusting the Linker}

The linker tries to add definitions for missing Prolog predicates. For this
purpose it scans libraries and adds those libraries containing the appropriate
definitions. 

It can be desirable to replace given libraries, e.g. to use an improved
version of the debugger. This can partially be done by modifying options. This
section provides instructions to perform tasks which can hardly be solved by
modifying options.

\begin{description}
  \item [require\_predicate({\em Predicate}).]\index{require\_predicate}\ \\
	This instruction tells the linker that the predicate {\em Predicate}\/
	is required and a appropriate definition should be linked. {\em
	Predicate}\/ is a predicate specification of the form {\em
	Name/Arity}\/ like in the following example

\begin{BoxedSample}
  require(paramodulate/4).
\end{BoxedSample}

  \item [library\_file({\em File}).]\index{library\_file}\ \\
	This instruction tells the linker to consider the file {\em File} to
	find additional Prolog code.

\begin{BoxedSample}
  library\_file(my\_debugger).
\end{BoxedSample}

  \item [library\_path({\em ListOfDirectories}).]\index{library\_path}\ \\
	This instruction prepends the list of directories {\em
	ListOfDirectories} in front of the search path for libraries as in

\begin{BoxedSample}
  library\_path(['../my\_trash']).
\end{BoxedSample}

  \item [provide\_definition({\em Code}).]\index{provide\_definition}\ \\
	This instruction provides the linker with additional Prolog code which
	may be included. The {\em Code}\/ is a list of Prolog clauses which
	are added to the target program when required.

	This is only recommended for small pieces of Prolog. Consider the use
	of a library instead.
\end{description}



\chapter{Administration of \ProCom}

%------------------------------------------------------------------------------
\section{Creating a Stand-Alone Executable}



\subsection{Installing \ProCom}\label{sec:installation}

The \ProCom{} distribution comes in a single tar file {\sf procom-{\em
VV}.tar.Z} where {\em VV} stands for a version number. First, you have to
choose a place where to install \ProCom. Change the current directory to this
directory and execute the shell command

|  uncompress < |{\sf {\em SomeDir/}procom-{\em VV}.tar.Z} {\tt\char"7C} | tar -xvf -|

where {\em SomeDir}\/ is the directory where the distribution file is located.
This command generates a directory named {\sf ProCom} which contains the
distribution files.


Run the shell command 

|  make config|

to generate the configuration file. This command will try to guess the place
where \ProCom{} has been installed. Unfortunately sometimes this may fail and
may produce wrong pathnames in {\sf config.pl}. Thus you should check this file
and change the value of |HERE| in {\sf Makefile} appropriately.

Finally run the shell command

|  make procom|

This will generate an \eclipse{} saved state named {\sf procom}. This file is
executable and should be placed on your search path. Obviously this step
requires \eclipse{} to be installed and accessible under the name |xeclipse|.

The saved state |procom| will try to automatically start an appropriate
interface. If it is desirable to use the Prolog interface you can make a saved
state |procom.st| which just contains the preloaded Prolog files without
starting anything automatically. This is done with the command

|  make procom.st|



\subsection{Recompiling \ProCom}\label{sec:recompile}

When you are recompiling \ProCom{} you have to consider some points.

\begin{itemize}
  \item Change the current directory to the installed \ProCom.
	We assume that \ProCom{} is properly installed.
  \item Edit the file {\sf ProCom/procom\_config.pl}\/ to reflect the provers
	which are loaded into the final executable
  \item Make sure that the libraries are located by absolute path names only.
	This is strongly recommended to allow any user from any directory to
	use \ProCom{} without problems.
  \item Run the shell command

	|  make PROCOM=|{\em name}

	where {\em name}\/ is the name of the final executable. When this
	command is finished without errors the executable {\em name}\/ can be
	tested.
\end{itemize}


\bibliographystyle{named}
\bibliography{gn-atp}

\end{document} %%%%%%%%%%%%%%%%%%%%%%%%%%%%%%%%%%%%%%%%%%%%%%%%%%%%%%%%%%%%%%%%
